\documentclass[../ana2u.tex]{subfiles}

\begin{document}
\setcounter{section}{10}
\section{Extrema}
\( \abb{f}{\R^2}{\R}, (x,y) \mapsto (x^2 + 2y^2) e^{-(x^2 + y^2)} \).

\begin{beh}[Die Mutter aller Extrema]
    \(f\) besitzt ein globales Minimum in \((0,0)\), globales Maximum in
    \((0, \pm 1)\) und keine weiteren lokalen Extrema.
\end{beh}
\begin{bew}
    \(f\) ist offensichtlich \( 2 \) mal stetig differenzierbar und 
    es gilt 
    \[ \ddxpartialboth{f}{x} (x,y) = (2x - 2x^3 - 4xy^2)e^{-x^2 + y^2} \]
    \[ \ddxpartialboth{f}{y} (x,y) = (4y - 2x^2y - 4y^3)e^{-x^2 + y^2} \]
    für \( (x,y) \in \R^2 \).
    Also betrachte 
    \[ \nabla f(x,y) = \begin{pmatrix} 0 \\ 0\end{pmatrix} \]
    \[ \Leftrightarrow x(2 - 2x^2 - 4y^2) = 0 \text{ und }
    y(4 - 2x^2 - 4y^2) = 0 \]
    \[ \Leftrightarrow (x = 0 = y) \vee (x = 0 \wedge 2 - 2y^2 = 0)
    \vee (1 - x^2 = 0 \wedge y = 0). \]
    Nullstellen des Gradienten sind also \( (0,0), (0, \pm 1) \) 
    und \( (\pm 1, 0) \). 
    Weiter: Wie sieht Hessematrix an diesen Punkten aus?
    \( \Rightarrow \) Minima, Maxima, indefinit etc.
\end{bew}
\end{document}