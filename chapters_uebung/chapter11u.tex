\documentclass[../ana2u.tex]{subfiles}

\begin{document}
\setcounter{section}{10}
\section{Extrema}
\( \abb{f}{\R^2}{\R}, (x,y) \mapsto (x^2 + 2y^2) e^{-(x^2 + y^2)} \).

\begin{beh}
    \(f\) besitzt ein globales Minimum in \((0,0)\), globales Maximum in
    \((0, \pm 1)\) und keine weiteren lokalen Extrema.
\end{beh}
\begin{bew}
    \(f\) ist offensichtlich \( 2 \) mal stetig differenzierbar und 
    es gilt 
    \[ \ddxpartialboth{f}{x} (x,y) = (2x - 2x^3 - 4xy^2)e^{-x^2 + y^2} \]
    \[ \ddxpartialboth{f}{y} (x,y) = (4y - 2x^2y - 4y^3)e^{-x^2 + y^2} \]
    für \( (x,y) \in \R^2 \).
    Also betrachte 
    \[ \nabla f(x,y) = \begin{pmatrix} 0 \\ 0\end{pmatrix} \]
    \[ \Leftrightarrow x(2 - 2x^2 - 4y^2) = 0 \text{ und }
    y(4 - 2x^2 - 4y^2) = 0 \]
    \[ \Leftrightarrow (x = 0 = y) \vee (x = 0 \wedge 2 - 2y^2 = 0)
    \vee (1 - x^2 = 0 \wedge y = 0). \]
    Nullstellen des Gradienten sind also \( (0,0), (0, \pm 1) \) 
    und \( (\pm 1, 0) \). 
    Weiter: Wie sieht Hessematrix an diesen Punkten aus?
    \( \Rightarrow \) Minima, Maxima, indefinit etc.
\end{bew}
\begin{bsp}
    \( \abb{f}{\R^2}{\R}, (x,y) \mapsto 2x^2 - 3xy^2 + y^4 \).\\
    Wir betrachten Geraden \((x,y) = t(a,b)\)
    für \(a, b \in \R\) durch den Nullpunkt.
    \[ g(t) := f(x(t), y(t)) = 2a^2t^2-3ab^2t^3+b^4t^4 \]    
    Es ist 
    \begin{align*}
        g'(t) &= 4a^2 t - 9 ab^2 t^2 + 4 b^4 t^3\\
        g''(t) &= 4a^2 - 18ab^2 t + 12 b^4 t^2
    \end{align*}
    Extremum in \(0\)?
    \[ g(0) = g'(0) = 0 \qquad g''(0) = 4a^2. \]
    Also hat \(g\) in \(0\) ein lokales Minimum und damit auch die 
    Einschränkung von \(f\) auf die Gerade.\\
    Aber \(f\) hat in \((0,0)\) kein lokales Minimum.
    \(f(0,0) = 0, f(x,y) = (y^2-x)(y^2-2x) \).\\
    Somit liegen in jeder Umgebung des Nullpunkts sowohl Punkte mit
    negativen als auch positiven Funktionswerten. 
\end{bsp}
\begin{bsp}
    Sei \( D := \overline{U_1}(0,0) = \set{x : \norm{x}_2 \leq 1}, 
    \abb{f}{D}{\R} \) definiert durch 
    \( f(x,y) = x^3-x^2+xy^2+y^2 \).\\
    Gibt es globale Extrema von \(f\) auf \(D\) und wenn ja,
    welche Punkte sind dies?\\
    Wir betrachten zunächst \( D^\circ = U_1(0,0) = \set{x : \norm{x}_2 < 1} \).
    Auf \( D^\circ \) ist \(f\) offensichtlich zweimal stetig differenzierbar.\\
    Die notwendige Bedingung für Extrema ist somit 
    \[ (\nabla f)(x,y) = 0 \]
    Für \((x,y) \in D \) gilt
    \begin{align*}
        f_x(x,y) &= 3x^2-2x+y^2\\
        f_y(x,y) &= 2y(x+1)\\
        f_xx(x,y) &= 6x-2
        f_{xy}(x,y) = f_{yx}(x,y) &= 2y\\
        f_{yy}(x,y) = 2(x+1)
    \end{align*}    
    Kritische Punkte von \(f\) sind gegeben durch 
    Lösungen von 
    \begin{align}
        3x^2 - 2x + y^2 &= 0 \\
        2x(x + 1) &= 0
    \end{align}
    Aus (2) folgt
    \[ 2y(x+1) = 0 \Rightarrow y = 0 \vee x = -1. \]
    Im Fall \(y=0\) folgt aus (1)
    \[ 3x^2 - 2x = 0 \Rightarrow x = 0 \vee x = \frac{2}{3} \].
    Im Fall \(x=-1\) folgt aus (1)
    \[y^2 = -5\].
    Somit sind alle Lösungen des Gleichungssystems (1), (2) 
    gegeben durch 
    \[ (x,y) \in \set{ (0,0), (\frac{2}{3}, 0) }. \]

    Für \((x,y) \in D\) gilt
    \[ H_f(x,y) = \begin{pmatrix}
        6x-2 & 2y\\
        2y & 2(x+1)
    \end{pmatrix} \]
    Insbesondere gilt also 
    \[ H_f(0,0) = \begin{pmatrix}
        -2 & 0 \\ 0 & 2
    \end{pmatrix} \text{ (indefinit)}, \quad
    H_f(\frac{2}{3}, 0) = \begin{pmatrix}
        2 & 0 \\ 0 & \frac{10}{3}
    \end{pmatrix}\text{ (pos. definit)}. \]
    Also in \((\frac{2}{3}, 0)\) ein lokales Minimum von \(f\).
    \[ f(\frac{3}{2}, 0) = -\frac{4}{27} \]
    Es bleibt noch die Funktion auf \(\delta D\) zu prüfen.
    Wir parametrisieren mit Hilfe der Funktion
    \[ \abb{\varphi}{[0,2\pi]}{\delta D}; t \mapsto (\cos t, \sin t) \]
    und betrachten
    \[ \abb{F}{[0,2\pi]}{\R} \]
    gegeben durch
    \[ F(t) = f(\varphi(t)) = \cos^3 t-\cos^2 t + \cos t \sin^2 t + \sin^2 t 
    = \cos t + \sin^2 t - \cos^2 t\]
    offensichtlich ist \(F\) unendlich oft differenzierbar
    \[ F'(t) = \sin t (4\cos 5 - 1) \]
    \[ F''(t) = -\cos 5 + 4\cos^2 t - 4 \sin^2 t  \]
    Es gilt also 
    \[ F'(t) = 0 \Rightarrow \sin t = 0 \vee \cos t = \frac{1}{4} \]
    Im Fall \( \sin t = 0 \) folgt 
    \[ F''(t) = -\cos t + 4 = \begin{cases}
        3, & t = 0\\
        5, & t = \pi
    \end{cases}. \]
    \(F\) hat also in \(t = 0\) und \(t = \pi\) lokale Minima, wobei
    \(F(0) = 0\) und \(F(\pi) = -2\) \\
    Im Fall \(\cos t = \frac{1}{4}\) gilt
    \[ F''(t) = - \frac{1}{4} + \frac{1}{4} - \frac{15}{4} 
    = - \frac{15}{4} < 0 \]
    Also hat \(F\) für die beiden \( \tilde{t} \in [0,2\pi] \) 
    mit \( \cos \tilde{t} = \frac{1}{4} \) ein lokales Maxima und
    \(F(\tilde{t}) = \frac{18}{16}\).\\
    Da \(F\) stetig und \([0,2\pi]\) kompakt ist, nimmt \(F\) auf \([0,2\pi]\)
    sein Minimum und Maximum an.
    \[ \underset{t\in [0,2\pi]}{\max} F(t) = \frac{18}{16}, \quad
    \underset{t\in [0,2\pi]}{\min} F(t) = -2. \]
    Da auch \(f\) stetig ist und \(D\) kompakt ist, nimmt \(f\) auf \(D\)
    sein globales Maximum und globales Minimum an.\\
    Da jedes globale Maximum auch lokales Maximum ist, muss \(f\) sein 
    globales Maximum auf \(\delta D\) annehmen 
    (es gibt keine lokalen Maxima von \(f\) in \(D^\circ)\)).
    \[ \max f(D) = \max f(\delta D) = \frac{18}{16} \]
    \[ \min f(D) = -2 \]
\end{bsp}
\end{document}