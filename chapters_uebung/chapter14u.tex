\documentclass[../ana2u.tex]{subfiles}

\begin{document}
\setcounter{section}{13}
\section{Blatt 13}
\begin{bsp}[Aufgabe 62]
    Sei \(K \subset R^d\) kompakt, \(\abb{f}{K}{K}\) mit
    \[ d(f(x), f(y)) < d(x,y), \abb{d}{\R^d \times \R^d}{\R^d} \]
    zu zeigen: \(f\) hat einen Fixpunkt.
\end{bsp}
\begin{bew}
    Wir definieren \(h(x) := d(f(x), x)\). Wir nehmen an \(h(x) > 0\)
    für alle \(x \in K\).\\
    Da \(f\) stetig ist, ist auch \(h\) stetig. Somit nimmt h auf der kompakten
    Menge \(K\) sein Minimum an. Sei \(\alpha := \underset{x \in K}{\min} h(x)\).
    Dann gilt \(\alpha > 0\). Außerdem gibt es ein \(\tilde{x} \in K\) mit
    \(h(\tilde{x}) = \alpha\).\\
    Per Voraussetzung gilt
    \[ d(f(f(\tilde{x})), f(\tilde{x})) < d(f(\tilde{x}), \tilde{x}) = h(\tilde{x}) 
    = \alpha \text{\Lightning} \]
    Dies widerspricht, dass \( \alpha \) das Minimum von \(h\) aud der Menge \(K\) ist.\\
    Somit muss ein \( x \in K \) existieren, sodass \(h(x) = 0\) ist.
\end{bew}
\begin{bsp}[Aufgabe 63 (i)]
    Seien \( k, l \in N, \abb{\gamma}{[0,1]}{\R^l} \) eine rektifizierbare Kurve, 
    \( \abb{f}{\R^l}{\R^k} \) Lipschitzstetig mit Konstante \( C > 0 \).\\
    Zu zeigen \( \abb{ f \circ g }{[0,1]}{\R^l} \) ist eine rektifizierbare Kurve 
    mit \( L( f \circ g ) \leq C L(\gamma) \).
\end{bsp}
\begin{bew}
    Sei \(n \in \N\) und \( Z = \set{t_0, t_1, \ldots, t_n} \) Zerlegung von \( [0,1] \)
    \[ L(f \circ g, Z) = \sum_{k=1}^n \norm{ f(\gamma(t_k)) - f(\gamma(t_{k-1})) }_2 
    \leq C \sum_{k=1}^n \norm{\gamma(t_k) - \gamma(t_{k-1})}_2 
    = C L(\gamma, Z). \]
    Bildet man das Supremum über alle Zerlegungen von \([0, 1]\) so erhält
    man die Rektifizierbarkeit von \(f \circ g\) und die Abschätzung
    \[L(f \circ g) \leq  CL(g)\]
\end{bew}
\begin{bsp}[Aufgabe 63 (ii)]
    a) Sei \(\abb{\gamma}{[-1, 1]}{\R^2}, \gamma(t) = \begin{pmatrix}
        t^2\\
        t^3
    \end{pmatrix}\) \\ 
    b) \( \abb{\gamma}{[0,1]}{\R^3}, \gamma(t) = \begin{pmatrix} 
        t^2 \cos t \\ 
        t^2 \sin t \\ 
        \frac{t^3}{3} 
    \end{pmatrix} \)
\end{bsp}
\begin{bew}
    a) \(\gamma\) ist stetig differenzierbar, somit rektifizierbar und mit der
    Formel über die Ableitung können wir die Bogenlänge berechnen
    \[ \gamma'(t) = \begin{pmatrix} 2t \\ 3t^2 \end{pmatrix} \]
    und somit 
    \[ \abs{\gamma}_2 = (4 t^2 + 9 t^4)^{1/2} \]
    für alle \( t \in [-1,1] \).
    \[ L(\gamma) = \integralx{\abs{\gamma(t)}_2 }{-1}{1}{t} 
    = 2 \integralx{ t (4 + 9t^2)^{1/2} }{0}{1}{t} \]
    \[ = \integralx{ (4 + 9s)^{1/2} }{0}{1}{s} = \frac{2}{27} (13^{3/2} - 8). \]
    b) Auch hier ist \( \gamma \) rektifizierbar, da es stetig differenzierbar ist.
    \[ \gamma'(t) = \begin{pmatrix}
        2 t \cos t - t^2 \sin t \\
        2 t \sin t + t^2 \cos t \\
        t^2
    \end{pmatrix} \]
    und somit 
    \[ \abs{\gamma(t)}_2 = (4t^2 + 2t^4)^{1/2} \] 
    wobei wir verwendet haben, dass 
    \[ \cos^2 t + \sin^2 t = 1. \]
    \[ L(\gamma) = \integralx{\abs{\gamma(t)}_2}{0}{1}{t} = \integralx{t(4 + 2t^2)^{1/2}}{0}{1}{t} \]
    \[ = \frac{1}{2} \integralx{(4 + 2s)^{1/2}}{0}{1}{s} 
    = \sqrt{6} - \frac{4}{3} \]
\end{bew}
\begin{bsp}
    \[ \gamma(t) := e^{-t} \begin{pmatrix} \cos t \\ \sin t \end{pmatrix}, \quad t \geq 0 \] 
    ist stetig differenzierbar, somit rektifizierbar.
    \begin{align*}
        L(\gamma) = \integralx{ \abs{ \ddxpartial{t} e^{-t}(\cos t, \sin t)^T }_2 }{0}{\infty}{t}\\
        &= \integral{ \sqrt{(-e^{-t})^2 (\sin^2 t + \cos^2 t)} }{0}{\infty}{t} \\
        &= \integralx{e^{-t}}{0}{1}{t} = 1.
    \end{align*}
    Nun 
    \[ \gamma(t) = g(t) \begin{pmatrix} \cos t \\ \sin t \end{pmatrix} \]
    \( t \geq 0, g(t) > 0 \) für alle \( t \in [0,\infty) \)
    \begin{enumerate}
        \item Muss rektifizierbar sein. 
        \item \(\integralx{\sqrt{\gamma'(t)^2}}{a}{b}{t}\)
    \end{enumerate}
    \(\Rightarrow g(t)\) stetig auf \([0, \infty)\) und \(\mathcal{C}^1\)
    auf \(0, \infty\).\\
    \(\abb{\gamma \in \mathcal{C}([a,b], \R^n)}{t}{L(\gamma\vert_{[a,t]})}\)
    a) \[\abb{\gamma} {[0,4\pi]}{\R^3}, \gamma(t) := \begin{pmatrix}
        \cos t \\ \sin t \\ \cosh t
    \end{pmatrix} \]
    b) \[\abb{\gamma}{[0,5]}{\R^2}, \tilde{gamma}(t) := \begin{pmatrix}
        3t2 \\ 2t^3
    \end{pmatrix} \]
    Da \(\gamma\) und \(\tilde{\gamma}\) beide stetig diffbar sind ist die zugehörige Bogenlängenfunktion
    \(\abb{S_\gamma}{0, 4\pi}{\R}\) bzw. \(\abb{S_{\tilde{\gamma}}}{[0,5]}{\R}\)    
    \begin{align*}
        S_\gamma(t) &= \integralx{\abs{\gamma'(\tau)}_2}{0}{1}{\tau} \\        
        S_{\tilde{\gamma}}(t) &= \integralx{\abs{\tilde{\gamma}'(\tau)}_2}{0}{1}{\tau}
    \end{align*}
    Wegen \( \gamma'(\tau) = \begin{pmatrix} -\sin \tau \\ \cos \tau \\ \sinh \tau \end{pmatrix} \) für alle \( \tau \in [0,4\pi] \), 
    \( \tilde{\gamma}'(t) = \begin{pmatrix} 6 \tau \\ 6 \tau^2 \end{pmatrix} \) für \( \tau \in [0,5] \).
    \begin{align*}
        \abs{\gamma'(t)}_2 &= \sqrt{ \sin^2 \tau + \cos^2 \tau + \sinh^2 \tau}\\
        &= \sqrt{1 + \sinh^2 \tau} = \sqrt{\cosh^2 \tau}\\
        &= \cosh \tau
    \end{align*}
    \[ S_\gamma (t) = \integralx{\cosh \tau}{0}{t}{\tau} 
    = [\sinh \tau]_0^t = \sinh t. \]    
    \[ \abs{ \tilde{\gamma}(\tau) }_2 = \sqrt{36 \tau^2 + 36 \tau^4} \]
    \[ S_{\tilde{\gamma}}(t) = \integralx{\sqrt{36\tau^2 + 36\tau^4}}{0}{t}{\tau} = 2(1 + t^2)^{3/2} - 2. \]
\end{bsp}
\begin{bsp}
    \begin{enumerate}
        \item \( \abb{f}{\R^2}{\R^2}, f(x,y) = \begin{pmatrix} e^x \\ xy \end{pmatrix} \)
        \item \( \abb{\gamma}{[0,2\pi]}{\R^2}; \gamma(t) = \begin{pmatrix} \cos t \\ \sin t \end{pmatrix} \)
    \end{enumerate}
    Berechne \( \int_\gamma f d\vec{x} \).
\end{bsp}
\begin{bew}
    (i) Da \(f\) stetig und \(\gamma\) stetig differenzierbar ist mit 
    \[ \gamma'(t) = \begin{pmatrix} -\sin t \\ \cos t \end{pmatrix} \]
    gilt 
    \[ \int_\gamma f(x,y)\; d(x,y) 
    = \integralx{ f(\gamma(t)) \abs{ \gamma'(t) }_2 }{0}{2\pi}{t} \]
    \[ = \integralx{ f(\cos t, \sin t) \begin{pmatrix} -\sin t \\ \cos t \end{pmatrix} }{0}{2\pi}{t} \]
    \[ = \integralx{ -e^{\cos t} \sin t + \sin t \cos^2 t }{0}{2\pi}{t} \]
    \[ = [e^{\cos t} - \frac{1}{3} \cos^3 t ]_0^{2\pi} 
    = (e - \frac{1}{3}) - (e - \frac{1}{3}) = 0. \]
    (ii) Sei \( \abb{f}{\R^3}{\R^3}, f(x,y,z) = \begin{pmatrix}  y\\-z\\x \end{pmatrix}\)
    \(f\) ist stetig, \(\gamma\) stetig differenzierbar mit
    \[ \gamma'(t) = \begin{pmatrix}
        \sinh t\\
        \cosh t\\
        \sinh t
    \end{pmatrix} \forall t \in [0, \ln 3] \]
    Also gilt:
    \begin{align*}
        \int_\gamma f(x,y,z) \; d(x,y,z)
        &= \integralx{ f(\sinh t, \cosh t, \sinh t) \begin{pmatrix} \cosh t \\ \sinh t \\ \cosh t \end{pmatrix} }{0}{\ln 3}{t}\\
        &= \integralx{ \sinh t \cosh t }{0}{\ln 3}{t}\\
        &= [t + \frac{1}{2}sinh^2(t)]^{ln(3)}\\
        &= \ln 3 + \frac{1}{2} \left(\frac{3 - \frac{1}{3}}{2}\right)^2 = \ln 3 + \frac{8}{9}
    \end{align*}
\end{bew}
\begin{bsp}
    (iii) \[\abb{f}{\R^2}{\R^2}; f(x,y) := \begin{pmatrix} x^2y \\ -y \end{pmatrix} 
    \abb{\gamma}{[0, 1]}{R^2}; \gamma(t) := \begin{pmatrix} t^2 \\ t^3 - 1 \end{pmatrix}\]
    Da \(f\) stetig und \(\gamma\) stetig differenzierbar mit
    \[ \gamma'(t) = \begin{pmatrix}
        2t\\3t^2
    \end{pmatrix} \forall t \in [0, 1] \]
    Somit gilt 
    \begin{align*}
        \int_\gamma f(x,y) \; d(x,y) &= \integralx{ f(\gamma(t)) \gamma'(t) }{0}{1}{t}\\
        &= \integralx{ f(t^2, t^3 - 1) \begin{pmatrix} 2t \\ 3t^2 \end{pmatrix} }{0}{1}{t} \\
        &= \integralx{ t^4(t^3 - 1)2t - (t^3 - 1)3t^2 }{0}{1}{t} \\
        &= \integralx{ 2t^8 - 5t^5 + 3t^2 }{0}{1}{t} \\
        &= \left[ \frac{2}{9}t^9 - \frac{5}{6}t^6 + t^3 \right]_0^1 = \frac{7}{18}.
    \end{align*}
\end{bsp}
\begin{bsp}[Aufgabe 67]
    Besitzt \(f\) ein Potenzial?
    \[ \abb{f}{\R^3}{\R^3}, f(x,y,z) = (y^2 + 2zx, z^2 + 2xy, x^2 + 2yz). \]
    Die Frage ist, ob eine Funktion \( \abb{\Phi}{\R^3}{\R} \) existiert, sodass 
    \[ \nabla \Phi(x,y,z) = f(x,y,z). \]
    Dann ist \(\Phi\) ein Potenzial von \(f\).
    \[ \Phi(x,y,z) = xy^2 + x^2 z + yz^2. \]
    \[ \nabla \Phi(x,y,z) = \begin{pmatrix} y^2 + 2xz \\ z^2 + 2xy \\ x^2 + 2yz \end{pmatrix} \]
    \[ f(x,y,z) = (y^2 + 2x^3 y x, 2y + z^3 x^2, y^2 + 3z^2 y x^2) \]
    hat kein Potential, da 
    \[ \partial_1 f_2(x,y,z) = 2xz^3 \neq 2y + 2xz^3 = \partial_2 f_1(x,y,z). \]
    Also erfüllt \(f\) die Integrationsbedingung nicht und besitzt kein Potenzial.
\end{bsp}
\end{document}