\documentclass[../ana2u.tex]{subfiles}

\begin{document}
\setcounter{section}{12}
\section{Taylor}

Sei \(G\subset \R^n\) offen, \(x, \xi \in G, h:= x-\xi\)
und \(\abb{f}{G}{\R} m\)-Mal stetig differenzierbar, dann heißt
das Polynom
\begin{align*}
    T_m(x) &= T_m(\xi+h) \\
    &= \sum_{\abs{\alpha} \leq m} 
    \frac{D^\alpha f(\xi)}{\alpha!}(x-\xi)^\alpha \\
    &= \sum_{k=0}^m \frac{1}{k!} f^{(k)}(\xi)(\underbrace{h, \ldots, h}_{k-\text{mal}})
\end{align*}
\[ \abb{f}{\R^3}{\R}, f(\vec{x}) = f(x,y,z) 
:= \sin x \cos y\; e^z \]
Entwicklungspunkt: \( (0,0) = \xi \).
\( f \) ist auf \( \R^3 \) beliebig oft stetig differenzierbar.
Es gilt \( f(0) = 0 \) also ist \(T_0(\vec{x}) = f(0) = 0\). \\
Für das \(1.\) Taylorpolynom brauchen wir
\[ \grad f(\vec{x}) = (\cos x \cos y, -\sin x \sin y, \sin x \cos y)^t e^z \]
und \( \grad f(0) = \begin{pmatrix} 1 \\ 0 \\ 0 \end{pmatrix} \).
Also 
\[ T_1(\vec{x}) = T_0(\vec{x}) + \scalarprod{\grad f(0)}{\vec{x}} = x \]
Für die zweite Ableitung bzw. die Hessematrix erhält man 
\[ H_f(0) = e^z
\begin{pmatrix}
    -\sin x \cos y & -\cos x \sin y & \cos x \cos y \\
    -\cos x \sin y & -\sin x \cos y & -\sin x \sin y \\
    \cos x \cos y & - \sin x \sin y & \sin x \cos y
\end{pmatrix}(0,0,0) = \begin{pmatrix}
    0 & 0 & 1\\
    0 & 0 & 0\\
    1 & 0 & 0
\end{pmatrix} \]
\[ T_2(\vec{x}) = \underbrace{T_0(\vec{x})}_{=0} + \underbrace{T_1(\vec{x})}_{=x} 
+ \frac{1}{2}(\vec{x})^t H_f(0) \vec{x} 
= x + \frac{1}{2}\begin{pmatrix}
    x & y & z
\end{pmatrix} \begin{pmatrix}
    0 & 0 & 1\\
    0 & 0 & 0 \\
    1 & 0 & 0
\end{pmatrix} \begin{pmatrix}
    x \\ y \\ z
\end{pmatrix} = x + xz \]
\begin{defi*}
    Sei \( U \subset \R^2 \) eine offene Menge und 
    \( \abb{f}{U}{\R}, (x,y) \mapsto f(x,y) \) eine 
    stetig differenzierbare Funktion.\\
    Die Höhenlinien von \(f\) sind 
    \[ N_f(c) := 
    \set{ (x,y) \in U: f(x,y) = c }, c \in \R. \]
    Sei \( (a,b) \in U, f(a,b) = c \) und 
    \[ \nabla f(a,b) \neq (0,0), \]
    d. h. \(\ddxpartialboth{f}{y}(a,b) \neq 0 \) und/oder 
    \(\ddxpartialboth{f}{x}(a,b) \neq 0\).\\
    Sei nun \(F(x,y) := f(x,y) - c\).\\ Wir verwenden Satz 17.1
    an und erhalten
    \begin{enumerate}
        \item Falls \( \ddxpartialboth{f}{y}(a,b) \neq 0 \), 
        so existieren Intervalle \( I_1, I_2 \subset \R \) 
        mit \( (a,b) \in I_2 \times I_2 \subset U \)
        und eine stetig differenzierbare Funktion \(\abb{g}{I_1}{I_2}\),
        sodass
        \[ N_f(c)  \cap (I_1 \times I_2) 
        = \set{(x,y) \in I_1 \times I_2: y = g(x)}\]
        \item Falls \(ddxpartialboth{f}{x}(a,b) \neq 0\), 
        so existieren Intervalle \(J_1, J_2 \subset \R \) mit
        \((a,b) \in J_1 \times J_2 \subset U\) und stetig 
        differenzierbare Funktion \(\abb{\psi}{J_2}{J_1}\), sodass
        \[ N_f(c) \cap (J_1 \times J_2) = \set{(x,y) \in J_1 \times J_2: x = \psi(y)} \]
    \end{enumerate}
\end{defi*}
\begin{bsp}
    \(\abb{f}{\R^2}{\R}, f(x,y) := x^2 +y^2 \)\\
    Es gilt \(\nabla f(x,y) = (2x, 2y)\)
    d. h. der Gradient verschwindet nur in \( (x,y) = (0,0) \).\\
    \[N_f(0) = \set{(x,y) \in \R^2 : x^2 + y^2 = 0} = \set{(0,0)} \]
    ist keine Linie im eigentlichen Sinne.\\
    Für \(c < 0\) ist \(N_f(c) = \emptyset\),
    für \(c < 0\) ist \(N_f(c)\) ein Kreis mit Radius \(sqrt(c)\) \\
    um den Nullpunkt.
    \(N_f(c)\) lässt sich durch die Vereinigung der folgenden
    vier Graphen beschreiben.
    \begin{align*}
        \Gamma_1 &:= \set{(x,y) \in \R^2: -\sqrt{c} < x < \sqrt{c}, y = \sqrt{c - x^2}} \\
        \Gamma_2 &:= \set{(x,y) \in \R^2: -\sqrt{c} < x < \sqrt{c}, y = -\sqrt{c - x^2}} \\
        \Gamma_3 &:= \set{(x,y) \in \R^2 : -\sqrt{c} < y < \sqrt{c}, x = \sqrt{c - y^2} } \\
        \Gamma_4 &:= \set{(x,y) \in \R^2: -\sqrt{c} < y < \sqrt{c}, x = -\sqrt{c - y^2}}
    \end{align*}    
\end{bsp}
\begin{bsp}
    Sei \(\abb{f}{\R^2 \times \R^2}{R^2}, 
    f(p,q,\lambda) = \lambda^2 + 2p\lambda + q = {(\lambda + p)}^2-(p^2 - q)\)\\
    Sind interessiert an der Menge 
    \(\set{(p,q, \lambda) \in \R^2 \times \R: f(p,q,\lambda) = 0}\)
    \[ G^{\pm} = \set{p,q, \lambda^\pm(p,q): p^2 > q)}, 
    \lambda^pm (p,q) = -p \pm \sqrt{p^2-q}\]
    \[ (\overline{G^+} \cap \overline{G^-}) = \set{(p,y,\lambda): p^2 = q, \lambda = -p} \]
    \[ N = G^+ \cup G^- \cup (\overline{G^+} \cap \overline{G^-}) \]    
    Sei nun 
    \[ \ddxpartialboth{f}{\lambda} (p_0, q_0, \lambda_0) 
    = 2(\lambda_0 + p_0) \neq 0 \Leftrightarrow p_0^2 - q-0 > 0\]
    Dann liegt \((p_0, q_0, \lambda_0)\) in einem der Graphen \(G^+\) bzw. \(G^-\)
    und \(N\) ist in einer Umgebung \(U \times V\) als Graph von 
    \(\lambda^+\) oder \(\lambda^-\) darstellbar (Satz 17.1)
    Gilt hingegen 
    \[\ddxpartialboth{f}{\lambda}(p_0,q_0,\lambda_0) = 2(p_0-\lambda_0) = 0\]
    \[ \Leftrightarrow p_0^2 - q_0 = 0 \]
    Hier macht Satz 17.1 keine Aussage! \\
    In der Tat gilt, dass sich die Menge 
    \(N\) in keiner Umgebung von \((p_0, q_0, \lambda_0)\) als Graph 
    \(\lambda = \lambda(p,q)\) darstellen lässt:
    \begin{enumerate}
        \item Für \(p^2 < q\) hat die Gleichung überhaupt keine Lösung. 
        \item Für \(p^2 = q\) gibt es genau eine Lösung.
        \item Für \(p^2 > q\) existieren zwei Lösungen \(\lambda^\pm(p,q)\)
    \end{enumerate}    
\end{bsp}
\subsection{Geometrische Anwendung des Satzes 17.1}
Sei \(f \subset C^1(\R^2)\) und \(z_0 \in \R\), so kann i. A. die
Niveaumenge
\[ M = \set{(x,y) \in \R^2: f(x,y) = z_0} \]
Punkte enthalten, die lokal nicht wie eine Linie aussehen 
(z. B. Kreuzungspunkte oder isolierte Punkte).
Ist aber \(Df(x,y) \neq 0\) für alle \((x,y) \in M\), so ist
\(M\) nach Satz 17.1 in der Nähe jedes Punktes als
\(C^1\)-Graph über \(x\)-Achse oder der \(y\)-Achse darstellbar
und damit nicht wirklich eine Höhenlinie im strengen Sinne des Wortes.

Teilmengen des \( \R^n \), die lokal aussehen wie ein Unterraum 
heißen Unterraummannigfaltigkeiten.
\begin{defi*}[Diffeomorphismus]
    Eine Abbildung \( \abb{f}{U}{V} \) zwischen 
    offenen Mengen \( U,V \subset \R^n \) Diffeomorphismus 
    der Klasse \( C^r \), wobei \( r \in \N \cup \set{\infty} \), 
    falls
    \begin{center}
        \( f \) bijektiv ist und sowohl \(f\) als auch 
        \( f^{-1}\ r \)-mal differenzierbar sind.
    \end{center}
\end{defi*}
\begin{bsp}
    Sei \(I \subset \R\) ein offenes Intervall. Ist \(f \in C^1(I)\)
    mit \(f' > 0\) auf ganz \(I\) (bzw. \(f' < 0\) auf ganz \(I\)),
    so ist \(J := f(I)\) ein offenes Intervall und \(\abb{f}{I}{J}\) 
    ist Diffeomorphismus.
    Dies folgt, da 
    \begin{itemize}
        \item \( \abb{f}{I}{J} \) ist bijektiv, da 
        streng monoton wachsend (bzw. fallend). Nach dem ZWS muss
        \(J\) ein offenes Intervall sein.
        \item Die Umkehrabbildung \(g = f^{-1}\) ist differenzierbar mit
        \[ g' = \frac{1}{f' \circ g} \]
        insbesondere ist \(g\) von der Klasse \(C^1\).
        \end{itemize}
\end{bsp}
\begin{bsp}[Umkehrung]
    Ist \(\abb{f}{I}{f(I)}\) ein \(C^1\)-Diffeomorphismus
    mit Umkehrabbildung \(\abb{g}{f(I)}{I}\), so ergibt die Kettenregel
    \[ g(f(x)) = x \Rightarrow g'(f(x))f'(x) = 1 \Rightarrow f'(x) \neq 0 \]     
    Nach dem ZWS gilt entweder \(f' < 0\) auf
    ganz \(I\) ofrt \( f' < 0 \) auf ganz \(I\).
    Insbesondere ist \(f\) streng monoton.
\end{bsp}
\begin{bsp}[Gegenbeispiel]
    \(\abb{f}{(-1, 1)}{(-1, 1)}; f(x) = x^3\) ist bijektiv, 
    streng monoton wachsend und stetig differenzierbar.\\
    Aber kein \(C^1\)-Diffeomorphismus, denn es gilt \(f'(0) = 0\)\\
    Die Umkehrabbildung 
    \[ \abb{g}{(-1,1)}{(-1,1)}, g(y) = \begin{cases}
        \sqrt[3]{y}, &\text{für } y \geq 0 \\
        \sqrt[3]{-y}, &\text{für } y < 0
    \end{cases}\]
    ist im Punkt \(y = 0\) nicht differenzierbar.
\end{bsp}
\begin{bsp}[Inversion]
    Die Inversion an der Sphäre \(S^{n-1} = \set{x \in \R : \abs{x} = 1}\) ist der 
    Diffeomorphismus
    \[ \abb{f}{\R^n\setminus \set{0}}{\R^n \setminus \set{0}}; f(x) = \frac{x}{\abs{x}^2} \]
    Die Abbildung \(f = f^-1\) ist von der Klasse \(C^\infty\). Die 
    beschränkte Menge \(U = \set{x \in \R^n: 0 < \abs{x} < 1}\) wird auf die
    unbeschänkte Menge
    \[ V = \set{y \in \R^n: \abs{y} > 1} \]
    abgebildet.
\end{bsp}
\begin{defi}[\(m\)-dimensionale Untermannigfaltigkeiten]
    Sei \(1 \leq m \leq n \). Eine Menge \( M \subset \R^n \) 
    heißt \(m\)-dimensionale Untermannigfaltigkeit des \(\R^n\) der 
    Klasse \( C^r \), wobei \( r \in \N \cup \set{\infty} \), falls gilt
    \begin{itemize}
        \item zu jedem \(p \in M\) gibt es eine offene Umgebung \(\Omega \subset \R^n\)
        und einen \(C^r\)-Diffeomorphismus \(\abb{\phi}{\Omega}{\phi(\Omega)}\) mit
        \[ \phi(M \cap \Omega) = (\R^m \times \set{0}) \cap \phi(\Omega) \]
    \end{itemize}
\end{defi}
\begin{bem}
    Den Diffeomorphismus nennt man auch (lokale) 
    Plättung von \( M \) oder auch Karte von \(M\).
\end{bem}
\begin{satz}[Untermannigfaltigkeitskriterium]
    Sei \( M \subset \R^n \) und \( m + k = n \). Dann 
    sind die folgenden Aussagen äquivalent.
    \begin{enumerate}
        \item \(M\) ist eine \(m\)-dimensionale Untermannigfaltigkeit der Klasse \(C^r\)
        \item Niveaumengenkriterium: Zu jedem \(p \in M\) gibt es eine offene
        Umgebung \(\Omega \subset \R^n\) und eine Funktion \(f \in C^r(\Omega, \R^k)\),
        sodass 
        \[ M \cap \Omega = f^{-1}(0) \]
        und 
        \[ \mathrm{rk} Df = k \text{ auf } \Omega \]
        \item Graphenkriterium: Zu jedem \(p \in M\) gibt es eine offene
        Umgebung \(U \times V \subset \R^m \times \R^k\) und
        \(g \in C^r(U, V)\), sodass nach geeigneter 
        Permutation der Koordinaten gilt:
        \(M \cap (U \times V) = \set{(x, g(x)): x \in U}\)
    \end{enumerate}
\end{satz}
\begin{bew}
    1. \( \Rightarrow \) 2.: Nach 1. gibt es zu jedem \( p \in M \) 
    eine lokale Plättung \( \abb{\phi}{\Omega}{\phi(\Omega)} \) der Klasse 
    \( C^r \) mit \( p \in \Omega \).\\
    Sei \(\abb{\pi^\bot}{\R^m \times \R^k}{\R^k}\) die Projektion auf den 
    zweiten Faktor. Sei \( f := \pi^\bot \circ \phi \) und \( q \in \Omega \),
    \begin{align*}
        f(q) = 0 &\Leftrightarrow \phi(q) \in \R^m \times \set{0}\\
        &\Leftrightarrow q \in \phi^{-1}(\R^m \times \set{0}) = M \cap \Omega
    \end{align*}
    Außerdem gilt: 
    \[ \mathrm{rk} Df = \mathrm{rk}(\pi^\bot \circ D\phi) = k \]
    und \(f \in C^r(\Omega, \R^k)\) \\
    2. \( \Rightarrow \) 3.: Nach evtl. Permutation der Koordinaten 
    ist \( Dg f(p) \) invertierbar, wobei \( (x,y) \in \R^n - \R^m \times \R^k \) 
    und 3. folgt nun aus Satz 17.1.\\
    3. \( \Rightarrow \) 1.: Es gelte die Graphendarstellung 
    ohne Permutation der Koordinaten. Sei weiter \( \Omega = U \times V \) und 
    \( \abb{\phi}{\Omega}{\phi(\Omega)}, 
    \phi(x,y) = (x, y - g(x)) \).\\
    Dann ist \( \phi \in C^r(\Omega, \R^m) \) injektiv 
    und es gilt 
    \[ D\phi(x,y) = \begin{pmatrix}
        E_m & 0 \\
        -Dg(x) & E_k
    \end{pmatrix} (E_j = id \in \R^j). \]
    \[ \Rightarrow \det D\phi(x,y) = 1 \;\forall (x,y) \in \Omega. \]
    Also ist \(\abb{\phi}{\Omega}{\phi(\Omega)} \) ein Diffeomorphismus
    der Klasse \(C^r\) nach dem Satz über inverse Funktionen. 
    Da \((x,y) \in M \cap \Omega\) genau dann, wenn \(y = g(x)\),
    also \(\phi(x,y) \in \R^k \times \set{0}\), ist die in 1. verlangte
    lokale Plättung gefunden.
\end{bew}
\end{document}