\documentclass[../ana2u.tex]{subfiles}

\begin{document}
\setcounter{section}{12}
\section{Taylor}

Sei \(G\subset \R^n\) offen, \(x, \xi \in G, h:= x-\xi\)
und \(\abb{f}{G}{\R} m\)-Mal stetig differenzierbar, dann heißt
das Polynom
\begin{align*}
    T_m(x) &= T_m(\xi+h) \\
    &= \sum_{\abs{\alpha} \leq m} 
    \frac{D^\alpha f(\xi)}{\alpha!}(x-\xi)^\alpha \\
    &= \sum_{k=0}^m \frac{1}{k!} f^{(k)}(\xi)(\underbrace{h, \ldots, h}_{k-\text{mal}})
\end{align*}
\[ \abb{f}{\R^3}{\R}, f(\vec{x}) = f(x,y,z) 
:= \sin x \cos y\; e^z \]
Entwicklungspunkt: \( (0,0) = \xi \).
\( f \) ist auf \( \R^3 \) beliebig oft stetig differenzierbar.
Es gilt \( f(0) = 0 \) also ist \(T_0(\vec{x}) = f(0) = 0\). \\
Für das \(1.\) Taylorpolynom brauchen wir
\[ \grad f(\vec{x}) = (\cos x \cos y, -\sin x \sin y, \sin x \cos y)^t e^z \]
und \( \grad f(0) = \begin{pmatrix} 1 \\ 0 \\ 0 \end{pmatrix} \).
Also 
\[ T_1(\vec{x}) = T_0(\vec{x}) + \scalarprod{\grad f(0)}{\vec{x}} = x \]
Für die zweite Ableitung bzw. die Hessematrix erhält man 
\[ H_f(0) = e^z
\begin{pmatrix}
    -\sin x \cos y & -\cos x \sin y & \cos x \cos y \\
    -\cos x \sin y & -\sin x \cos y & -\sin x \sin y \\
    \cos x \cos y & - \sin x \sin y & \sin x \cos y
\end{pmatrix}(0,0,0) = \begin{pmatrix}
    0 & 0 & 1\\
    0 & 0 & 0\\
    1 & 0 & 0
\end{pmatrix} \]
\[ T_2(\vec{x}) = \underbrace{T_0(\vec{x})}_{=0} + \underbrace{T_1(\vec{x})}_{=x} 
+ \frac{1}{2}(\vec{x})^t H_f(0) \vec{x} 
= x + \frac{1}{2}\begin{pmatrix}
    x & y & z
\end{pmatrix} \begin{pmatrix}
    0 & 0 & 1\\
    0 & 0 & 0 \\
    1 & 0 & 0
\end{pmatrix} \begin{pmatrix}
    x \\ y \\ z
\end{pmatrix} = x + xz \]
\end{document}