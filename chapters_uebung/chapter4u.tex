\documentclass[../ana2u.tex]{subfiles}
\begin{document}
\setcounter{section}{3}
\section{Riemann Integral}
\begin{defi}
    Sei \( \abb{f}{[a,b]}{\R} \) eine beschränkte Funktion und 
    \( Z := \set{a=x_0, x_1, \dots, x_{n-1}, x_n = b} \)
    Zerlegung von \([a,b]\)    
    Dann heißt 
    \[ O(z) := \sum_{i=1}^n (x_i - x_{i-1}) 
    \underset{x\in(x_{i-1}, x_i)}{\sup} f(x) \]
    die Obersumme zur Zerlegung \(Z\) und 
    \[ U(Z) = \sum_{i=1}^n (x_i - x_{i-1}) 
    \underset{x\in(x_{i-1},x_i)}{\inf} f(x)  \]
    Durch Bildung des Supremums bzw. Infimums
    über alle Zerlegungen von \( [a,b] \) erhält man
    \[ \int_a^{*b} f(x) \dx 
    = \integral{f(x)}{a}{b} := \inf O(z) 
    \text{ Oberintegral} \]
    \[ \int_{*a}^b f(x) \dx 
    = \integral{f(x)}{a}{b} := \sup O(z) 
    \text{ Unterintegral} \]
    Gilt
    \[ \int_{*a}^b f(x) \dx = \int_a^{*b} f(x) \dx \]
    so heißt dies Riemann Integral von \(f\)
    \[ \integral{f(x)}{a}{b} \]
    Sei nun \( \abb{f}{[a,b]}{\R} \) stetige Regelfunktion. \\
    \( \Rightarrow f \) gleichmäßig stetig auf \([a,b]\).\\
    Sei \( Z=\set{x_0, \dots, x_n} \) von \([a,b]\)
    \[ \varphi_i(x) = \begin{cases}
        \underset{x\in[x_{i-1}, x_i]}{\sup} f(x), \; x_{i-1} < x \leq x_i \\
        0, \text{ sonst}
    \end{cases}\]
    \[ \psi_i (x) = \begin{cases}
        \underset{x \in [x_{i-1}, x_i]}{\inf} f(x), \; x_{i-1} < x < x_i\\
        0 \text{ sonst}
    \end{cases} \]
    Und man sieht, da \( f \) gleichmäßig stetig auf 
    \( [a,b] \) ist, dass 
    \[ ||f - \sum_{i=1}^n \varphi_i ||_\infty < \varepsilon \]
    für \( \abs{x_{i-1} - x_i} < \delta \).
\end{defi}

\end{document}