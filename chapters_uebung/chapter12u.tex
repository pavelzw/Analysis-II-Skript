\documentclass[../ana2u.tex]{subfiles}
\begin{document}
\setcounter{section}{11}
\section{Differenzieren von Integralen}
\[ f(x) = \integralx{g(x,y)}{a(x)}{b(x)}{y} \tag{\(*\)} \]
\[ a(x), b(x) \in \mathcal{C}^1(\R), 
g \in \mathcal{C}^1(\R^2) \]

\begin{beh}
    \[ f'(x) = b'(x)g(x, b(x)) - a'(x)g(x,a(x))
    + \integralx{\ddxpartial{x} g(x,y)}{a(x)}{b(x)}{y} \]
\end{beh}
\begin{bew}
    Wir definieren 
    \[ F(x_1, x_2, x_3) 
    = \integralx{g(x_3, y)}{a(x_2)}{b(x_1)}{y}. \]
    Dann gilt \( f(x) = F(x,x,x) \).\\
    Unter der Annahme, dass \( F \ \mathcal{C}^1 \) ist, 
    gilt
    \[ g'(x) = \ddxpartialboth{F}{x_1}(x,x,x) 
    + \ddxpartialboth{F}{x_2}(x,x,x) 
    + \ddxpartialboth{F}{x_3}(x,x,x) \]
    Wir betrachten zunächst \(\ddxpartialboth{F}{x_1}, 
    \ddxpartialboth{F}{x_2}\).
    Wir verwenden die \(1-\)Dimensionale Kettenregel und HDI
    \[ F(x_1, x_2, x_3) = G(b(x_1), a(x_2),x_3) \]
    mit 
    \[ G(b,a,x_3) = \integralx{g(x_3y)}{a}{b}{y} \]
    Somit
    \[ \ddxpartialboth{F}{x_1}(x_1,x_2,x_3) 
    = \ddxpartialboth{G}{b} \ddxpartialboth{b}{x_1} = b'(x_1)g(x_3, b(x_1)) \]
    \[ \ddxpartialboth{F}{x_2}(x_1,x_2,x_3) = \ddxpartialboth{G}{a} \ddxpartialboth{a}{x_2} = -a'(x_2)g(x_3,a(x_2)) \]
    Als Komposition stetiger Funktionen sind \(\ddxpartial{x_1}F, \ddxpartial{x_2}F\)
    stetig.
    Merke: \( x_1, x_2 \) sind fest, wenn wir 
    \( \ddxpartial{F}{x_3} \) betrachten, also 
    können wir die Notation vereinfachen.\\
    Wir müssen also zeigen , dass
    \[ \ddxpartial{x} \integralx{g(x,y)}{a}{b}{y} 
    = \integralx{\ddxpartialboth{g}{x} (x,y)}{a}{b}{y} \]
    und dass dies wieder eine stetige Funktion ist.
\end{bew}
\end{document}