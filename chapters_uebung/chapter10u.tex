\documentclass[../ana2u.tex]{subfiles}

\begin{document}
\setcounter{section}{9}
\section{Multilineare Abbildungen}
\( \abb{D^2 f = D(Df)}{U}{\mathscr{L}(\R^n, \mathscr{L}(\R^n, \R^m))} \).\\
Sei \(f\ k\)-mal differenzierbar. 
\[ D^k f = D(D^{k-1} f) : U \rightarrow \mathscr{L}(\R^n, \mathscr{L}(\R^n), \ldots) \]
Was ist das? (Handout)
Das Schöne ist, man kann zeigen, dass 
isometrisch isomorph zum Raum aller stetigen multilinearen 
Abbildungen \( \underbrace{\R^n \times \ldots \times \R^n}_{k\text{-mal}} 
\rightarrow \R^m \) ist.

\begin{defi*}
    Eine Funktion \(\abb{f}{\R^n\times \ldots \times \R^n}{\R}\) heißt \(k\)-linear,
    wenn für jedes \(l \in \set{1,\ldots,k}\) und jede Wahl 
    \(x_1, \ldots, x_{l-1}, x_l, x_{l+1}, \ldots, x_k \in \R^n\)
    die Abbildung
    \[ \abb{f(x_{l-1}, x_l, x_{l+1}, \ldots, x_k)}{\R^n}{\R} \]
    linear ist.    
\end{defi*}
\begin{satz*}
    Sei \( \abb{f}{\R^n \times \ldots \times \R^n}{\R}\ k \)-linear, 
    dann ist \(f\) stetig auf \( \R^n \times \ldots \times \R^n \) 
    und es gilt 
    \[ \norm{ f(x_1, \ldots, x_k) } 
    \leq \norm{f} \prod_{j=1}^k \norm{x_j}. \]
\end{satz*}
\begin{bsp}
    2-lineare Funktion \(f\),
    \begin{align*}
        f(x_1+h_1, x_2+h_2) &= f(x_1, x_2+h_2) + f(h_1, x_1 +h_2)\\
        &= f(x_1, x_2) + f(h_1, x_2) + f(x_1, h_2) + f(h_1, h_2)
    \end{align*}
\end{bsp}
\end{document}