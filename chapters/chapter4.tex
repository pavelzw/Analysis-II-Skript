%26.04.2019
\documentclass[../ana2.tex]{subfiles}
\begin{document}

\setcounter{section}{3}
\section{Integrieren}
Bild:
%Bild
\subsection{Treppenfunktionen und ihr Integral}
\begin{defi}
    Eine Funktion \( \abb{\varphi}{[a,b]}{\C} \) 
    (oder \(\R, \R^d\))
    heißt Treppenfunktionen, falls es Punkte 
    \( a=x_0 < x_1 < \cdots < x_n = b \) gibt, sodass 
    \( \varphi \) auf jedem Intervall \( (x_{k-1}, k_k) \)
    konstant ist.    
\end{defi}
\begin{bem}
    Egal, was in \(x_k\) passiert.
    %Bild
\end{bem}
\begin{bem}
    Eine Menge \( Z = \set{x_0, \ldots, x_n} \) nennen 
    wir Zerlegung von \( [a,b] \), falls 
    \[ a = x_0 < x_1 < \cdots < x_k < \cdots < x_n = b \]
    ist.
\end{bem}
Die Menge aller Treppenfunktionen \( T[a,b] \) bildet 
einen Vektorraum (siehe Lemma 4).
\begin{defi}[Integral einer Treppenfunktionen]
    Ist \( \abb{\varphi}{[a,b]}{\C} \) und 
    \( Z = \set{x_0, \ldots, x_n} \) eine Zerlegung 
    \gqq{passend} zu \( \varphi \), d.\ h.\  \( \varphi \) 
    ist auf jedem offenen Teilintervall \( (x_{k-1}, x_k) \) 
    kostant, so definieren wir 
    \[ \integral{\varphi(x)}{a}{b} = \integral{\varphi(x)}{a}{b} 
    := \sum_{k=1}^n c_k (\underbrace{x_k - x_{k-1}}_{= \Delta x_k})
    = \sum_{k=1}^n c_k \Delta x_k \]
\end{defi}
Frage: Ist der Wert der obigen Summe unabhängig von der Zerlegung?
\begin{lem}
    Die obige Definition ist unabhängig von der Zerlegung.
\end{lem}
\begin{bew}
    Sei \( \varphi \) feste Treppenfunktion, \(Z\) eine passende 
    Zerlegung, \( Z=\set{x_0, x_1,\dots, x_n}, x_0=a<x_1<\dots<x_n=b \) \\
    \( \varphi \) ist konstant.
    
    \[ I(Z) := \sum_{k=1}^n c_k \Delta x_k \]
    Sei \(\tilde{Z}\) andere passende Zerlegung: 
    z.z.: \(I(Z) = I(\tilde{Z})\)
    Erste Beobachtung: Nennen \( Z \) Verfeinerung von \( Z \), 
    falls \( Z \subset Z' \). Also \( Z' \) enthält alle Punkte 
    von \(Z\) und vielleicht mehr.
    \begin{beh}
        \[ I(z) = I(Z') \; \forall \text{Verfeinerungen } Z' \text{ von } Z. \]
    \end{beh}
    \begin{bew}
        Angenommen \(Z'\) enthält genau einen Punkt \(y\) mehr 
        als \(Z\).
        \[ \Rightarrow \exists k : x_{k-1} < y < x_k. \]
        \[ Z = \set{x_0,\ldots,x_n} \]
        \[ Z' = \set{x_0,\ldots,x_{k-1}, x_k,\ldots,x_n} \]
        \(\Rightarrow \) in \(I(Z)\) ist der Summand 
        \( c_k(x_k - x_{k-1}) \) zu ersetzen mit 
        \[ c_k(y - x_{k-1}) + c_k(x_k - y) 
        = c_k(x_k - x_{k-1}). \]
        alle anderen Summanden sind gleich.
        \[ \Rightarrow I(Z) = I(Z'). \]
    \end{bew}
    Allgemeiner Fall: Ist \(Z'\) eine Verfeinerung von \(Z\)
    füge eine schrittweise neue Punkte zu \(Z\).
    \[ \Rightarrow \text{Verfeinerungen } Z = Z_0' 
    \subset Z_1' \subset Z_2' \subset \cdots \subset Z_k' = Z'. \]
    \( Z_k' \setminus Z_{k-1}' \) enthält genau einen Punkt.
    \[ \Rightarrow I(Z) = I(Z'_0) = I(Z'_1) 
    = \cdots = I(Z'_k) = I(Z'). \]
    2. Beobachtung: Gegeben zwei Zerlegungen \(Z_1, Z_2\), die zu 
    \( \varphi \) passen.\\
    \( \Rightarrow \tilde{Z} := Z_1 \cup Z_2 \).\\
    \( \Rightarrow \tilde{Z} \) Verfeinerung von \( Z_1, Z_2 \).
    \[ \overundersett{Erste}{Beobachtung}{\Rightarrow} 
    I(Z_1) = I(\tilde{Z}) = I(Z_2). \]
\end{bew}
\begin{lem}
    Seien \( \varphi, \psi \) Treppenfunktionen und 
    \( \alpha, \beta \ni \C \).
    \begin{enumerate}
        \item \( \alpha \varphi + \beta \psi \) ist Treppenfunktion.
        \item Linearität: \( \integral{(\alpha \varphi + \beta \psi)}{a}{b} 
        = \alpha \integral{\varphi}{a}{b} 
        + \beta \integral{\psi}{a}{b} \).
        \item Beschränktheit: \( \abs{\integral{\varphi}{0}{b}}
        \leq \integral{\abs{\varphi}}{0}{b} \leq (b-a||\psi||_\infty) \). \\
        \( ||\varphi||_\infty 
        := \underset{x \in [a,b]}{\sup} \abs{\varphi(x)} \).
        \item Monotonie: Sind \( \varphi, \psi \) reell und 
        \( \varphi \leq \psi \).
        \[ \Rightarrow \integral{\varphi}{a}{b} 
        \leq \integral{\psi}{a}{b}. \]
    \end{enumerate}    
\end{lem}
\begin{bew}
    Seien \( Z_1, Z_2 \) Zerlegungen passend zu \( \varphi \) bzw. 
    \( \psi \).\\
    \( \tilde{Z} := Z_1 \cup Z_2 \) passt auch zu \( \varphi \) und 
    \( \psi \).\\
    \( \Rightarrow \) Linearkombination \( \alpha \varphi + \beta \psi \) 
    ist Treppenfunktion mit passender Zahl \( \tilde{Z} \).\\
    Nehme \( \tilde{Z} = \set{x_0,\ldots, x_m} \).
    Z. B. \( \varphi = c_k \) auf \( (x_{k-1}, x_k) \).
    %align
    \[ \Rightarrow \abs{ \integral{\varphi}{a}{b} } 
    = \abs{\sum_{k=1}^m c_k (x_k - x_{k-1}) } \]
    \[ \leq \sum_{k=1}^m \abs{c_k} (x_k - x_{k-1}) \]
    \[ = \integral{\abs{\varphi}}{a}{b}. \]

    \[ \leq \underbrace{\underset{k=1,\dots,m}{\max}}
    _{\underset{x \in [a,b]}{\abs{\varphi(x)}} \abs{x_k}}
    \underbrace{\sum_{k=1}^m (x_k-x_{k-1})}_{=b-a} \]
    \[ = ||\varphi||_\infty (b-a) \Rightarrow \text{3)} \]
    2) und 4) analog.
\end{bew}
\begin{bem}
    \[ \mathds{1}_A (x) = \begin{cases}
        1, x \in A \\
        0, x \notin A
    \end{cases}, A\subset \R. \]
    charakteristische Funktion von \(A\).\\
    \( \Rightarrow \varphi \) Treppenfunktion
    \[ \varphi = \sum_{k=1}^m c_k \mathds{1}_{A_k}. \]
    \( A_k \) offenes Intervall oder Punkt.
\end{bem}
\subsection{Regelfunktion und ihr Integral}
Wir haben 
\[ \abb{\int_a^b}{T[a,b]}{\C} \text{ (oder } \R \text{)} \]
ist eine lineare Abbildung.\\
Treppenfunktion \( \varphi_n \rightarrow f \) gleichmäßig.
\[ \abs{ \underbrace{\integral{\varphi_n} {a}{b}}_{I(\varphi_n)} 
- \underbrace{\integral{\varphi_m} {a}{b}}_{I(\varphi_m)} } 
= \abs{ \integral{(\varphi_n - \varphi_m)}{a}{b} }. \]
\[ \oversett{(3)}{=} (b-a) ||\varphi_n-\varphi_m||_\infty 
\rightarrow 0, n,m \rightarrow \infty \]
\end{document}