%26.04.2019
\documentclass[../ana2.tex]{subfiles}
\begin{document}

\setcounter{section}{3}
\section{Integrieren}
Bild:
%Bild
\subsection{Treppenfunktionen und ihr Integral}
\begin{defi}
    Eine Funktion \( \abb{\varphi}{[a,b]}{\C} \) 
    (oder \(\R, \R^d\))
    heißt Treppenfunktion, falls es Punkte 
    \( a=x_0 < x_1 < \cdots < x_n = b \) gibt, sodass 
    \( \varphi \) auf jedem Intervall \( (x_{k-1}, x_k) \)
    konstant ist.    
\end{defi}
\begin{bem}
    Egal, was in \(x_k\) passiert.
    %Bild
\end{bem}
\begin{bem}
    Eine Menge \( Z = \set{x_0, \ldots, x_n} \) nennen 
    wir Zerlegung von \( [a,b] \), falls 
    \[ a = x_0 < x_1 < \cdots < x_k < \cdots < x_n = b \]
    ist.
\end{bem}
Die Menge aller Treppenfunktionen \( T[a,b] \) bildet 
einen Vektorraum (siehe Lemma 4).
\begin{defi}[Integral einer Treppenfunktionen]
    Ist \( \abb{\varphi}{[a,b]}{\C} \) und 
    \( Z = \set{x_0, \ldots, x_n} \) eine Zerlegung 
    \gqq{passend} zu \( \varphi \), d.\ h.\  \( \varphi \) 
    ist auf jedem offenen Teilintervall \( (x_{k-1}, x_k) \) 
    konstant, so definieren wir 
    \[ \integral{\varphi(x)}{a}{b} 
    := \sum_{k=1}^n c_k (\underbrace{x_k - x_{k-1}}_{= \Delta x_k})
    = \sum_{k=1}^n c_k \Delta x_k \]
\end{defi}
\begin{lem}
    Die obige Definition ist unabhängig von der Zerlegung.
\end{lem}
\begin{bew}
    Sei \( \varphi \) feste Treppenfunktion, \(Z\) eine passende 
    Zerlegung, \( Z=\set{x_0, x_1,\dots, x_n}.\\
    a=x_0<x_1<\cdots<x_n=b \) \\
    \( \varphi \) ist konstant.
    
    \[ I(Z) := \sum_{k=1}^n c_k \Delta x_k \]
    Sei \(\tilde{Z}\) andere passende Zerlegung: 
    z.z.: \(I(Z) = I(\tilde{Z})\)
    Erste Beobachtung: Nennen \( Z \) Verfeinerung von \( Z \), 
    falls \( Z \subset Z' \). Also \( Z' \) enthält alle Punkte 
    von \(Z\) und vielleicht mehr.
    \begin{beh}
        \[ I(z) = I(Z') \; \forall \text{Verfeinerungen } Z' \text{ von } Z. \]
    \end{beh}
    \begin{bew}
        Angenommen \(Z'\) enthält genau einen Punkt \(y\) mehr 
        als \(Z\).
        \[ \Rightarrow \exists k : x_{k-1} < y < x_k. \]
        \[ Z = \set{x_0,\ldots,x_n} \]
        \[ Z' = \set{x_0,\ldots,x_{k-1}, x_k,\ldots,x_n} \]
        \(\Rightarrow \) in \(I(Z)\) ist der Summand 
        \( c_k(x_k - x_{k-1}) \) zu ersetzen mit 
        \[ c_k(y - x_{k-1}) + c_k(x_k - y) 
        = c_k(x_k - x_{k-1}). \]
        alle anderen Summanden sind gleich.
        \[ \Rightarrow I(Z) = I(Z'). \]
    \end{bew}
    Allgemeiner Fall: Ist \(Z'\) eine Verfeinerung von \(Z\)
    füge eine schrittweise neue Punkte zu \(Z\).
    \[ \Rightarrow \text{Verfeinerungen } Z = Z_0' 
    \subset Z_1' \subset Z_2' \subset \cdots \subset Z_k' = Z'. \]
    \( Z_k' \setminus Z_{k-1}' \) enthält genau einen Punkt.
    \[ \Rightarrow I(Z) = I(Z'_0) = I(Z'_1) 
    = \cdots = I(Z'_k) = I(Z'). \]
    2. Beobachtung: Gegeben zwei Zerlegungen \(Z_1, Z_2\), die zu 
    \( \varphi \) passen.\\
    \( \Rightarrow \tilde{Z} := Z_1 \cup Z_2 \).\\
    \( \Rightarrow \tilde{Z} \) Verfeinerung von \( Z_1, Z_2 \).
    \[ \overundersett{Erste}{Beobachtung}{\Rightarrow} 
    I(Z_1) = I(\tilde{Z}) = I(Z_2). \]
\end{bew}
\begin{lem}
    Seien \( \varphi, \psi \) Treppenfunktionen und 
    \( \alpha, \beta \in \C \).
    \begin{enumerate}
        \item \( \alpha \varphi + \beta \psi \) ist Treppenfunktion.
        \item Linearität: \( \integral{(\alpha \varphi + \beta \psi)}{a}{b} 
        = \alpha \integral{\varphi}{a}{b} 
        + \beta \integral{\psi}{a}{b} \).
        \item Beschränktheit: \( \abs{\integral{\varphi}{0}{b}}
        \leq \integral{\abs{\varphi}}{0}{b} \leq (b-a) \cdot ||\psi||_\infty \). \\
        \( ||\varphi||_\infty 
        := \underset{x \in [a,b]}{\sup} \abs{\varphi(x)} \).
        \item Monotonie: Sind \( \varphi, \psi \) reell und 
        \( \varphi \leq \psi \).
        \[ \Rightarrow \integral{\varphi}{a}{b} 
        \leq \integral{\psi}{a}{b}. \]
    \end{enumerate}    
\end{lem}
\begin{bew}
    Seien \( Z_1, Z_2 \) Zerlegungen passend zu \( \varphi \) bzw. 
    \( \psi \).\\
    \( \tilde{Z} := Z_1 \cup Z_2 \) passt auch zu \( \varphi \) und 
    \( \psi \).\\
    \( \Rightarrow \) Linearkombination \( \alpha \varphi + \beta \psi \) 
    ist Treppenfunktion mit passender Zahl \( \tilde{Z} \).\\
    Nehme \( \tilde{Z} = \set{x_0,\ldots, x_m} \).
    Z. B. \( \varphi = c_k \) auf \( (x_{k-1}, x_k) \).
    %align
    \[ \Rightarrow \abs{ \integral{\varphi}{a}{b} } 
    = \abs{\sum_{k=1}^m c_k (x_k - x_{k-1}) } \]
    \[ \leq \sum_{k=1}^m \abs{c_k} (x_k - x_{k-1}) \]
    \[ = \integral{\abs{\varphi}}{a}{b}. \]

    \[ \leq \underbrace{\underset{k=1,\dots,m}{\max}}
    _{\underset{x \in [a,b]}{\abs{\varphi(x)}} \abs{x_k}}
    \underbrace{\sum_{k=1}^m (x_k-x_{k-1})}_{=b-a} \]
    \[ = ||\varphi||_\infty (b-a) \Rightarrow \text{3)} \]
    2) und 4) analog.
\end{bew}
\begin{bem}
    \[ \mathds{1}_A (x) = \begin{cases}
        1, x \in A \\
        0, x \notin A
    \end{cases}, A\subset \R. \]
    charakteristische Funktion von \(A\).\\
    \( \Rightarrow \varphi \) Treppenfunktion
    \[ \varphi = \sum_{k=1}^m c_k \mathds{1}_{A_k}. \]
    \( A_k \) offenes Intervall oder Punkt.
\end{bem}
\subsection{Regelfunktionen und ihre Integrale}
Wir haben 
\[ \abb{\int_a^b}{T[a,b]}{\C} \text{ (oder } \R \text{)} \]
ist eine lineare Abbildung.\\
Treppenfunktion \( \varphi_n \rightarrow f \) gleichmäßig.
\[ \abs{ \underbrace{\integral{\varphi_n} {a}{b}}_{I(\varphi_n)} 
- \underbrace{\integral{\varphi_m} {a}{b}}_{I(\varphi_m)} } 
= \abs{ \integral{(\varphi_n - \varphi_m)}{a}{b} }. \]
\[ \oversett{(3)}{=} (b-a) ||\varphi_n-\varphi_m||_\infty 
\rightarrow 0, n,m \rightarrow \infty \]
\begin{defi}[Regelfunktion]
    Sei \( I \) ein Intervall mit Anfangspunkt \(a\) 
    und Endpunkt \(b\). Eine Funktion 
    \[ \abb{f}{[a,b]}{\C} \] 
    heißt Regelfunktion, falls sie 
    \begin{enumerate}
        \item in jedem Punkt \(x \in (a,b)\) 
        einen linksseitigen und rechtsseiten 
        Grenzwert hat.
        
        \item im Fall \( a \in I \) in \(a\) einen 
        rechtsseiten Grenzwert hat und falls 
        \( b \in I \) in \(b\) 
        einen linksseitigen Grenzwert hat.
    \end{enumerate}
    Wir bezeichnen den \( \C \)-Vektorraum der 
    Regelfunktionen mit \( R(I) \).
    \begin{bspe}
        \begin{enumerate}
            \item Die stetigen Funktionen auf \( I \rightarrow \C \).
            \item Die monotonen Funktionen \( I \rightarrow \R \).
        \end{enumerate}      
    \end{bspe}
    Sind \( f,g \) Regelfunktionen, so sind auch 
    \( \abs{f}, g\cdot f \) Regelfunktionen, sowie im 
    reellen Fall auch \( \min\set{f,g}, \max\set{f,g} \).
\end{defi}
\begin{satz}[Approximationssatz]
    Eine Funktion \( \abb{f}{[a,b]}{\C} \) ist 
    eine Regelfunktion genau dann, wenn 
    \[ \forall \varepsilon > 0 \; \exists 
    \text{Treppenfunktion } \varphi \in T[a,b] \]
    mit 
    \[ ||f - \varphi||_\infty 
    = \underset{x \in [a,b]}{\sup} |f(x) - \varphi(x)| 
    \leq \varepsilon \]
    (dieses \(\varphi\) nennt man \( \varepsilon \)-Approximationsfunktion
    für \(f\))
\end{satz}
\begin{bew}
    Siehe Skript. (handschriftlich)
\end{bew}
\begin{kor}
    \( \abb{f}{[a,b]}{\C} \) ist eine Regelfunktion 
    \[ \Leftrightarrow \exists \text{Folge } 
    (\varphi_n)_n \text{ von Treppenfunktionen} \]
    \( \varphi_n \in T[a,b] \) mit \( ||f-\varphi_n||_\infty 
    \rightarrow 0 (n\rightarrow \infty) \).
\end{kor}
\begin{bew}
    \gqq{\( \Leftarrow \)}: Klar. Sei \( \varepsilon > 0\), 
    da \( ||f-\varphi_n||_\infty \rightarrow 0 \) folgt 
    \( ||f-\varphi_n|| < \varepsilon \) für fast alle \(n\)
    
    \gqq{\( \Rightarrow \)}: Satz \(6\) 
    mit \( \varepsilon = \frac{1}{n} \).
\end{bew}
\begin{kor}
    \( f \) ist eine Regelfunktion
    \[ \exists (\varphi_k)_k, \psi_k \in T[a,b] 
    \text{ mit } f = \sum_{k=1}^\infty \psi_k. \]
\end{kor}
\begin{bew}
    \gqq{\( \Rightarrow \)}: Wähle \( \varphi_k \in T[a,b] \)
    mit \( ||f-\varphi_k||_\infty \leq 2^{-k} \)
    \[ \psi_1 := \varphi_1, \psi_2 := \varphi_2 - \varphi_1, 
    \ldots, \psi_n := \varphi_n - \varphi_{n-1} \]
    \[ \Rightarrow |f(x) - \sum_{k=1}^n \psi_k(x)| 
    = |f(x)-\varphi_n(x)| \leq ||f-\varphi_n||_\infty \leq 2^{-k} \]
    Beachte: \( ||\psi_k||_\infty = ||\varphi_k - \varphi_{k-1}||_\infty
    \leq ||\varphi_k-f||\infty + ||f-\varphi_{k-1}||_\infty
    \leq 2^{-k} +  2^{-(k-1)} = \frac{3}{2^k} \).

    Somit 
    \[ \sum_{k=1}^\infty ||\psi_k||_\infty < \infty. \]
    Reihe konvergiert absolut in \( ||\cdot||_\infty \).

    \gqq{\( \Leftarrow \)}: Gegeben \( (\psi_k)_k \).
    \[ \phi_n := \sum_{k=1}^n \psi_k \in T[a,b]. \]
    \( \Rightarrow ||f - \varphi_n||_\infty \rightarrow 0 \).
\end{bew}
\begin{kor}
    Sei \(I\) ein Intervall. Jede Regelfunktion 
    \[ \abb{f}{I}{\C} \]
    ist bis auf abzählbar viele Stellen stetig.
\end{kor}
\begin{bew}
    Sei \( a,b \in I, J := [a,b] \)
    \[ \oversett{Kor. 8}{\Rightarrow} f 
    = \sum_{k=1}^\infty \psi_k
    , \quad \psi_k \in T[a,b]. \]
    \( \psi_k \) hat nur endlich viele 
    Unstetigkeitsstellen.\\
    \( \sum_{k=1}^\infty \psi_k \) hat höchstens abzählbar viele 
    Unstetigkeitsstellen in \( [a,b] \).
    
    Ist \( I = (c,d) \Rightarrow I 
    = \bigcup_{n\in\N} 
    [c + \frac{1}{n}, d - \frac{1}{n}] \) \\
    \( \Rightarrow f\) hat auch \( (c,d) \) nur abzählbar 
    abzählbar (= abzählbar) viele Unstetigkeitsstellen.
\end{bew}
\begin{kor}
    Jede Regelfunktion \( \abb{f}{[a,b]}{\C} \) 
    ist beschränkt. 
\end{kor}
\begin{bew}
    Sei \( \varphi \in T[a,b] \) eine \(1\)-Approximation 
    von \( f \). \( ||f - \varphi||_\infty \leq 1 \).
    \[ \Rightarrow ||f||_\infty 
    = || f - \varphi + \varphi ||_\infty 
    \leq ||f - \varphi||_\infty + ||\varphi||_\infty 
    \leq 1 + ||\varphi||_\infty. \]
\end{bew}
\subsection{Integration von Regelfunktionen 
über kompakten Intervallen}
\begin{satzdefi}
    Sei \( \abb{f}{[a,b]}{\C} \) eine Regelfunktion
    \( \forall \) Folgen \( (\varphi_n)_n, \varphi_n \in T[a,b] \)
    mit \( ||f - \varphi_n||_\infty \rightarrow 0 \) existiert.
    \[ \integral{f(x)}{a}{b} 
    := \limes{n} \integral{\varphi_n(x)}{a}{b}. \]
    Der Grenzwert hängt nicht von der Wahl der Folge
    \( (\varphi_n)_n \) ab.\\
    Wir nennen den Grenzwert das Integral von 
    \( f \) über \( [a,b] \).
\end{satzdefi}
\begin{bew}
    1. Schritt: \( I_n := \integral{\varphi_n(x)}{a}{b} \).
    \begin{beh}
        \( I_n \) ist eine Cauchy-Folge.
    \end{beh}
    \begin{bew}
        \( I_n - I_m \oversett{Lem. 4}{=} 
        \integral{(\varphi_n - \varphi_m)}{a}{b} \).

        \begin{align*}
            &\overundersett{Lemma}{4}{\Rightarrow} \abs{I_n - I_m} \\
            &\leq \integral{\abs{\varphi_n - \varphi_m}}{a}{b} \\
            &\leq (b-a) ||\varphi_n - \varphi_m||_\infty \\
            &\leq (b-a) \left( ||\varphi_n - f||_\infty 
            + ||f - \varphi_m \right) \rightarrow 0.
        \end{align*}
    \end{bew}
    2. Schritt: Unabhängigkeit von der Folge
    \begin{bew}
        Seien \( (\varphi_n)n, (\psi_n)_n \) zwei Folgen 
        von Treppenfunktionen mit 
        \[ ||f-\varphi_n||_\infty \overset{n\rightarrow \infty}{\rightarrow} 0, 
        ||f-\psi_n||_\infty \overset{n\rightarrow \infty}{\longrightarrow} 0 \]
        Konstruktion neuer Folge \(\chi_n \) durch Reisverschlussverfahren:
        \( \varphi_1, \psi_1, \varphi_2, \psi_2, \ldots \)
        Das heißt
        \[ \chi_{2n} = \varphi_n, \chi_{2n-1} = \psi_n. \]
        Haben immer noch
        \[ ||f - \chi_n||_\infty \overset{n \rightarrow \infty}{\rightarrow} 0 \]
        \( \overundersett{Schritt}{1}{\rightarrow} \limes{n} \integral{\chi_n}{a}{b} \)
        existiert.
        Teilfolgensatz:     
        \[ \limes{n} \integral{\chi_{2n}}{a}{b} = \limes{n} 
        \integral{\chi_{2n+1}}{a}{b} \]
        \[ = \limes{n} \integral{\varphi_n}{a}{b} 
        = \limes{n} \integral{\psi_n}{a}{b} \]
    \end{bew}
\end{bew}
\begin{kor}
    Das Integral ist für stetige und monotone 
    Funktionen definiert.
\end{kor}
\begin{bew}
    Klar.
\end{bew}
\begin{satz}
    Seien \( f,g \in R([a,b]),\; \alpha, \beta \in \C \).
    \begin{enumerate}[label=(\alph*)]
        
        \item Linearität: \[ \integral{(\alpha f + \beta g)}{a}{b}
        = \alpha \integral{f}{a}{b} + \beta \integral{g}{a}{b} \]
        \item Beschränktheit: \[ \abs{\integral{f}{a}{b}} \leq 
        \integral{\abs{f}}{a}{b} 
        \leq (b-a)||f||_\infty. \]
        \item Monotonie: \[ \integral{f}{a}{b} 
        \leq \integral{g}{a}{b}, \text{ falls } f \leq g \]
    \end{enumerate}
\end{satz}
\begin{bew}
    Sei \((\varphi_n)_n, (\psi_n)_n \subset T[a,b] \) mit 
    \( ||f-\varphi_n||_\infty \rightarrow 0, 
    ||g-\psi_n||_\infty \rightarrow 0, n \rightarrow \infty \)
    
    (a): Da \( ||(\alpha f+\beta g) - (\alpha \varphi_n + \beta \psi_n) ||_\infty \)
    \[ \leq \abs{\alpha} ||f - \varphi_n||_\infty 
    + \abs{\beta} ||g - \psi_n||_\infty \rightarrow 0 \]
    
    folgt 
    \[ \integral{(\alpha f + \beta g)}{a}{b} \oversett{Def 11}{=}
    \limes{n} \integral{(\alpha f + \beta g)}{a}{b}
    = \oversett{Lemma 4}{=} \alpha \limes{n} \integral{\varphi_n}{a}{b}
    + \beta \limes{n} \integral{\psi_n}{a}{b} \]
    \[ \oversett{Satz 11}{=} \alpha \integral{f}{a}{b} 
    + \beta \integral{g}{a}{b}. \]

    (b): \( || \abs{f} - \abs{\varphi_n}||_\infty \leq ||f-\varphi_n||_\infty
    \rightarrow 0 (n \rightarrow \infty) \)
    \[ \Rightarrow \integral{\abs{f}}{a}{b}
    = \limes{n} \integral{\abs{\varphi_n}}{a}{b} \]

    \[ \Rightarrow \abs{\integral{f}{a}{b}} = \limes{n} \abs{\integral{\varphi_n}{a}{b}}
    \leq \limes{n} 
    \integral{\underbrace{\abs{\varphi_n}}_{\leq (b-a)||\varphi_n||_\infty}}{a}{b} 
    = \integral{\abs{f}}{a}{b} \]
    
    \[ \leq \limes{n} ||\varphi_n||_\infty (b-a)  
    = ||f||_\infty (b-a) \]
    
    (c): Seien \( f \leq g \) und \( \varphi_n, \psi_n \) reellwertig.
    Problem: \( \varphi_n \leq \psi_n \) ist im Allgemeinen falsch.\\
    Setze \[ \varphi_n^- 
    := \varphi_n - ||f - \varphi_n||_\infty 
    \leq f. \]
    \[ \psi_n^+ := \psi_n + ||g - \psi_n||_\infty \geq g. \]
    \[ \Rightarrow \varphi_n^- \leq f \leq g \leq \psi_n^+ \]
    und \( \varphi_n^-, \psi_n^+ \in T[a,b] \)
    und \( ||f - \varphi_n^-||_\infty \rightarrow 0, 
    ||g - \psi_n||_\infty \rightarrow 0, n\rightarrow \infty \).
    \[ \Rightarrow \integral{f}{a}{b}
    \oversett{Def. 11}{=} \limes{n} 
    \integral{\varphi_n^-}{a}{b} \]
    \[ \leq \limes{n} \integral{\psi_n^+}{a}{b} 
    = \integral{g}{a}{b}. \]
\end{bew}
\begin{satz}
    Seien \( a < b < c \), \(f\) eine Regelfunktion auf 
    \( [a,c] \). Dann gilt 
    \[ \integral{f}{a}{c} 
    = \integral{f}{a}{b} + \integral{f}{b}{c} \tag{\(*\)}. \]
\end{satz}
\begin{bew}
    Ist \( f \in T[a,c] \), so ist \( (*) \) offensichtlich wahr.
    Durch Approximieren erweitern wir dies auf 
    \( f \in R([a,c]) \).\\
    Man beachte: ist \( \varphi_1 \in T[a,b]: \abs{f(x) - \varphi_1(x)} 
    \leq \varepsilon \forall x\in[a,b] \)
    und \( \varphi_2 \in T[b, c]: |f(x)-\varphi_2| \leq \varepsilon \forall x \in [b,c]\)    
    \( \Rightarrow \varphi(x) := \begin{cases}
        \varphi_1(x), x\in [a,b]\\
        \varphi_2(x), x \in (b,c]
    \end{cases} \in T[a,c] \) und erfüllt \( \abs{f(x) - \varphi(x)} \leq \varepsilon 
    \; \forall x \in [a,c] \).
\end{bew}
\begin{notation}
    Wir definieren
    \[ \integral{f}{a}{a} = 0 \;\forall a\in\R. \]
    \[ \integral{f}{a}{b} := - \integral{f}{b}{a}, \text{ falls } b < a \]
\end{notation}
\begin{bem}
    Ist \( \abb{f}{[a,b]}{\R} \) eine Regelfunktion,
    so folgt aus der Monotonie:
    
    \[ \inf \set{f(x): x\in[a,b]} \cdot (b-a) \leq \integral{f}{a}{b} 
    \leq \sup \set{f(x): x \in [a,b]} \cdot (b-a) \]
    
    Ist \( \abb{f}{[a,b]}{\R} \) stetig, so existiert 
    \( \rho \in (a,b) \) mit
    \[ \integral{f}{a}{b} = (b-af(\rho)) \]
\end{bem}
\begin{satz}[Allgemeiner Mittelwertsatz für Integrale]
    Es sei \( \abb{f}{[a,b]}{\R} \) stetig, 
    \( p \in R[a,b] \) mit 
    \[ p \geq 0 \text{ (Gewichtsfunktion)}. \]
    Dann gilt: \( \exists \rho \in (a,b) \) mit 
    \[ \integral{f(x) p(x)}{a}{b} 
    = f(\rho) \integral{p(x)}{a}{b}. \]    
\end{satz}
\begin{bew}
    Sei \( m = \min f \) auf \([a,b]\), \(M=\max f\) auf \([a,b]\)
    \[ \Rightarrow m p \leq f p \leq M p \]
    \[ \oversett{Monotonie}{\Rightarrow} m \integral{p}{a}{b}
    \leq \integral{fp}{a}{b} \leq M \integral{p}{a}{b}  \]

    \[ \Rightarrow \exists \mu \in [m,M]: 
    \integral{fp}{a}{b} = \mu \integral{p}{a}{b} \]
    Weiter gilt, da \(f\) stetig ist, dass 
    \(\rho \in [a,b]\) mit \( f(\xi) = \mu \).
\end{bew}


\end{document}