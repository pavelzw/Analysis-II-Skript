\documentclass[../ana2.tex]{subfiles}

\begin{document}
\setcounter{section}{16}
\section{Satz über implizite Funktionen (und inverse Funktionen)}
Situation: Sei \(\R^n, \R^m\) und \(\abb{F}{\R^n \times \R^m}{\R^m}\) gegeben. 
Angenommen, es existiert \(x_0 \in \R^n\), \(y_0 \in \R^m\):
\[ F(x_0, y_0) = 0. \]
Frage: Können wir die \( m \) Gleichungen lokal 
(d. h. in der Nähe von \( x_0 \)) nach \(y\) 
auflösen, d. h. gibt es eine Funktion 
\( \abb{g}{\R^n}{\R^m} \) mit 
\[ F(x, g(x)) = 0 \; \forall x \text{ nahe bei } x_0 \]
\subsection*{Linearisierung} 
\[ F(x,y) = \underbrace{F(x_0, y_0)}_{=0} + \underbrace{DF(x_0, y_0) \begin{pmatrix}
    x - x_0 \\ y - y_0
\end{pmatrix} }_{
    = D_1 F(x_0, y_0)[x - x_0] + D_2 F(x_0, y_0)[y - y_0]
}+ \text{ höhere Ordnungsterme}\]
mit \\
\( D_1 F = D_x F =  \) (verallgemeinerte) 
partielle Ableitung nach \( x_1,\ldots, x_n \)\\
\( D_2F = D_yF = \) (verallgemeinerte) 
partielle Ableitung nach \(y_1, \ldots , y_n \)
\[ A = \abb{D_1 F(x_0, y_0)}{\R^n}{\R^n} \text{ linear} \]
\[ B = \abb{D_2 F(x_0, y_0)}{\R^m}{\R^m} \text{ linear} \]
\( \Rightarrow \) linearer Fall: 
\(0 = F(x, y) = A[x-x_0] + B[y-y_0]\)
\[ \Rightarrow y = y_0 - B^{-1}A[x - x_0] \]
ist die einzige Lösung, sofern \( B \) invertierbar ist.
\begin{satz}[Satz über implizite Funktionen]
    Gegeben \(\R^n, \R^m, U \subset \R^n \times \R^m\) offen, 
    \((x_0, y_0) \in U\) und \(\abb{F}{U}{\R^m}\) stetig differenzierbar
    in \(U\). Ist
    \[ F(x_0, y_0) = c \in \R^m \]
    und sind die verallgemeinerten partiellen Ableitungen von \(F\)
    nach \(y\), also
    \[ \abb{D_yF(x_0,y_0) = D_2F(x_0, y_0)}{\R^m}{\R^m} \]
    invertierbar, dann existieren offene Umgebungen \(\R^n \supset U_1\)
    von \(x_0\) und \(\R^m \supset U_2\) von \(y_0, U_1 \times U_2 \subset U\)
    und eine differenzierbare Funktion \(\abb{g}{U_1}{U_2}\) mit 
    \[F(x, g(x)) = c \; \forall x \in U_1\]
    Ferner ist 
    \[ D g(x) = -(D_2 F(x, g(x)))^{-1} \ D_1 F(x, g(x)) \]
    für alle \( x\in U_1 \). Außerdem ist für \( x \in U_1, y \in U_2 \)
    \[ F(x,y) = c \Leftrightarrow y = g(x). \]
    D. h. \(g(x)\) ist die einzige Lösung der Gleichung
    \[F(x, y) = c\]
\end{satz}
Für den Beweis brauchen wir eine leichte Verschärfung von
Banach.
\begin{lem}
    Sei \( M \) ein vollständiger metrischer Raum und \( P \) 
    (Parameterraum) ein weiterer metrischer Raum und für \( x\in P \)     
    sei \(\abb{T_x}{M}{M}\) eine \(\alpha-\) Kontraktion mit \(\alpha < 1\)
    wobei \(\alpha\) unabhängig von \(x\) ist. Dann exisitert für jedes \(x \in P\)
    ein Fixpunkt \(y_x \in M\) von \(T_x\), d. h. 
    \[ y_x = T_x(y_x) \]
    Ferner: Ist für alle \(y \in M\) die Abbildung 
    \(P \ni x \mapsto T_x(y)\) stetig, so ist
    \( x \mapsto y_x \) stetig.
\end{lem}
Motivation:
\( y = f(x) \). Inverse \( f^{-1}(x) \) ist implizit definit 
\( x = f(y) \). \( F(x,y) = x - f^{-1}(y) = 0 \).\\
Funktionssphäre in \( \R^3 \) ist Lösung von 
\( F(x,y,z) = 0 \).
\[ F(x,y,z) = x^2 + y^2 + z^2 - 1 \]
\( F(x,y,z) = x^{27}y^5 - z^{16}x^9 + x^4 y^4 z^4 - 8xyz - 1 \).
\begin{bew}
    Aus Satz 16.2 folgt \( \forall x \in P \) 
    existiert \( y_x = T_x(y,x) \), also ein Fixpunkt.\\
    Stetigkeit des Fixpunktes in dem Parameter \( x \): 
    Seien \( x_1, x_2 \in P, y_1 = y_{x_1}, y_2 = y_{x_2}, 
    T_1 = T_{x_1}, T_2 = T_{x_2} \).
    \begin{align*}
        d(y_{x_1}, y_{x_2}) &= d(y_1, y_2) \\
        &= d(T_1(y_1), T_2(y_2)) \\
        &\leq d(T_1(y_1), T_2(y_2)) + d(T_1(y_2), T_2(y_2)) \\
        &\leq \alpha d(y_1, y_2) + d(T_1(y_2), y_2)
    \end{align*}
    und da \( 1-\alpha > 0 \), folgt 
    \[ d(y_1, y_2) \leq \frac{1}{1-\alpha} d(T_1(y_2),y_2) 
    \overset{x_1 \rightarrow x_2}{\longrightarrow} 0. \]
    Also folgt die Stetigkeit.
\end{bew}
\begin{bew}(von Satz 1)
    Setze \(\abb{L = D_2F(x_0, y_0) = D_yF(x_0,y_0)}{\R^m}{\R^m}\) 
    ist invertierbar. O. B. d. A. sei \( c = 0 \), sonst 
    \( \tilde{F} := F(x,y) - c = 0 \).\\
    Die Gleichung \( F(x,y) = c \) sei äquivalent zu 
    \[ y = y - L^{-1}F(x,y) =: G(x,y) \]
    \(\Rightarrow\) benötigen zu festem \(x\) einen Fixpunkt
    der Abbildung \(y \mapsto G(x, y)\) (d. h. \( T_x := G(x, \cdot) \)).
    Kontraktionseigenschaft von \(T_x\):\\
    Da \(LL^{-1} = \id_{\R^m} =: \mathds{1}_{\R^m} \), folgt 
    \begin{align*}
        T_x(y_1)-T_x(y_2) &= G(x,y_1)-G(x,y_2) \\
        &= y_1 -L^{-1}F(x,y_1) -y_2+L^{-1}F(x,y_2) \\
        &= y_1 -y_2 - L^{-1}(F(x,y_1)-F(x,y_2)) \\
        &= L^{-1}(L(y_1-y_2)-(F(x,y_1)-F(x,y_2)))
    \end{align*}
    Beachte:
    \begin{align*}
        F(x,y_1)-F(x,y_2)-L(y_1,y_2) 
        &= F(x,y_1)-F(x,y_2) - D_2F(x_0,y_0)[y_1,y_2] \\
        &= F(x,y_1)-F(x,y_2) - D_1F(x_0,y_0)[x-x]
        -D_2F(x_0,y_0)[y_1-y_2]
    \end{align*}
    Seien jetzt \(\delta_1, \eta > 0\), sodass
    \[ M := B_{\delta_1}^{\R^n}(x_0) \times B_\eta^{\R^m}(y_0) \subset U \]
    Aus Korollar 11.3 folgt nun 
    \begin{align*}
        &\abs{ F(x,y_1) - F(x, y_2) - L(y_1 - y_2) } \\
        &\leq \abs{ y_1 - y_2 } \underbrace{\underset{(x,y) \in M}{\sup} 
        \norm{DF(x,y)-DF(x_0,y_0)}}_{(+)}
        \overset{\delta_1, \eta \rightarrow 0}{\rightarrow} 0
    \end{align*}
    Wähle \( \delta_1, \eta > 0 \), sodass \( (+)_{\delta_1, \eta} 
    \leq \frac{1}{2 \norm{L^{-1}}} \).\\
    \( \forall x \in B_{\delta_1}(x_0), y \in B_{\eta}(y_0) \) ist 
    \begin{align*}
        \abs{ T_x(y_1) - T_x(y_2) } &= \abs{ G(x,y_1) - G(x,y_2) } \\
        &\leq \norm{L^{-1}} \frac{1}{2 \norm{L^{-1}}} \abs{y_1 - y_2} \\
        &= \frac{1}{2} \abs{y_1-y_2}
    \end{align*}
    Somit ist \(T_x\) eine \(\frac{1}{2}-\)Kontraktion für jedes
    \(x \in B_{\delta_1}(x_0)\).\\
    Gilt \( \abb{T_x}{\overline{B}_\eta^{\R^m} (y_0) }{\overline{B}_\eta(y_0)} \)?\\
    Da \( G(x_0, y_0) = y_0 \)
    \begin{align*}
        \abs{ T_x(y) - y_0 } &= \abs{ G(x,y) - y_0 } \\
        &= \abs{ G(x,y) - G(x_0, y_0) } \\
        &\leq \underbrace{\abs{G(x,y)-G(x,y_0)}}_{\leq \frac{1}{2}\abs{y-y_0}} 
        + \abs{G(x,y_0)-G(x_0,y_0)} 
    \end{align*}
    Wähle \(\delta_2 > 0\), sodass \(\abs{G(x,y_0)-G(x_0,y_0)} < \frac{\eta}{2}\)
    für \(x \in B_{\delta_2}(x_0)\).\\
    Sei \(\delta := \min \set{\delta_1, \delta_2} > 0\)
    \begin{align*}
        \Rightarrow \abs{T_x(y)-y_0} &< \frac{1}{2} \abs{y-y_0} + \frac{\eta}{2}\\
        &\leq \eta
    \end{align*}
    \(\forall y \in \overline{B}_\eta(y_0), x \in \overline{B}_\delta(x_0)\)    
    \[ \Rightarrow \abb{ T_x }{ \overline{B}_\eta(y_0) }{B_\eta(y_0) 
    \subset \overline{B}_\eta(y_0)} \]
    \[ \oversett{Lem. 2}{\Rightarrow} \forall x \in B_\delta(x_0) 
    \; \exists ! y : y = y_x =: g(x) \in B_\eta(y_0) \]
    \[ F(x,g(x)) = F(x,y_x) = 0 \]
    Außerdem gilt \( \abb{g}{\overline{B}_\delta(x_0)}{\overline{B}_\eta(y_0)} \) 
    ist stetig. \\
    Es gilt sogar \(\abb{g}{\overline{B}_\delta(x_0)}{B_\eta(y_0}\) (siehe (\(*\))),
    denn 
    \[ \abs{g(x)-y_0} < \frac{1}{2}\abs{g(x)-y_0}+ \frac{\eta}{2} < \eta 
    \; \forall x \in \overline{B}_\delta(x_0) \]
    d. h. \(g(x) \in B_\eta(y_0) \; \forall x \in \overline{B}_\delta(x_0)\)\\
    \( \Rightarrow \abb{ g }{B_\delta(x_0)}{B_\eta(y_0)} \) ist stetig 
    und \( \forall y \in B_\eta(y_0) \) und \( \forall x \in B_\delta(x_0) \) 
    gilt: 
    \[ F(x,y) = 0 \Leftrightarrow y = g(x). \]
    Setze \(U_1 := B_\delta(x_0) \subset \R^n, U_2 := B_\eta(y_0) \subset \R^m\).\\
    Nun bleibt noch zu zeigen, dass 
    \( \abb{g}{U_1}{U_2} \) stetig differenzierbar ist.\\
    Sei \( (x_1, y_1) \in U_1 \times U_2 \), \( y_1 = g(x_1) \).\\
    Da \(F\) differenzierbar ist in \( (x_1, y_1) \), folgt 
    \[ F(x,y) = F(x_1, y_1) + DF(x_1, y_1)[\begin{pmatrix}
        x - x_1 \\ y - y_1
    \end{pmatrix}] 
    + \varphi(x,y) (\abs{ x - x_1 }^2 + \abs{y - y_1}^2)^{\frac{1}{2}} \]
    \[ = F(x_1,y_1) + D_1 F(x_1, y_1)[x - x_1]
    + D_2 F(x_1, y_1)[y - y_1]
    + \varphi(x,y)(\abs{x-x_1}^2 + \abs{y - y_1}^2)^{\frac{1}{2}} \]
    mit \( \limesx{(x,y)}{(x_1, y_1)} \varphi(x,y) = 0 \).\\
    Da \( (x,y) \mapsto D_2F(x,y) \) stetig und 
    \( D_2 F(x_0, y_0) \) invertierbar ist, können wir 
    \( \delta, \eta > 0 \) so klein wählen, dass 
    \( \abb{D_2 F(x,y) }{\R^m}{\R^m} \) invertierbar ist 
    für alle \( (x,y) \in B_\delta(x_0) \times B_\eta(y_0) \)
    (siehe Lemma 3).\\
    \(\Rightarrow L := D_2F(x_1,y_1)\) ist invertierbar für 
    \((x_1,y_1) \in B_\delta(x_0) \times B_\eta(y_0)\).\\
    Setze \( h = D_1F(x_1,y_1) \).\\
    \[ \Rightarrow F(x,y) = F(x_1, y_1) + h[x-x_1] + L[y - y_1] 
    + \varphi(x,y) (\abs{ x - x_1 }^2 + \abs{ y - y_1 }^2)^{1/2} \]
    \[ \Rightarrow y = y_1 - L^{-1}h[x - x_1] - 
    (\abs{ x - x_1 }^2 + \abs{y - y_1}^2)^{1/2} L^{-1}[\varphi(x,y)] \]
    \[ \Rightarrow g(x) = g(x_1) - 
    L^{-1}h[x-x_1] - \abs{\begin{pmatrix}
        x-x_1 \\ y - y_1
    \end{pmatrix}} L^{-1}[\varphi(x,y)] \tag{\(**\)} \]
    \[ \overset{(**)}{\Rightarrow} \abs{g(x) - g(x_1)} 
    \leq \norm{ L^{-1}h }\abs{ x - x_1 } + L^{-1}[\varphi(x,y)] 
    \cdot ( \abs{x - x_1}^2 + \abs{ g(x) - g(x_1) }^2)^{1/2} \]
    Da 
    \begin{align*}
        (\alpha^2+\beta^2)^{1/2} &\leq 2 \max (\alpha^2,\beta^2)^{1/2} = 2^{1/2} \max(\alpha, \beta) \\
        &\leq \sqrt{2} (\alpha+\beta)
    \end{align*}
    \(\forall \alpha, \beta > 0\)\\
    folgt 
    \[ ( \abs{x - x_1}^2 + \abs{ g(x) - g(x_1) }^2 )^{1/2}
    \leq \sqrt{2} \abs{ x - x_1 } + \sqrt{2} \abs{ g(x) - g(x_1) } \]
    \[ \Rightarrow \abs{g(x)-g(x_1)} \leq \norm{L^{-1}h} \abs{x-x_1}+\abs{L^{-1}[\varphi(x,y)]} 
    \sqrt{2} \abs{x-x_1} + \abs{L^{-1}[\varphi(x,y)]}\sqrt{2} \abs{g(x)-g(x_1)} \tag{***} \]
    ACHTUNG! \( x \mapsto g(x) \) ist stetig und \( \varphi(x,y) \rightarrow 0 \) 
    für \( x \rightarrow x_1, y \rightarrow y_1 \)
    \[ \Rightarrow \varphi(x,y) \rightarrow 0, x \rightarrow x_1. \]
    \[ \Rightarrow \abs{ L^{-1}[\varphi(x,y)] } \rightarrow 0 \text{ für } x \rightarrow x_1 \]
    Also existiert \( \gamma > 0 \) mit \( B_\gamma(x_1) \subset 
    B_\delta(x_0) \) und 
    \[ \sqrt{2} \abs{L^{-1}[\varphi(x,y)]} \leq \frac{1}{2} 
    \;\forall x \in B_\gamma(x_1) \]
    \begin{align*}
        \overset{(***)}{\Rightarrow} 
        \abs{ g(x) - g(x_1) } &\leq 2 \norm{ L^{-1}h }\abs{x - x_1} 
        + \abs{x - x_1} \\
        &= (2 \norm{L^{-1} h} + 1)\abs{x-x_1} \;\forall x \in B_\gamma(x_1) \\
        &= \beta\abs{x - x_1}
    \end{align*}
    d. h. \(g\) ist (lokal) Lipschitz-stetig.
    \begin{align*}
        \overset{(**)}{\Rightarrow} \abs{ g(x) - g(x_1) + L^{-1}h \abs{x - x_1} } 
        &= \abs{ L^{-1}[\varphi(x,g(x))] } (\abs{ x - x_1 }^2 + \abs{ g(x) - g(x_1) }^2)^{1/2} \\
        &\leq (1 + \beta^2)^{1/2} \abs{ L^{-1}[\varphi(x, g(x))] } \abs{x - x_1} 
        \;\forall x \in B_\gamma(x_1)
    \end{align*}
    \( \Rightarrow \forall x \in B_\gamma(x_1), x \neq x_1 \)
    \[ \abs{\frac{g(x)-g(x_1)+ L^{-1}h\abs{x-x_1}}{\abs{x-x_1}}}  \leq (1+\beta^2)^{1/2} 
    \underbrace{\abs{L^{-1}[\varphi(x,y)]} }_{\rightarrow 0,\ x \rightarrow x_1} \]
    \[ \Rightarrow \limesx{x}{x_1} \frac{g(x)-g(x_1)+L^{-1}h\abs{x-x_1}}{\abs{x-x_1}} = 0 \]
    d. h. \(g\) ist differenzierbar in \(x_1\) und die Ableitung 
    ist \( -L^{-1}h \), d. h. 
    \[ Dg(x_1) = -L^{-1}h = -(D_y F(x,g(x)) )^{-1} D_x F(x,g(x))
    = -(D_2 F(x,g(x)))^{-1}D_1 F(x, g(x)). \]
    Somit ist \(g\) stetig differenzierbar und die obige Formel gilt für \(Dg\).
\end{bew}
Brauchten im Beweis des Satzes über implizite Funktionen:
\( A \in \mathcal{L}(\R^m, \R^m) \) invertierbar. 
\( \Rightarrow \exists \) offene Umgebung \( U(A) \)
so, dass \( \forall B \in U: B \) invertierbar.
\begin{lem}
    Sei \( \abb{A}{\R^m}{\R^m} \) linear und invertierbar und 
    \( \abb{B}{\R^m}{\R^m} \) linear mit \(\norm{B-A} \leq 1\) (Operatornorm). \\
    Dann ist \( B \) invertierbar!
\end{lem}
\begin{bem}
    Erinnerung: Operatornorm \(   \)
    \[\norm{A} := \underset{\substack{h \in \R^m \\ \abs{h} = 1}}{\sup} \abs{A h} 
    = \underset{\substack{h \in \R^n \\ h \neq 0}}{\sup} \frac{ \abs{Ah} }{\abs{h}}
    = \underset{\substack{h \in \R^m \\ \abs{h} \leq 1}}{\sup} \abs{A h} \]
    \[ \norm{A} \leq \norm{A}_{HS} 
    = \left( \sum_{j,k = 1}^n A_{j,k}^2 \right)^{1/2} \]
\end{bem}
\begin{bew}
    1. Schritt: Ist \(C \in \mathcal{L}(\R^m, \R^m), \norm{C} \leq 1\)

    \( \Rightarrow \mathds{1} - C \) ist invertierbar.\\
    Idee: Geometrische Reihe zum Laufen bringen. \( \frac{1}{1-X} = \sum_{n=0}^\infty X^n \)
    \[ C^0 := \mathds{1}, C^1 := C\]
    \[ C^{n+1} := C C^n, n \in \N_0 \]
    \[ S_l := \sum_{n=0}^l C^n \]
    \[ (\mathds{1} - C)S_l = S_l - C S_l 
    = S_l - C \sum_{n=0}^l C^{n+1} 
    = \sum_{n=0}^l C^{n} - \sum_{n=0}^{l+1} C^n
    = C^0 - C^{l+1} = \mathds{1} - C^{l+1}. \tag{\(*\)}\]
    Genauso: \( S_l(\mathds{1} - C) = \mathds{1} - C^{l+1} \).\\
    Beachte: \[A, B \in \mathcal{L}(\R^m, \R^m), 
    \norm{A B} \leq \norm{A} \norm{B}. \]
    \[\abs{ABh} \leq \norm{A} \abs{Bh} \leq \norm{B} \abs{h} \quad (h \in \R^m) \]
    \[\Rightarrow \norm{C^n} = \norm{C C^{n-1}} \leq \norm{C} \norm{C^{n-1}}
    \leq ... \leq \norm{C}^n\]
    Sei \(h \in \R^m\). \( S_l h = \sum_{n=0}^l C^l h \).
    Behauptung: absolut konvergente Reihe.\\
    Betrachte \( \sum_{n=0}^l \abs{ C^n h } 
    \leq \sum_{n=0}^l \norm{C}^n \abs{h} \) konvergiert für \( l \rightarrow \infty \).
    \[ \sum_{n=0}^\infty \norm{C}^n \abs{h} \frac{1}{1 - \norm{C}} \abs{h} < \infty, 
    \text{ da } \norm{C} < 1. \]
    \( \Rightarrow \sum_{n=0}^\infty C^n h \) ist absolut konvergente Reihe in
    \( \R^m \) für jedes \( h \in \R^m \).
    Wir setzen \( S_\infty h := \sum{n=0}^\infty C^n h (h \in \R^m)\)
    \[ \abs{ S_\infty h } \leq \sum_{n=0}^\infty \abs{ C^n h } 
    \leq \frac{1}{1 - \norm{C}}\abs{h} < \infty. \]
    \[ \Rightarrow \norm{S_\infty} = \underset{\abs{h} = 1}{\sup} 
    \abs{ S_\infty h } \leq \frac{1}{1 - \norm{C}} < \infty, 
    \text{ da } \norm{C} < 1. \]
    \[ S_\infty(h_1 + h_2) 
    = \limes{l} S_l(h_1 + h_2) 
    = \limes{l} (S_l h_1 + S_l h_2)
    = S_\infty h_1 + S_\infty h_2. \]
    \[ \Rightarrow S_\infty \in \mathcal{L}(\R^m, \R^m). \]
    Beh: \( S_\infty = (\mathds{1} -C)^{-1} \): 
    \[ \overset{(*)}{\Rightarrow}  ( \mathds{1} - C)S_l h = (\mathds{1} - C^{l+1})h 
    = h - C^{l+1}h \]
    \[ S_l (\mathds{1} - C) h = h - C^{l+1}h \]
    \( \mathds{1} - C \) stetig.
    \[\Rightarrow \limes{l} (/mathds{1}-C) S_lh = (\mathds{1}-C) S_\infty h\]

    \[ \limes{l} C^{l+1} h = 0, \text{ denn } \abs{C^{l+1}h} \leq \norm{C^{l+1}} 
    \abs{h} \leq \norm{C}^{l+1} \abs{h} \rightarrow 0, l \rightarrow \infty. \]
    \[ \Rightarrow (\mathds{1} - C) S_\infty h = h \;\forall h \in \R^m \]
    Genauso \[S_\infty(\mathds{1}-C)h = \limes{l} S_l(\mathds{1}-C)h 
    = \limes{l}(h-C^{l+1 h}) = h\]
    \( \Rightarrow S_\infty \) ist das Inverse zu \( \mathds{1} - C \). 
    \[ S_\infty = \sum_{n=0}^\infty C^n \text{ (Von Neumann Reihe)}. \]
    
    2. Schritt: Sei \( A \in \mathcal{L}(\R^m, \R^m) \) invertierbar.\\
    \[ B \in \mathcal{L}(\R^m, \R^m), B = A - (A - B) 
    = A (\mathds{1} - (\mathds{1} - A^{-1}B)) 
    = A (\mathds{1}- \underbrace{(A^{-1}(A-B))}_{=: C})
    = A(\mathds{1} - C). \]
    Falls \( \mathds{1} - C \) invertierbar ist, dann ist 
    \( A(\mathds{1} - C) \) invertierbar mit 
    \( (A(\mathds{1} - C))^{-1} = (\mathds{1} - C)^{-1} A^{-1} \).
    \[\Rightarrow \text{Ist } 1 > \norm{C} = \norm{A^{-1}(A-B)}\] 
    \[ \norm{ A^{-1} (A - B) } \leq \norm{A^{-1}} \norm{A - B} \]
    \( \Rightarrow \) In der offenen Umgebung \( U \) von \(A\)
    \[ U = \set{ B \in \mathcal{L}(\R^m, \R^m) : \norm{ A^{-1} (A - B) } < 1 } \] 
    ist B invertierbar.
\end{bew}
\begin{kor}
    Sei \( U \subset \R^n \) offen. 
    \[ \abb{A}{U}{\mathcal{L}(\R^m, \R^m)} \]
    stetig und \( \abb{A(x_0)}{\R^m}{\R^m} \) invertierbar 
    für ein \( x_0 \in U \). \( \Rightarrow \exists \) offene Umgebung 
    \( \tilde{U} \subset U \) von \( x_0 \), sodass \( \forall x \in \tilde{U} \): 
    \( A(x) \) ist invertierbar.
\end{kor}
\begin{bew}
    Nach Lemma 3 \( \exists \) offene Umgebung 
    \( \Omega \subset \mathcal{L}(\R^m, \R^m) \) 
    von \( A(x_0) \), sodass jedes \( B \in \Omega \) invertierbar ist.\\
    Setzen \( \tilde{U} := \set{x \in U: A(x) \subset \Omega} \).\\
    \( \Rightarrow \forall x \in \tilde{U}\) ist \(\abb{A(x)}{\R^m}{\R^m}\) invertierbar.
\end{bew}
\begin{bem}
    Ist \( A \in \mathcal{L}(\R^m, \R^m) \), so können wir 
    \( A \) durch eine (\(m \times m\))-Matrix beschreiben.\\
    \( A \) \gqq{\(=\)} \( (A_{j,l})_{j,l = 1,\ldots,m} \).
    \[ \norm{A} \leq \norm{A}_{HS} 
    = \left( \sum_{j,l = 1}^m \abs{A_{j,l}}^2 \right)^{1/2} \]
    Ist \(U\ni x \rightarrow A(x) \in \mathcal{L}(\R^m \R^m)\)
    \(\Rightarrow (A_{j,l}(x))_{j,l=1,\ldots,m}\) zugehörige Matrix.
    ist stetig in \( x_0 \): \( \limesx{x}{x_0} \norm{A(x) - A(x_0)} = 0 \).\\
    \[ \norm{ A(x) - A(x_0) }\leq \norm{ A(x) - A(x_0) }_{HS}
    = \left( \sum_{j,l=1}^m (A_{j,l}(x) - A_{j,l}(x_0))^2 \right)^{1/2}. \]
    \( \Rightarrow \) Ist für jedes \(j, l = 1,\ldots,m\) die Abb 
    \(U \ni x \mapsto A_{j,l}(x)\) stetig,
    so ist \( U \ni x \mapsto A(x) \) stetig.\\
    Umgekehrt: Es gilt 
    \[ \abs{ A_{j,l} } \leq \norm{A} \]
    \[D_y F(x,y) = (\partial_{y_1} F(x,y), \partial_{y_2} F(x,y), \ldots, 
    \partial_{y_m} F(x, y)) \]
    \[ \abs{A_{j,k}} 
    = \abs{ \scalarprod{e_j}{A e_k} } 
    \leq \abs{e_j} \abs{A e_k} 
    \leq \norm{A} \abs{e_k} = \norm{A}. \]
\end{bem}
\begin{satz}[Inverse Funktionen]
    Sei \( y_0 \in V \subset \R^n \), \( V \) offen, 
    \( f \in \mathcal{C}^1(v, \R^m) \). Ist \( Df (y_0) \) 
    invertierbar.
    Eine Umgebung U von \(x_0 := f(x_0) \) und eine \(\mathcal(C^1)\)-Funktion 
    \(\abb{g}{U}{\R^n}\) mit
    \[ f(g(x)) = x \;\forall x \in U. \]
    Außerdem ist \( g(U) := \tilde{V} \subset V \) eine offene Umgebung
    von \( y_0 \) und 
    \( \abb{g}{U}{\tilde{V}} \) ist bijektiv.
    \begin{bem}
        Also: Ist \(f(y) = x\) und \(x \in U, y \in \tilde(V) \Leftrightarrow y = g(x)\)
        \(g = f^{-1}\) d.h. g ist Inverse zu f
    \end{bem}
    Die Ableitung von \(g\) berechnet sich als 
    \( Dg(x) = Df(y)^{-1} \) mit \( f(y) = x \)
    \[ Df(g(x))^{-1} = (Df(f^{-1}(x)))^{-1}. \]
\end{satz}
\begin{bew}
    Setzen \( F(x,y) := f(y) - x \).
    \[ \Rightarrow F(x_0, y_0) = \underbrace{f(y_0)}_{=x_0} - x_0 = 0 \]
    \[ D_y F(x,y) = D_2 F(x,y) = Df(y) \]
    \[ D_y F(x_0, y_0) 
    = \abb{Df(y_0)}{\R^n}{\R^n} \]
    ist invertierbar.\\
    Anwendung des Satzes über implizite Funktionen.\\
    \( \exists \abb{g}{U}{\tilde{V}}, U \) offene Umgebung 
    von \(y_0\), \(\tilde{V}\) offene Umgebung von \( x_0 = f(y_0) \)
    mit \(0= F(x, g(x)) = f(g(x))-x\), d. h. \( x = f(g(x)) \;\forall x \in U \).

    und \(g\) ist \(\mathcal{C}^1\).\\
    Formel für \(Dg(x)\): Haben \(x = f(g(x)) \;\forall x\in U\).
    \[ \Rightarrow \mathds{1}_n = D_x(x) = D(f(g(x))) 
    = Df(g(x)) \circ D g(x) \]
    \[ \Rightarrow D g(x) = D f(g(x))^{-1}(Df(g(x)) \circ Dg(x))
    = (Df(y))^{-1} \mathds{1}_n = Df(y)^{-1}, g(x) = y. \]
\end{bew}
\end{document}