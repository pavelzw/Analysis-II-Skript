\documentclass[../ana2.tex]{subfiles}

\begin{document}
\setcounter{section}{16}
\section{Satz über implizite Funktionen (und inverse Funktionen)}
Situation: Sei \(\R^n, \R^m\) und \(\abb{F}{\R^n \times \R^m}{\R^m}\) gegeben. 
Angenommen, es existiert \(x_0 \in \R^n\), \(y_0 \in \R^m\):
\[ F(x_0, y_0) = 0. \]
Frage: Können wir die \( m \) Gleichungen lokal 
(d. h. in der Nähe von \( x_0 \)) nach \(y\) 
auflösen, d. h. gibt es eine Funktion 
\( \abb{g}{\R^n}{\R^m} \) mit 
\[ F(x, g(x)) = 0 \; \forall x \text{ nahe bei } x_0 \]
\subsection*{Linearisierung} 
\[ F(x,y) = \underbrace{F(x_0, y_0)}_{=0} + \underbrace{DF(x_0, y_0) \begin{pmatrix}
    x - x_0 \\ y - y_0
\end{pmatrix} }_{
    = D_1 F(x_0, y_0)[x - x_0] + D_2 F(x_0, y_0)[y - y_0]
}+ \text{ höhere Ordnungsterme}\]
mit \\
\( D_1 F = D_x F =  \) (verallgemeinerte) 
partielle Ableitung nach \( x_1,\ldots, x_n \)\\
\( D_2F = D_yF = \) (verallgemeinerte) 
partielle Ableitung nach \(y_1, \ldots , y_n \)
\[ A = \abb{D_1 F(x_0, y_0)}{\R^n}{\R^n} \text{ linear} \]
\[ B = \abb{D_2 F(x_0, y_0)}{\R^m}{\R^m} \text{ linear} \]
\( \Rightarrow \) linearer Fall: 
\(0 = F(x, y) = A[x-x_0] + B[y-y_0]\)
\[ \Rightarrow y = y_0 - B^{-1}A[x - x_0] \]
ist die einzige Lösung, sofern \( B \) invertierbar ist.
\begin{satz}[Satz über implizite Funktionen]
    Gegeben \(\R^n, \R^m, U \subset \R^n \times \R^m\) offen, 
    \((x_0, y_0) \in U\) und \(\abb{F}{U}{\R^m}\) stetig differenzierbar
    in \(U\). Ist
    \[ F(x_0, y_0) = c \in \R^m \]
    und sind die verallgemeinerten partiellen Ableitungen von \(F\)
    nach \(y\), also
    \[ \abb{D_yF(x_0,y_0) = D_2F(x_0, y_0)}{\R^m}{\R^m} \]
    invertierbar, dann existieren offene Umgebungen \(\R^n \supset U_1\)
    von \(x_0\) und \(\R^m \supset U_2\) von \(y_0, U_1 \times U_2 \subset U\)
    und eine differenzierbare Funktion \(\abb{g}{U_1}{U_2}\) mit 
    \[F(x, g(x)) = c \; \forall x \in U_1\]
    Ferner ist 
    \[ D g(x) = -(D_2 F(x, g(x)))^{-1} \ D_1 F(x, g(x)) \]
    für alle \( x\in U_1 \). Außerdem ist für \( x \in U_1, y \in U_2 \)
    \[ F(x,y) = c \Leftrightarrow y = g(x). \]
    D. h. \(g(x)\) ist die einzige Lösung der Gleichung
    \[F(x, y) = c\]
\end{satz}
Für den Beweis brauchen wir eine leichte Verschärfung von
Banach.
\begin{lem}
    Sei \( M \) ein vollständiger metrischer Raum und \( P \) 
    (Parameterraum) ein weiterer metrischer Raum und für \( x\in P \)     
    sei \(\abb{T_x}{M}{M}\) eine \(\alpha-\) Kontraktion mit \(\alpha < 1\)
    wobei \(\alpha\) unabhängig von \(x\) ist. Dann exisitert für jedes \(x \in P\)
    ein Fixpunkt \(y_x \in M\) von \(T_x\), d. h. 
    \[ y_x = T_x(y_x) \]
    Ferner: Ist für alle \(y \in M\) die Abbildung 
    \(P \ni x \mapsto T_x(y)\) stetig, so ist
    \( x \mapsto y_x \) stetig.
\end{lem}
\begin{bew}
    Aus Satz 16.2 folgt \( \forall x \in P \) 
    existiert \( y_x = T_x(y,x) \), also ein Fixpunkt.\\
    Stetigkeit des Fixpunktes in dem Parameter \( x \): 
    Seien \( x_1, x_2 \in P, y_1 = y_{x_1}, y_2 = y_{x_2}, 
    T_1 = T_{x_1}, T_2 = T_{x_2} \).
    \begin{align*}
        d(y_{x_1}, y_{x_2}) &= d(y_1, y_2) \\
        &= d(T_1(y_1), T_2(y_2)) \\
        &\leq d(T_1(y_1), T_2(y_2)) + d(T_1(y_2), T_2(y_2)) \\
        &\leq \alpha d(y_1, y_2) + d(T_1(y_2), y_2)
    \end{align*}
    und da \( 1-\alpha > 0 \), folgt 
    \[ d(y_1, y_2) \leq \frac{1}{1-\alpha} d(T_1(y_2),y_2) 
    \overset{x_1 \rightarrow x_2}{\longrightarrow} 0. \]
    Also folgt die Stetigkeit.
\end{bew}
\end{document}