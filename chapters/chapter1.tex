%23.04.2019
\documentclass[../ana2.tex]{subfiles}
\begin{document}

\setcounter{section}{0}
\section{Der Satz von Taylor}

Ist \( \abb{f}{I = (c,d)}{\R} \) differenzierbar in 
\( a \in I \) \\
\[ \Leftrightarrow f(x) = f(a) + f'(a)(x-a) + R(x), 
\limesx{x}{a} \frac{R(x)}{x-a} = 0. \]
Frage: geht dies besser? \( \abb{f}{I}{\R} \) 
\gqq{nette} Funktion, \( a \in I \).
Existiert ein Polynom \( p \) vom Grad \( n\in\N \) 
so, dass 
\[ f(x) = p(x) + R(x), 
\limesx{x}{a} \frac{R(x)}{{(x-a)}^n} = 0? \]
Falls ja, wie bezeichnet man \( p \)?

Spezialfall: \(f\) ist ein Polynom. 
\( f(x) = \sum_{k=0}^n b_k x^k \). \\
\( f \) ist umschreibbar in ein Polynom von \( x-a \). \\
\begin{align*}
    f(x) &= \sum_{k=0}^n b_k \underbrace{{(x - a + a)}^k}_{
        = \sum_{l=0}^k \binom{k}{l} {(x-a)}^l a^{k-l}
    } \\
    &= \sum_{\substack{0 \leq k \leq n \\ 0 \leq l \leq k}} %Indizes?
    b_k \binom{k}{l} a^{k-l} {(x-a)}^l \\
    &= \sum_{l=0}^n \left( \sum_{k=l}^n 
    b_k \binom{k}{l} a^{k-l} \right) {(x-a)}^l \\
    &= \sum_{l=0}^n c_l {(x-a)}^l.\\
    &= \sum_{l=0}^n \frac{f^{(l)}(a)}{l!} {(x-a)}^l 
    \text{ umentwickeltes Polynom}. \\
\end{align*}
\begin{align*}
    c_0 &= \sum_{k=0}^n b_k a^k = f(a) \\
    c_1 &= \sum_{k=1}^n b_k k a^{k-1} = f'(a) \\
    c_2 &= \sum_{k=2}^n b_k \frac{k(k-1)}{2} a^{k-2} 
    = \frac{1}{2} f''(a) \\
    c_3 &= \sum_{k=3}^n b_k 
    \underbrace{\binom{k}{3}}_{
        \substack{= \frac{k!}{3!(k-3)!} 
        \\= \frac{k(k-1)(k-2)}{3 \cdot 2} }
    } a^{k-3}
    = \frac{1}{3!} f''(a)
\end{align*}
\[ \oversett{Induktion}{\Rightarrow} c_l 
= \frac{1}{l!} f^{(l)}(a) 
\Leftrightarrow f^{(l)}(a) = l!c_l \]

\begin{lem}[Erste Version von Taylor]
    Sei \(n \in \N, \abb{f}{I=(c,d)}{\R}\) \((n+1)\)-mal 
    differenzierbar. Dann gibt es zu jedem 
    \( a \in I \) und \( x\in I \) ein 
    \( \zeta \) zwischen \(a\) und \(x\) 
    (d.\ h.\ ist \( a < x \Rightarrow a < \zeta < x, 
    x < a \Rightarrow x < \zeta < a \)) so, dass 
    \[ f(x) = \sum_{l=0}^n \frac{f^{(l)}(a)}{l!} {(x-a)}^l 
    + \underbrace{ \frac{f^{(n+1)}(\zeta)}{(n+1)!} {(x-a)}^{n+1} 
    }_{ \text{\gqq{Fehler}} }. \]
\end{lem}
\begin{bew}
    Folgt aus Satz von Rolle! \\
    Hilfsfunktion: Sei \( a \leq t \leq x \) 
    (oder \( x \leq t \leq a \)), \( m \in \R \). 
    \begin{align*}
        g(t) &:= f(x) - f(t) - f'(t)(x-t) - \cdots 
        - \frac{ f^{(n)}(t) }{ n! } {(x-t)}^n - m 
        \frac{ {(x-t)}^{n+1} }{ (n+1)! } \\
        &= f(x) - \sum_{l=0}^{n} 
        \frac{ f^{(l)}(t) }{ l! }{(x-t)}^l
        - m \frac{ {(x-t)}^{n+1} }{ (n+1)! }
    \end{align*}

    \( \Rightarrow g(x) = 0 \) legen \( m\in \R \) so fest, 
    dass auch \( g(a) = 0 \) ist. \\
    \( \oversett{Rolle}{\Rightarrow} \exists \zeta \) zwischen 
    \(a\) und \(x\) mit \( g'(\zeta) = 0 \). \\
    \[ g'(t) = 0 - \underbrace{ 
        f'(t) - \sum_{l=1}^n \left( 
        \frac{ f^{(l+1)}(t) }{ l! } {(x-t)}^l 
        - \frac{ f^{(l)}(t) }{ (l-1)! } {(x-t)}^{l-1}
    \right) }_{\text{Teleskopsumme}}
    + \frac{ m {(x-t)}^{n} }{ n! }. \]
    \[ \Rightarrow g'(t) = -\frac{ f^{(n+1)}(t) }{ n! } {(x - \zeta)}^n
    + m \frac{ {(x-t)}^n }{ n! } \]
    Mit Rolle:
    \[ 0 = g'(\zeta) = -\frac{ f^{(n+1)}(\zeta){(x-\zeta)}^n }{ n! } 
    + \frac{ m{(x-\zeta)}^n }{ n! } \]
    \[ \Rightarrow m = f^{(n+1)}(\zeta) \]
    \[ \Rightarrow g(t) = f(x) - \sum_{l=0}^{n} \frac{ f^{(l)}(t) }{ (l)! }
    \cdot {(x-t)}^l - \frac{ f^{(l+1)}(\zeta) }{ (n+1)! } {(x-t)}^{n+1} \]
    Setze \( t=a \Rightarrow g(a) = 0 \), löse nach \(f(x)\) auf.
\end{bew} 
\begin{defi}
    Sei \( \abb{f}{(c,d)}{\R} \) \(n\)-mal differenzierbar, 
    \( a \in (c,d) \). Dann heißt 
    \[ T_n(f,a)(x) := \sum_{l=0}^n 
    \frac{f^{(l)}(a)}{l!} {(x-a)}^l \]
    das \(n\)-te Taylorpolynom von \(f\) an der Stelle \(a\) 
    und \[ R_n(f,a) := f - T_n(f,a) \] heißt Restglied.
\end{defi}
\begin{satz}[Taylor]
    Sei \( \abb{f}{I}{\R} \) \( n \)-mal differenzierbar. Dann gibt es zu
    \( a \in I \) eine Funktion \( \abb{R_n(f,a)}{I}{\R} \) mit
    \[ f = T_n(f, a) + R_n(f,a) \]
    und \( R_n(f,a) \) ist gegeben durch
    \[ R_n(f,a)(x) = \frac{1}{n!} 
    \left( f^{(n)}(\zeta) - f^{(n)}(a) \right) {(x-a)}^n \]
    für ein \( \zeta \) zwischen \( a \) und \( x \). \\
    \( \zeta = \zeta(x) \)
\end{satz}
\begin{bew}
    Lemma 1
    \begin{align*}
        \Rightarrow f(x) &= T_{n-1}(f, a)(x) +
        \frac{ f^{(n)}(\zeta){(x-a)}^n }{ n! } \\
        &= T_n(f,a)(x) - \frac{ f^{(n)}(a) }{n!} {(x-a)}^n 
        + \frac{f^{(n)}(\zeta)}{n!} {(x-a)}^n.
    \end{align*}
\end{bew}
\begin{bem}
    Ist \( f^{(n)} \) stetig in \( a \). \\
    \( \zeta = \zeta(x), a < \zeta(x) < x \) oder \( x < \zeta(x) < a \)
    \[ \Rightarrow \abs{ \frac{ R_n(f,a)(x) }{ {(x-a)}^n } } 
    = \frac{1}{n!} \abs{ f^{(n)}(\zeta) - f^{(n)}(a) } 
    \rightarrow 0, x \rightarrow a. \]
    \[ \limesx{x}{a} \frac{ R_n(f,a)(x) }{ {(x-a)}^n } = 0. \]
    Fehler geht schneller als \( {(x-a)}^n \) gegen Null.
\end{bem}
\begin{defi}[Landau Symbole]
    \( I = (a,b) \) \\
    \( \abb{g,h}{I}{\R}, h > 0\) (\(a=-\infty, b=\infty \) erlaubt.)
    Dann heißt \( g = o(h), x \rightarrow a \) 
    (oder \( x \rightarrow b \)), falls 
    \[ \limesx{x}{a} \frac{g(x)}{h(x)} = 0. \]
    \zB{} \[ g = o(|x-a|^n), x \rightarrow a 
    \Leftrightarrow \limesx{x}{a} 
    \frac{g(x)}{|x-a|^n} = 0 \]

    \[ g = O(n), x \rightarrow a \]
    falls \[ \limessupx{x}{a} \abs{\frac{g(x)}{h(x)}} < \infty. \]
    \( \Leftrightarrow \exists \delta > 0, C < \infty: 
    \abs{ \frac{g(x)}{h(x)} } \leq C 
    \;\forall x \in (a - \delta, a + \delta) \)

    \[ g = o(1) \Leftrightarrow \limes{x} g(x) = 0 \]
    Satz von Taylor: Ist \( f^{(n)} \) in \( a \) stetig, so ist
    \( R_n(f,a) = o(|x-a|^n) \)
\end{defi}
\begin{kor}
    \(I\) offenes Intervall, \( \abb{f}{I}{\R} \) 
    \(n\)-mal stetig differenzierbar und \(a\in I\). 
    Dann gilt 
    \[ f(x) = T_n(f,a)(x) + o(\abs{x-a}^n). \]
\end{kor}
\begin{bsp}
    Logarithmus: \( f(x) = \ln(1+x) \). \\
    \( f'(x) = \frac{1}{x+1}, f''(x) = -(x+1)^{-2} \)
    \[ \oversett{Induktion}{\Rightarrow} f^{(n)}(x) 
    = {(-1)}^{n-1} (n-1)! {(1+x)}^{-n}. \]
    \begin{align*}
        \Rightarrow T_n(f,0)(x) &= \sum_{k=0}^{n} \frac{f^{(k)}(0)}{k!}x^k \\
        &= \sum_{k=0}^n \frac{{(-1)}^{k-1}}{k} x^k
    \end{align*}
    \[ \Rightarrow \ln(x+1) = \sum_{k=0}^n 
    \frac{ {(-1)}^{k-1} }{k} x^k + R_n(x) \]
    und 
    \[ R_n(x) = \frac{1}{n!} 
    \left(
        f^{(n)}(\zeta) - f^{(n)}(a)
    \right) x^n = \frac{{(-1)}^{n-1}}{n} 
    \left( (1 + \zeta)^{-n} - 1 \right) x^n. \]
    Oder (Lemma 1).
    \[ \ln(x+1) = \sum_{k=0}^{n} \frac{(-1)^k}{k!} x^k
    + R_{n+1}(x) \]   
    \begin{align*}
        R_{n+1}(x) &= \frac{f^{(n+1)}(\zeta)}{(n+1)!} x^{n+1} \\
        &= \frac{(-1)^n}{n+1} \cdot \frac{x^{n+1}}{(1+ \zeta)^{n+1}} \\
        &= \frac{(-1)^n}{n+1} 
        \cdot \left( \frac{x}{1+ \zeta} \right)^{n+1}
    \end{align*} 
    \( \zeta \) ist zwischen \(0\) und \(1\), \( x > -1 \).\\
    \( f(0) = \ln 1 = 0 \).\\
    \( n = 1 \): \( \ln(1+x) = x + R_2(x) 
    = x - \frac{1}{2} \frac{x^2}{(1+\zeta)^2} \).\\
    Ist \( 0 < x \leq 1 \): 
    \[ \abs{ R_{n+1}(x) } = \frac{1}{n+1} 
    \abs{\frac{x}{1+\zeta}}^n \leq \frac{1}{n+1} \rightarrow 0, 
    n\rightarrow \infty. \]
    \[ \Rightarrow \ln(1+x) = \limes{n} \sum_{k=1}^n 
    \frac{ {(-1)}^{k-1} }{ k } x^k 
    = \sum_{k=1}^\infty \frac{ {(-1)}^{k-1} }{k} x^k. \]
    \( -1 < x < 0 \):
    \begin{align*}
        \abs{ R_{n+1}(x) }
        &= \abs{ \frac{(-1)^n}{n+1} \cdot 
        \left( \frac{x}{1+\zeta} \right)^{n+1} } \\
        &= \frac{1}{n+1} \abs{\frac{x}{1+\zeta}}^{n+1} 
    \end{align*}
    brauchen: \( \abs{\frac{x}{1+\zeta}} \leq 1 \). \\
    Sei \( -\frac{1}{2} \leq x < 0: -\frac{1}{2} < \zeta < 0 \)    
    \[ \Rightarrow \frac{\abs{x}}{\abs{1+\zeta}} \leq \frac{\abs{x}}{1-\abs{\zeta}}
      \leq \frac{\frac{1}{2}}{1 - \frac{1}{2}} = 1. \]

    \[ \abs{ R_{n+1}(x) } \leq \frac{1}{n+1} 
    \rightarrow 0, n \rightarrow \infty. \]
    Folgerung: für \( -\frac{1}{2} \leq x \leq 1 \) ist 
    \[ \ln(1+x) = \sum_{k=1}^\infty 
    \frac{ {(-1)}^{k-1} }{k} x^k. \]
\end{bsp}
\( \abb{f}{I}{\R}, I = [c,d], a \in I. \)
\[ T_n(f,a) = \sum_{k=0}^n \frac{f^{(k)}(a)}{ k! } 
(x-a)^k \text{ Taylorpolynom } n\text{-ter Ordnung} \]

\[ R_n(f,a) := f - T_n(f, a) \]
\[ \oversett{immer}{\Rightarrow} f = T_n(f,a) + R_n(f,a) \]



%Zusatz?
\( \oversett{Lemma 1}{\Rightarrow} \exists \zeta(f,n,a,x) \)
\[ \frac{ R_n(f,g)(x) }{(x-a)^n} = \frac{1}{n!} 
[f^{(n)}(\zeta) - f^{(n)}(a)] (x-a)^n \]
\( \Rightarrow \frac{ \abs{R_n(f, x)} }{ \abs{x-a}^n } 
\leq \frac{1}{n!} \abs{ f^{(n)}(\zeta(x)) - f^{(n)}(a) } \)
mit \( R_n(f,a)(x) = \frac{1}{(n+1)!} 
f^{(n+1)}(\zeta(f,n,a,x)) (x-a)^{n+1} \) eine Darstellung 
des Fehlers (Lagrange).

Aus dem Satz: (eine andere Fehlerdarstellung)
\( R_n(f,x) = \frac{1}{n!} [f^{(n)}(\zeta(x)) - f^{(n)}(a)] \)
\[ \exists \zeta = \zeta(x) = \zeta(x,f,n,a) \]

\begin{bsp}
    \begin{align*}
        f(x) &= \ln(1+x), -1 < x < \infty
        f^{(j)}(x) &= (-1)^{j-1}(j-1)!(1+1)^{-j}        
    \end{align*}    
    \[ \overundersett{Fehlerdarstellung}{aus Lemma 1}{\Rightarrow} 
    \exists \zeta(x) \text{ zwischen } 0 \text{ und } x: \]
    \[ R_n(\ln(1+x),0) = \frac{(-1)^n}{(n+1)!} \cdot n! 
    \cdot (1+\zeta)^{-(n+1)}x^{n+1} \]
    leztes Mal war \( -\frac{1}{2} \leq x \leq 1 \)
    \[ \abs{R_n(f, 0)} \leq \frac{1}{n+1} \overset{n \rightarrow \infty}{\rightarrow} 0\]
\end{bsp}

\end{document}