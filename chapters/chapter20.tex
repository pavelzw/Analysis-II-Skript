\documentclass[../ana2.tex]{subfiles}

\begin{document}
\setcounter{section}{19}
\section{Etwas gewöhnliche Differentialgleichungen}
Allgemein \( t \in \R, \abb{y}{I}{\R^d} \) oder 
\( D \subset \R \times \R^d \times \ldots \times \R^d = R \times \R^{d(n+1)} \)
\( \abb{F}{D}{\R^d} \).\\
Dann heißt 
\[ F(t, y(t), y'(t) \ldots, y^{(n)}(t)) = 0 \]
gewöhnliche Differentialgleichung der Ordnung \(n\).
\begin{bsp}[Newton]
    \( m a = F \) (Kraft \(F\), Masse \(m\), Beschleunigung \(a\)).\\
    \( x(t) \) Ort\\
    \( \dot{x}(t) \) Geschwindigkeit \\
    \( a(t) = \ddot{x}(t) \) Beschleunigung\\
    Brauchen: Ort \(x(t_0)\), Geschwindigkeit \( \dot{x}(t_0) \)\\
    \( x(t) \in \R^3\), oder \( \R^2, \R^6, \R^{3N} \) mit \(N\): Teilchenanzahl
    \[G(t, x'') = x''(t) - \frac{F(t)}{m} = 0 \]
\end{bsp}
\begin{defi}
    \(n, d \in \N, D \subset \R \times \R^d \times \cdots \times \R^d = \R \times \R^{dn}\).\\
    \(\abb{f}{D}{\R^d}\)\\
    Eine Gleichung der Form 
    \[ y^{(n)}(t) = f(t, y(t), y'(t), \ldots, y^{(n-1)}(t)) \]
    explizite Form der gewöhnlichen Differentialgleichung \(n\)-ter Ordnung.\\
    Anfangsbedingung: \( y(t_0) = y_0, y'(t_0) = y'_0 
    \ldots y^{(n-1)}(t_0) = y^{(n-1)}(t_0) \in \R^d \).\\
    Ang. \( v(t) := (y(t), y'(t), \ldots, y^{(n-1)}(t)) = (v_1(t), \ldots, v_n(t)) \)\\
    \[ v'(t) = (v'_1(t), \ldots, v'_n(t)) 
    = (v_2(t), \ldots, v_n(t), f(t,v(t))) \]
    Vektorwertige gewöhnliche Differentialgleichung erster Ordnung.
\end{defi}
\begin{bsp}
    Newton 
    \[ \ddot{x}(t) = \frac{1}{m}F(t, v(t)), v(t) = \dot{x_0}(t) \]
    \[ \ddx{t} \begin{pmatrix}
        x(t) \\ v(t)
    \end{pmatrix} = \begin{pmatrix}
        \dot{x}(t) \\ \dot{v}(t)
    \end{pmatrix} = \begin{pmatrix}
        v(t) \\ \ddot{x}(t)
    \end{pmatrix} = \begin{pmatrix}
        v(t) \\ \frac{1}{m} F(t, x(t))
    \end{pmatrix} \]
    z. B. suche differenzierbare Funktionen 
    \[ \abb{y}{\R}{\R} \]
    mit \( y'(t) = a y(t) \quad a \in \R \).\\
    Z. B. \( y(t) = c e^{at} \) macht es (\(c \in \R\))!\\
    Ang. \( y'= a y \).\\
    \[ h(t) := y(t) e^{-at} = \frac{y(t)}{e^{at}} \]
    \[ h'(t) = y'(t) e^{-at} + y(t)(-a e^{-at})
    = ay(t) e^{-at} - ay(t) e^{-at} = 0 \]
    \( \Rightarrow h \) ist konstant, d. h. 
    \( y(t) e^{-at} = h(t) = h(0) 
    = y(0) e^{-a 0} = y(0) \)
    \[ \Rightarrow y(t) = y(0) e^{at} = c e^{at}. \]
    z.B. \(y'(t) = a(t) y(t)\)\\
    \(\Rightarrow\) jede Lösung ist von der Form
    \[ y(t) = c \exp(A(t)), A(t) = \integralx{a(s)}{0}{t}{s} \]
    Allgemeinere Situation:\\
    \(\abb{y}{I}{\R^d}, y'(t) = f(t, y(t))\)\\
    \( t_0 \in I, y(t_0) = y_0 \)\\
    Sei \(U \subset \R^d, \abb{f}{I\times U}{\R^d}\).\\
    Anfangswertproblem: \\
    Frage: Welche Bedingungen sollte \(f\) erfüllen, damit
    wir eine wenigstens lokal eindeutige Lösung haben?\\
    D. h. \( \exists \delta > 0: t_0 \pm \delta \in I \)\\
    Funktion \( \abb{y}{(t_0 - \delta, t_0 + \delta)}{U} \)
    differenzierbar.
    \[ y'(t) = f(t, y(t)) \;\forall t\in (t_0 - \delta, t_0 + \delta) \]
    und \( y(t_0) = y_0 \).\\
    Umschreiben: Angenommen \(\abb{y}{(t_0-\delta, t_0 + \delta)}{\R^d}\) ist Lösung. \((*)\)\\
    \[ y(t) - y(t_0) = \integralx{\ddx{s}y(s)}{t_0}{t}{s} 
    = \integralx{y'(s)}{t_0}{t}{s} 
    = \integralx{f(s, y(s))}{t_0}{t}{s} \]
    \[ \Rightarrow y(t) = y_0 + \integralx{f(s, y(s))}{t_0}{t}{s} \tag{\(**\)} \]
    Fix\gqq{punkt}gleichung.
\end{bsp}

\[ (*) \begin{cases}
    y' = f(t,y) \\
    t(t_0)
\end{cases} \]
\( \abb{y}{I}{\R^d} \) stetig diffenzierbar, 
\( t_0 \in I, y_0 \in U \subset \R^d \)\\
Integrieren von \( t_0 \) bis \(t\)
\[ y(t) = y(t_0) + \integralx{y'(s)}{t_0}{t}{s} 
= y_0 + \integralx{f(s, y(s))}{t_0}{t}{s} \tag{\(**\)} \]
\begin{lem}
    \(y\) ist stetig diffenzierbare Lösung von \((*) \Leftrightarrow y\)
    ist stetig und Lösung von \((**)\)\\
    \(\Rightarrow \) Raum in dem wir Lösungen suchen ist 
    \(\mathcal{C}([t_0-\delta, t_0+\delta], U)\)
    oder \(\mathcal{C}([t_0, t_0+\delta], U)\)\\
    O. B. d. A. \(t_0 \leq t \leq t_0 + \delta\)
\end{lem}
Fakt 1: \( h \in \mathcal{C}([t_0. t_0 + \delta], \R^d) \) 
hat Norm 
\[\norm{h}_\infty := \underset{t_0 \leq t \leq t_0 + \delta}{\sup} \abs{h(t)}, h(t) \in \R^d \]
\[ = \norm{h}_{\infty, I} I = [t_0, t_0 + \delta] \]
\begin{align*}
    \forall t \in [t_0, t_0 + \delta] : \abs{h_1(t) + h_2(t)} &\leq \abs{h_1(t)} + \abs{h_2(t)}\\
    &\leq \underset{t_0 \leq t \leq t_0 + \delta}{\sup} \abs{h_1(t)} 
    + \underset{t_0 \leq t \leq t_0 + \delta}{\sup} \abs{h_2(t)}\\
    &= \norm{h_1}_\infty + \norm{h_2}_\infty
\end{align*}
\[ \Rightarrow \norm{h_1 + h_2}_\infty \leq \norm{h_1}_\infty + \norm{h_2}_\infty \]
ist vollständig normierter Vektorraum.
\( y \in \mathcal{C}([t_0, t_0 + \delta], \R^d) \). Definiere 
\[ G(y)(t) := y_0 + \integralx{f(s, y(s))}{t_0}{t}{s} \]
\( G(y) \in \mathcal{C}([t_0, t_0 + \delta], \R^d) \) sofern alles gut geht.\\
Angenommen \(U \subset \R^d\) offen, \( y_0 \in U \).
\[ \Rightarrow r>0: \overline{B_r}(y_0) \subset B_{2r}(y_0) \subset U. \]
Rechteck \(R := [t_0, t_0+\delta] \times \overline{B_r}(y_0) \subset I \times U\)\\
\(\abb{f}{I\times U}{\R^d}\) stetig. 
\[ M := \underset{(t,z) \in R}{\max} \abs{f(t,z)} < \infty. \]
Angenommen \( y \in \mathcal{C}([t_0, t_0 + \delta], \overline{B_r}(y_0)) \)
\[ \abs{G(y)(t) - y_0} = \abs{\integralx{f(s, y(s))}{t_0}{t}{s}} 
\leq \integralx{\underbrace{\abs{f(s, y(s))}}_{\leq M}}{t_0}{t}{s} \leq M(t - t_0) \]

Setzen \( \alpha := \min(\delta_0, \frac{r}{M}) \)
\[ \Rightarrow \forall t_0 \leq t \leq \alpha : \abs{G(y)(t)- y_0} \leq 
M(t - t_0) \leq M \frac{r}{m} = r \]
\[ \Rightarrow \abb{ G }{ \mathcal{C}([t_0, t_0 + \alpha], \overline{B_r}(y_0)) }{ \mathcal{C}([t_0, t_0 + \alpha], \overline{B_r}(y_0)) } \]
ersteres ist eine abgeschlossene Teilmenge von \( \mathcal{C}([t_0, t_0 + \alpha], \R^d) \).\\
Große Frage: Ist dies eine Kontraktion? \( y_1, y_2 \in X_{t_0, \alpha} \)
\[ \norm{G(y_1) - G(y_2) }_{\infty, [t_0, t_0 + \alpha]} \overset{!}{\leq} 
L \norm{ y_1 - y }_{\infty, [t_0, t_0 + \alpha]} \text{ und } L < 1 \]
\[ G(y_1)(t) - G(y_2)(t) 
=\abs{ \integralx{f(s, y_1(s))}{t_0}{t}{s} - \integralx{f(s, y_2(s))}{t_0}{t}{s} } \]
\[ = \integralx{\abs{ f(s,y_1(s)) - f(s, y_2(s)) }}{t_0}{t}{s} \]
Zusatzbedingung: \( f \) ist (lokal) Lipschitzstetig in 2 Koordinaten.\\
Brauchen \( \abs{ f(s,z_1) - f(s, z_2) } 
\leq L\abs{z_1 - z_2} \;\forall t_0 \leq s \leq t_0 + \alpha, z_1, z_2 \in \overline{B_r}(y_0) \)
\[ \Rightarrow \abs{G(y_1)(t) - G(y_2)(t)} 
\leq L \integralx{ \abs{ y_1(s) - y_2(s) } }{t_0}{t}{s} 
\leq L \norm{ y_1 - y_2 }_{\infty, [t_0, t_0 + \alpha]} 
\cdot (t - t_0) 
\leq \alpha L \norm{y_1 - y_2}_{\infty, [t_0, t_0 + \alpha]} \]
\[ \Rightarrow \norm{G(y_1) - G(y_2)}_{\infty , [t_0, t_0 + \alpha]} 
\leq \alpha L \norm{ y_1 - y_2 }_{\infty, [t_0, t_0 + \alpha]} \]
Wähle \( \alpha = \min(\delta, \frac{r}{M}, \frac{1}{2L}) \)
\[ \Rightarrow \alpha L \leq \frac{1}{2} 
\Rightarrow \abb{G}{ \mathcal{C}([t_0, t_0 + \alpha], \overline{B_r}(y_0)) }{\mathcal{C}([t_0, t_0 + \alpha], \overline{B_r}(y_0))} \]
ist eine Kontraktion.\\
\( \oversett{Banach}{\Rightarrow} \exists! \) Funktion 
\( y \in \mathcal{C}([t_0, t_0 + \alpha], \overline{B_r}(y_0)) : y = G(y) \)\\
Frage: Braucht man die Lipschitzbedingung auf \(f\)?
Antwort: Ja, sonst geht Eindeutigkeit verloren!
\begin{bsp}
    \(f(t, z) = \abs{z}^{1/2}, z \in \R \)\\
    \[ f(s, z_1) - f(s, z_2) = \integralx{\ddx{s} f(t, (1-s)z_1 + s z_2)}{0}{1}{s} 
    \leq \integralx{ \scalarprod{\nabla_z f(t, (1-s)z_1+sz_2)}{z_2 - z_1} }{0}{1}{s} \]
    \[ \leq \underset{z \in B_r(y_0)}{\sup} \nabla_2 f(t,z) \abs{z_1 - z_2} \]
    \[ \ddxpartialboth{f}{z} = \frac{1}{2\abs{z}^{1/2}} \] 
    ist nicht Lipschitzstetig auf \(\R\).\\
    \[ y' = c\abs{y}^{1/2}, 
    y(0) = 0 \]
    Z. B. \( y(t) = 0 \;\forall t \in \R \).\\
    \( y(t) = t^{\alpha}, t \geq 0 \)
    \[ y' = \alpha t^{\alpha - 1} 
    \overset{!}{=} c t^{\alpha/2} \]
    \[ \alpha - 1 = \frac{\alpha}{2}, \alpha = 2. \]
    Aber auch 
    \[ y(t) = \begin{cases}
        0 & t < 0 \\
        t^2 & t \geq 0
    \end{cases} \] 
    ist auch eine Lösung von \( y' = 2 \abs{y}^{1/2} \)
    auch, \(y(t) = (\max(t, t_0))^2 \; t_0 \geq 0\)
    \(\Rightarrow y'=2\abs{y}^{1/2} y(0) = 0\).
\end{bsp}
Stetige Abhängigkeit von den Daten\\
\( t_0 \in \R \) est, Anfangsbedingung \( y(t_0) = y_0 \)\\
Gr\o{}nwallargument: \( y_0, y_1 \) zwei Startwerte.\\
\( \Rightarrow \) Lösung \( y_0(t), y_1(t) \) auf gemeinsamen Zeitintervall \( [t_0, t_0 + \alpha] \)
\[ y_0(t) - y_1(t) = y_0 + \integralx{ f(s, y_0(s)) }{t_0}{t}{s} - (y_1 + \integralx{f(s, y_1(s))}{t_0}{t}{s}) \]
\[ = y_0 - y_1 + \integralx{ (f(s, y_0(s)) - f(s, y_1(s))) }{t_0}{t}{s} \]
\( \Rightarrow w(t) := \abs{ y_0(t) - y_1(t) } 
\leq \abs{ y_0 - y_1 } 
+ \integralx{ \abs{ f(s, y_0(s)) - f(s, y_1(s)) } }{t_0}{t}{s} \)
\[ \leq \abs{ y_0 - y_1 } + L \integral{ \abs{ y_0(s) - y_1(s) } }{t_0}{t}{s} \]
\[= \abs{y_0 - 0} + L \integralx{w(s)}{t_0}{t}{s} \]
O. B. d. A. \( t_0 = 0 \)\\
Haben: \( 0 \leq w(t) \leq A + L \integralx{w(s)}{0}{t}{s} \)
\(0 \leq t \leq \alpha > 0\)
\[ \Rightarrow \dot{U}(t) = w(t) \leq A + L U(t) \]
\[ \Rightarrow \dot{U}(t) - LU(t) \leq A. \]
\( \alpha(t) = e^{-L t} \)\\
\( \dot{\alpha}(t) -L \alpha \)
\[ \ddx{t}(\alpha(t) U(t)) 
= \alpha(t) \dot{U}(t) + \dot{\alpha}(t) U(t) 
= \alpha(t) \dot{U}(t) - L \alpha(t) U(t)
= \alpha(t) (\dot{U}(t) - LU(t)) 
\leq A \alpha(t) = A e^{Lt} \]
\[ \oversett{Integrieren}{\Rightarrow} \alpha(t) U(t) - \alpha(0) U(0) 
= \integralx{ \ddx{s} (\alpha(s) U(s)) }{0}{t}{s} 
\leq A \integralx{ e^{Ls} }{0}{t}{s} 
= A \frac{1}{L} (1 - e^{Lt}) \]
\[ \Rightarrow U(t) \leq \frac{A}{L}(e^{Lt} - 1) \]
\[ \Rightarrow w(t) \leq A + L U(t) \leq A + L \frac{A}{L} (e^{Lt} - 1) \]
\[ \Rightarrow 0 \leq w(t) \leq A + A e^{Lt} - A = Ae^{Lt} \]
\[ \Rightarrow \abs{ y_0(t) - y_1(t) } \leq \abs{ y_0(0) - y_1(0) }e^{Lt}. \]
\end{document}