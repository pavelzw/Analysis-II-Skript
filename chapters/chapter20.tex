\documentclass[../ana2.tex]{subfiles}

\begin{document}
\setcounter{section}{19}
\section{Etwas gewöhnliche Differentialgleichungen}
Allgemein \( t \in \R, \abb{y}{I}{\R^d} \) oder 
\( D \subset \R \times \R^d \times \ldots \times \R^d = R \times \R^{d(n+1)} \)
\( \abb{F}{D}{\R^d} \).\\
Dann heißt 
\[ F(t, y(t), y'(t) \ldots, y^{(n)}(t)) = 0 \]
gewöhnliche Differentialgleichung der Ordnung \(n\).
\begin{bsp}[Newton]
    \( m a = F \) (Kraft \(F\), Masse \(m\), Beschleunigung \(a\)).\\
    \( x(t) \) Ort\\
    \( \dot{x}(t) \) Geschwindigkeit \\
    \( a(t) = \ddot{x}(t) \) Beschleunigung\\
    Brauchen: Ort \(x(t_0)\), Geschwindigkeit \( \dot{x}(t_0) \)\\
    \( x(t) \in \R^3\), oder \( \R^2, \R^6, \R^{3N} \) mit \(N\): Teilchenanzahl
    \[G(t, x'') = x''(t) - \frac{F(t)}{m} = 0 \]
\end{bsp}
\begin{defi}
    \(n, d \in \N, D \subset \R \times \R^d \times \ldots \times \R^d = \R \times \R^{dn}\).\\
    \(\abb{f}{D}{\R^d}\)\\
    Eine Gleichung der Form 
    \[ y^{(n)}(t) = f(t, y(t), y'(t), \ldots, y^{(n-1)}(t)) \]
    explizite Form der gewöhnlichen Differentialgleichung \(n\)-ter Ordnung.\\
    Anfangsbedingung: \( y(t_0) = y_0, y'(t_0) = y'_0 
    \ldots y^{(n-1)}(t_0) = y^{(n-1)}(t_0) \in \R^d \).\\
    Ang. \( v(t) := (y(t), y'(t), \ldots, y^{(n-1)}(t)) = (v_1(t), \ldots, v_n(t)) \)\\
    \[ v'(t) = (v'_1(t), \ldots, v'_n(t)) 
    = (v_2(t), \ldots, v_n(t), f(t,v(t))) \]
    Vektorwertige gewöhnliche Differentialgleichung erster Ordnung.\\
\end{defi}
\begin{bsp}
    Newton 
    \[ \ddot{x}(t) = \frac{1}{m}F(t, v(t)), v(t) = \dot{x_0}(t) \]
    \[ \ddx{t} \begin{pmatrix}
        x(t) \\ v(t)
    \end{pmatrix} = \begin{pmatrix}
        \dot{x}(t) \\ \dot{v}(t)
    \end{pmatrix} = \begin{pmatrix}
        v(t) \\ \ddot{x}(t)
    \end{pmatrix} = \begin{pmatrix}
        v(t) \\ \frac{1}{m} F(t, x(t))
    \end{pmatrix} \]
    z. B. suche differenzierbare Funktionen 
    \[ \abb{y}{\R}{\R} \]
    mit \( y'(t) = a y(t) \quad a \in \R \).\\
    Z. B. \( y(t) = c e^{at} \) macht es (\(c \in \R\))!\\
    Ang. \( y'= a y \).\\
    \[ h(t) := y(t) e^{-at} = \frac{y(t)}{e^{at}} \]
    \[ h'(t) = y'(t) e^{-at} + y(t)(-a e^{-at})
    = ay(t) e^{-at} - ay(t) e^{-at} = 0 \]
    \( \Rightarrow h \) ist konstant, d. h. 
    \( y(t) e^{-at} = h(t) = h(0) 
    = y(0) e^{-a 0} = y(0) \)
    \[ \Rightarrow y(t) = y(0) e^{at} = c e^{at}. \]
    z.B. \(y'(t) = a(t) y(t)\)\\
    \(\Rightarrow\) jede Lösung ist von der Form
    \[ y(t) = c \exp(A(t)), A(t) = \integralx{a(s)}{0}{t}{s} \]
    Allgemeinere Situation:\\
    \(\abb{y}{I}{\R^d}, y'(t) = f(t, y(t))\)\\
    \( t_0 \in I, y(t_0) = y_0 \)\\
    Sei \(U \subset \R^d, \abb{f}{I\times U}{\R^d}\).\\
    Anfangswertproblem: \\
    Frage: Welche Bedingungen sollte \(f\) erfüllen, damit
    wir eine wenigstens lokal eindeutige Lösung haben?\\
    D. h. \( \exists \delta > 0: t_0 \pm \delta \in I \)\\
    Funktion \( \abb{y}{(t_0 - \delta, t_0 + \delta)}{U} \)
    differenzierbar.
    \[ y'(t) = f(t, y(t)) \;\forall t\in (t_0 - \delta, t_0 + \delta) \]
    und \( y(t_0) = y_0 \).\\
    Umschreiben: Angenommen \(\abb{y}{(t_0-\delta, t_0 + \delta)}{\R^d}\) ist Lösung. \((*)\)\\
    \[ y(t) - y(t_0) = \integralx{\ddx{s}y(s)}{t_0}{t}{s} 
    = \integralx{y'(s)}{t_0}{t}{s} 
    = \integralx{f(s, y(s))}{t_0}{t}{s} \]
    \[ \Rightarrow y(t) = y_0 + \integralx{f(s, y(s))}{t_0}{t}{s} \tag{\(**\)} \]
    Fix\gqq{punkt}gleichung.
\end{bsp}
\end{document} 