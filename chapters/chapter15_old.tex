\documentclass[../ana2.tex]{subfiles}

\begin{document}
\setcounter{section}{14}
\section{Taylor}
\[ \abb{g}{[0,1]}{\R} \text{ (oder \(\R^m\))} \]
\( g \) stetig differenzierbar auf \( (0,1) \) 
\[ \Rightarrow g(1) - g(0) = \integralx{\ddx{s} g(s)}{0}{1}{s} \]
\[ g(1) = g(0) + \underbrace{\integralx{\ddx{s}s g'(s)}{0}{1}{s}}
_{\substack{[sg'(x)]_0^1 = \integralx{sg''(s)}{0}{1}{s} \\
= g'(1) - \integralx{sg''(s)}{0}{1}{s}}} \]
\[ = g(0) + g'(1) - \integralx{sg''(s)}{0}{1}{s} \]
Mist.\\
Besser: \( 1 = \ddx{s} (1 - s) \)
\[ \Rightarrow g(1) = g(0) + \integralx{1 \cdot g'(s)}{0}{1}{s} \]
\[ = g(0) + g'(0) + \integralx{(1-s)g''(s)}{0}{1}{s} \]
mit
\[ \integralx{1 \cdot g'(s)}{0}{1}{s}
= \integralx{-(\ddx{s} (1-s))g'(s)}{0}{1}{s} = \underbrace{[-(1-s)g'(s)]_0^1}_{g'(0)}
+ \integralx{(1-s)g''(s)}{0}{1}{s} \]
\[ g(1) = g(0) + g'(0) + \integralx{(1-s)g''(s)}{0}{1}{s} 
= g(0) + g'(0) + \integralx{-\ddx{s} (\frac{1}{2} (1-s)^2) g''(s)}{0}{1}{s} \]
\[ = g(0) + g'(0) + \frac{1}{2}g''(0) 
+ \frac{1}{2}\integralx{(1-s)^2 g'''(s)}{0}{1}{s} \]
\[ \integralx{(1-s)^2 \underbrace{g'''(s)}_{g'''(s) - g'''(0) + g'''(0)}}{0}{1}{s} \]
\[ = \integralx{(1-s)^2(g'''(s) - g'''(0)}{0}{1}{s} 
+ \underbrace{\integralx{(1-s)^2}{0}{1}{s} g'''(0) }_{=\frac{1}{2}} \]
\[ \Rightarrow g(1) = g(0) 
+ g'(0) + \frac{1}{2}g''(0) 
+ \frac{1}{2\cdot 3} g'''(0) 
+ \frac{1}{2} \integralx{(1-s)^2(g'''(s) - g'''(0))}{0}{1}{s} \]
Man nehme z.B.
\[ g(t) = f(x+tv) \]
Hatten \(x \neq 0: R(x) = \frac{\scalarprod{x, Ax}}{\scalarprod{x}{x}}\)
mit \(\abb{A}{\R^n}{\R^n}\) symmetrisch.
\[ \lambda_1 = \max R(x) \text{ größter Eigenwert von } A \]
\[ \lambda_n = \min R(x) \text{ kleinster Eigenwert von } A \]
\(\Rightarrow\) Man nehme irgendein \(x \neq 0\)
\begin{align*}
    \Rightarrow &\lambda_1 \geq R(x) = \frac{\scalarprod{x, Ax}}{\scalarprod{x}{x}}\\
    &\lambda_n \leq R(x).
\end{align*}
\begin{bsp}
    \( H = \frac{1}{2m} p^2 + V, p = -i \hbar\nabla \) Impulsoperator, 
    \( \abb{V}{\R^d}{\R} \) Potential \\
    Schödingeroperator: in \(L^2(\R^d) \scalarprod{f}{g} 
    = \integral{f(x)}{\R^d}{} \) in Masse \\
    \(h = \) Plancksches Wirkungsquantum, \( \hbar = \frac{h}{2\pi} \).\\
    \[ R(\varphi) = \frac{\scalarprod{\varphi}{H\varphi}}{\scalarprod{\varphi}{\varphi}} 
    = \frac{\scalarprod{\varphi}{(-\frac{\hbar^2}{2m} A + V)\varphi}}{\scalarprod{\varphi}{\varphi}} \]
    obere Schranke an die Grundzustandsenergie.
\end{bsp}
\end{document}