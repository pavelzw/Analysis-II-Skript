\documentclass[../ana2.tex]{subfiles}

\begin{document}
\setcounter{section}{11}
\section{Existenz der Ableitung}
\begin{defi}
    \(U \subset \R^n \) offen, \(\abb{f}{U}{\R^m}\) heißt
    stetig differenzierbar auf \(U\), falls \(f\) für alle
    \(x \in U\) differenzierbar ist und    
    \[ \abb{Df}{U}{\mathscr{L}(\R^n, \R^m)} \]
    eine stetige Funktion ist.
\end{defi}
\begin{bem}
    Seien \(X, Y\) normierte Räume. 
    \( \mathscr{L}(X,Y) \) ist ein Vektorraum mit
    der Operatornorm \( (A \in \mathscr{L}(X,Y)) \)
    \[ \norm{A} = \underset{\substack{x\in X\\x\neq 0}}{\sup}
    \frac{\norm{Ax}_Y}{\norm{x}_X} 
    = \underset{\substack{u\in X\\ \norm{u}_X = 1}}{\sup}
    \norm{Au}_Y
    \quad a \in \mathscr{L}(X,Y) \]
\end{bem}
\begin{bem}
    Eine Funktion \(\abb{G}{U}{\mathscr{L}(X,Y)}\) ist 
    steig in \(x_0 \in U\), falls 
    \[ \limesx{x}{x_0} \underbrace{\norm{G(x) - G(x_0)}}_{
        = \norm{(G(x) - G(x_0))[u]}_Y
    } = 0 \]
\end{bem}
Ist \(X=\R^m, Y = \R^m\), so ist \(\mathscr{L}(\R^n, \R^m)\)
wieder ein endlich dimensionaler reeller Vektorraum, also sind 
alle Normen auf \( \mathscr{L}(\R^n, \R^m) \) äquivalent.
Ist \( A \in \mathscr{L}(\R^n, \R^m) \), so können wir \(A\) 
auch als \( m \times n \)-Matrix darstellen.
\[ A = (a_{jk})_{\substack{ 1 \leq j \leq m\\1 \leq k \leq n }} \]
\begin{defi*}[Hilbert-Schmidt Norm]
    Wir definieren
    \[ \norm{A}_{HS} = \norm{A}_2 := \left( 
        \sum_{j=1}^m \sum_{k=1}^n \abs{a_{jk}}^2 \right)^{1/2} \]
    die sogenannte Hildert-Schmidt Norm.
\end{defi*}
Somit erhalten wir für \( \abb{G}{U}{\mathscr{L}(\R^n, \R^m)} \) 
für jedes \( x \in U \) eine Matrix
\[ G(x) = (g_{jk}(x))_{\substack{1 \leq j \leq m \\ 
1 \leq k \leq n}} \]
und \( \abb{G}{U}{\mathscr{L}(\R^n, \R^m)} \) ist stetig 
in \(x_0 \in U\), falls
\[ \limesx{x}{x_0} \underbrace{\norm{G(x) - G(x_0)}_{HS}}_{
    (= \sum_{j,k} (g_{jk}(x)-g_{jk}(x_0))^2)^{1/2}} = 0 \]
Eine wichtige Folgerung:\\
\( \abb{G}{U}{\mathscr{L}(\R^n, \R^m)}; x \mapsto G(x) 
= (g_{jk}(x))_{\substack{1 \leq j \leq m 
\\ 1 \leq k \leq n}} \)
ist stetig in \( x_0 \in U \)
\( \Leftrightarrow \) alle \( \abb{g_{j,k}}{U}{\R} \) 
sind stetig in \(x_0\) für alle 
\( 1 \leq j \leq m, 1 \leq k \leq n \).\\

Ist \(\abb{f}{U}{\R^m} \) in \(x_0\) differenzierbar, so lässt 
sich \(Df(x_0)\) durch die Jacobimatrix darstellen.
\[ \begin{pmatrix}
    \partial_1 f(x_0) & \cdots & \partial_n f(x_0)
\end{pmatrix} = \begin{pmatrix}
    \partial_1 f_1(x_0) & \cdots & \partial_n f_1(x_0) \\
    \vdots && \vdots \\
    \partial_1 f_m(x_0) & \cdots & \partial_n f_m(x_0)
\end{pmatrix}. \]
\( \Rightarrow \) die Ableitung \( Df(x_0) \) existiert 
\( \forall x_0 \in U \subset \R^n \), so ist 
\( \abb{Df}{U}{\mathscr{L}(\R^n, \R^m)} \) 
stetig in \( x_0 \in U \) \\
\( \Leftrightarrow \) alle partiellen Ableitungen 
\( \abb{\partial_k f_j}{U}{\R} \) sind stetig in 
\(x_0 \; \forall 1 \leq j \leq m, 1 \leq k \leq n \) \\
\( \Leftrightarrow \) alle partiellen Ableitungen 
\( \abb{\partial_k f}{U}{\R^m} \)
sind stetig in \(x_0 \; \forall 1 \leq k \leq n\).
Hauptbeobachtung:\\
Die Stetigkeit aller partiellen Ableitungen 
impliziert die Existenz von \(Df\).\\
Erinnerung: \\
Ist \(\abb{f}{U}{\R^m}\)
differenzierbar in \( x\in U \)\\
\( \Rightarrow \) alle partiellen Ableitungen 
\( \partial_j f(x), 1 \leq j \leq n \) existieren
und es gilt
\[ Df(x)[h] = \sum_{j=1}^n \partial_j f(x) h_j. \]
Hier \( \R^n = \R \underbrace{\times \ldots \times}_{n\text{-mal}} \R \)
Es wird nützlich sein, dies etwas flexibler zu 
gestalten.\\
Seien \( d_1, \ldots, d_l \in \N \) mit
\[ n = \sum_{j=1}^l d_j. \]
Schreiben 
\[ R^n = \R^{d_1} \times \ldots \times \R^{d_l} \]
\( \Rightarrow h \in \R^n \) lässt sich schreiben
als \( h = (h_1,\ldots, h_l), h_j \in \R^{d_j}, j=1,\ldots,l \).
\begin{defi}
    Sei \( \R^n = \R^{d_1} \times \dots \times \R^{d_l}, U \subset \R^n \),
    \( x = (x_1, \ldots, x_l)^t \in U \).
    Eine Funktion \( \abb{f}{U}{\R^m} \) heißt verallgemeinert
    partiell differenzierbar in \(x\) in der \(j\)-ten 
    Variable (\(j = 1, \ldots, l\)), falls die Funktion 
    \[ h_j \mapsto f(x_1,\ldots, x_{j-1}, x_j + h_j, 
    x_{j+1}, \ldots, x_l) \in \R^m \] 
    in \( h_j = 0 \) differenzierbar ist.\\
    Wir schreiben \(D_jf(x)\) für diese Ableitung.\\
    Man beachte: \( \abb{D_jf(x)}{\R^{d_j}{\R^m}} \) ist linear.
\end{defi}
\begin{bem}
    Ist \(U \subset \R^n\) offen und 
    \( x = (x_1, \ldots, x_l) 
    \in \R^{d_1} \times \ldots \times \R^{d_l} \in U \),
    so gibt es offene Mengen \( U_j \subset \R^{d_j} \) 
    mit \( x_j \in U_j, j = 1,\ldots, l \), sodass 
    der \gqq{Quader} \( \tilde{U} := U_1 \times \cdots \times U_l \subset U \) 
    ist.
\end{bem}
\begin{lem}
    Notation wie Definition 2. \( U \subset \R^n \) 
    offen, \( x \in U, x = (x_1, \ldots, x_l)^t \in U, 
    x_j \in \R^{d_j}, 1 \leq j \leq l \).
    Ist \(\abb{f}{U}{R^m}\) in \(x\) differenzierbar,
    so existieren alle (verallgemeinerten) partiellen 
    Ableitungen in \( x \) und für \( h \in \R^n, 
    h = (h_1, \ldots, h_l)^T, h_j \in \R^{d_j}, 1 \leq j \leq l \) 
    gilt 
    \[ Df(x)[h] = \sum_{j=1}^l D_j f(x)[h_j]. \]
\end{lem}
\begin{bew}
    Sei \(\R^n = \R^{d_1} \times \ldots \times \R^{d_j}\) und
    \( \abb{I_j}{\R^{d_j}}{\R^n}; h_j \mapsto (0, \ldots, 0, h_j, 0, \ldots, 0)^t \) 
    die kanonische Einbettung von \(\R^{d_j}\) in
    \[\R^n = \R^{d_1} \times \R^{d_{j-1}} \times \R^{d_j} 
    \times \R^{d_{j+1}} \times \ldots \times \R^{d_l}\]
    Die Abbildung ist stetig und linear und somit 
    differenzierbar mit Ableitung \( I_j \).\\
    Man beachte: \( \exists U_j \subset \R^{d_j} \) offen mit \(0 \in U_j\)
    und \(\abb{f_j}{U_j}{\R^m}; 
    a_j \mapsto f(x_1, \ldots, x_{j-1}, x_j + a_j, x_{j+1}, \ldots, x_l)\)
    \[ f_j(a_j) = f(x + I_j(a_j)). \]
    Nach der Kettenregel ist \( f_j \) in \( a_j = 0 \) 
    differenzierbar und somit existiert \( D_j f(x) \) 
    die gegeben ist durch 
    \[ D_j f(x) = Df(x) \circ DI_j 
    = Df(x) \circ I_j \tag{\(*\)} \]
    Da für alle \(h \in \R^n = \R^{d_1} \times \ldots \times \R^{d_j},
    h = \sum_{j=1}^l I_j(h_j) \) folgt
    \[ Df(x)[h] = Df(x)[\sum_{j=1}^l I_j(h_j)]
    = \sum_{j=1}^l 
    \underbrace{Df(x)I_j(h_j)]}_{= Df(x) \circ I_j(h_j) = D_jf(x)[h_j]} \]
\end{bew}
\begin{satz}
    Schreiben \( \R^n = \R^{d_1} \times \ldots \times \R^{d_j} \),
    \( d_j \in \N , 1 \leq j \leq l \) mit \( \sum_{j=1}^l d_j = n \).\\
    Sei \( U_j \in \R^{d_j} \) offen und 
    \( U = U_1 \times \ldots \times U_l \subset \R^n \) (auch offen).\\
    Eine Funktion \( \abb{f}{U}{\R^m} \) ist stetig differenzierbar auf 
    \( U \Leftrightarrow \) alle verallgemeinerten partiellen Ableitungen 
    \( D_jf \) sind stetig auf U.\\
    In diesem Fall gilt die Formel aus Lemma 3.
\end{satz}
\begin{bew}
    \gqq{\( \Rightarrow \)} Lemma 3.\\
    \gqq{\( \Leftrightarrow \)}: Es reicht den Fall, \( \R^n, \R^{d_1} \times \R^{d_2} \) 
    zu betrachten. Der Allgemeine Fall folgt per Induktion. 
    Also sei \( x = \begin{pmatrix} x_1 \\ x_2 \end{pmatrix} \in \R^{d_1} \times \R^{d_2} \), 
    \( U_j \subset \R^{d_j}, j = 1,2 \) offen, \( x \in U_1 \times U_2 \). 
    Bild: 
    % Bild
    Beachte \(h = \begin{pmatrix}
        h_1 \\
        h_2
    \end{pmatrix} \in \R^n = \R^{d_1} \times \R^{d_2} \)
    \[ \Rightarrow h \rightarrow 0 \Leftrightarrow 
    \underbrace{h_1 \rightarrow 0}_{\in \R^{d_1}}
    \wedge \underbrace{h_2 \rightarrow 0}_{\in \R^{d_2}} \]
    \[ \abs{h} = (\abs{h_1}^2 + \abs{h_2}^2)^{1/2} \]
    Wegen Lemma 3 vermuten wir 
    \[ Df(x)[h] = D_1f(x)[h_1] + D_2f(x)[h_2]. \]
    Also sei \( x + h = 
    \begin{pmatrix} x_1 + h_1 \\ x_2 + h_2 \end{pmatrix} 
    \in \R^{d_1} \times \R^{d_2} \).\\
    Wir schreiben 
    \begin{align*}
        &f(x + h) - f(x) - D_1f(x)[h_1] - D_2 f(x)[h_2] \\
        &= f(x_1 + h_1, x_2 + h_2) - f(x_1 + h_1, x_2) 
        - D_2 f(x_1 + h, x_2)[h_2] \tag{\Romannum{1}}\\ 
        &+ D_2 f(x_1 + h, x_2)[h_2] - D_2 f(x_1, x_2)[h_2] \tag{\Romannum{2}} \\
        &+ f(x_1 + h_1, x_2) - f(x_1, x_2) - D_1 f(x_1, x_2)[h_1] \tag{\Romannum{3}}.
    \end{align*}
    Da \(f\) in \(x \in (x_1, x_2)^t\) in \(x_1\) partiell 
    differenzierbar ist, folgt 
    \[ \limesx{h_1}{0} \frac{\Romannum{3}}{\abs{h_1}} = 0 \Rightarrow
    \limesx{h}{0} \frac{\Romannum{3}}{\abs{h}} = 0 \]
    da \( \abs{h} = (\abs{h_1}^2 + \abs{h_2}^2)^{1/2} \geq \abs{h_1} \)
    \begin{align*}
        \abs{II} &= \abs{ D_2 f(x_1 + h_1, x_2)[h_2] - D_2 f(x_1, x_2)[h_2] } \\
        &= \abs{ (D_2 f(x_1 + h_1, x_2) - D_2 f(x_1, x_2))[h_2] }\\
        &\leq \norm{ D_2 f(x_1 + h_1, x_2) - D_2 f(x_1, x_2) } \cdot \underbrace{\abs{h_2}}_{\leq \abs{h}}.
    \end{align*}
    Da \( D_2f(x) \) stetig ist in \(x\), folgt 
    \[ \frac{\abs{II}}{\abs{h}} \leq \norm{ D_2 f(x_1 + h_1, x_2) - D_2 f(x_1, x_2) } 
    \overset{h\rightarrow \infty}{\longrightarrow} 0. \]
    Es bleibt also 
    \[ I = f(x_1 + h_1, x_2 + h_2) - f(x_1 + h_1, x_2) - D_2 f(x_1 + h_1, x_2)[h_2] \]
    zu betrachten.
    Aus Korollar 11.3 folgt
    \[ \abs{\Romannum{1}} = \underbrace{\abs{h_2}}_{\abs{h}} \underset{
    \substack{
        \rho \in \R^{d_1}\\
        \abs{\rho} \leq \abs{h_1}
    }}{\sup} \norm{D_2f(x_1+\rho, x_2) - D_2f(x_1, x_2)} \]
    \[ \Rightarrow \frac{\Romannum{1}}{\abs{h}} \leq \underset{
        \substack{
            \rho \in \R^{d_1}\\
            \abs{\rho} \leq \abs{h_1}
        }}{\sup} \norm{D_2f(x_1+\rho, x_2) - D_2f(x_1, x_2)}
        \overset{h \rightarrow 0}{\rightarrow} 0 
        (\text{Da } D_2f \text{ stetig ist}) \]    
    Insgesamt folgt also 
    \[ \limesx{h}{0} \frac{1}{\abs{h}} (f(x + h) - f(x) 
    - D_1f(x_1, x_2)[h_1] 
    - D_2 f(x_1, x_2)[h_2] ) = 0. \]
    Somit ist \(f\) in \(x = \begin{pmatrix}x_1\\x_2\end{pmatrix} \) 
    diffbar und es gilt 
    \[ Df(x)[h] = D_1 f(x)[h_1] + D_2f(x)[h_2]. \]    
    Nun zur Induktion:
    Sei \( \R^n = \R^{d_1} \times \R^{d_2} 
    \times \R^{d_3}, U_j \subset \R^{d_j} \) offen, 
    \( U = U_1 \times U_2 \times U_3, \abb{f}{U}{\R^m} \). 
    Angenommen alle partiellen Ableitungen 
    \( \abb{D_j f(x)}{\R^{d_j}}{\R^m} \) 
    sind stetig (in \(x\) auf \(U\)). \\
    Schreiben \( x = \begin{pmatrix} \tilde{x} x_3 \end{pmatrix} 
    \in (\R^{d_1} \times \R^{d_2}) \times \R^{d_3} 
    = \R^{d_1 + d_2} \times \R^{d_3} \).        
    Ausdem bisher gezeigten folgt: Die partielle Ableitung von
    \(f\) bzgl. \(\tilde{x} \in U_1 \times U_2\) existiert und 
    ist stetig in \(x\).\\
    D. h. bzgl. der Zerlegung \( \R^n = \R^{d_1 + d_2} \times \R^{d_3} \) 
    existiert \[ \abb{\tilde{D_1} f(x)}{\R^{d_1 + d_2}}{\R^m} \]
    \( D_2f(x) \) ist weiterhin stetig.\\
    \( \Rightarrow \) aus dem bisher gezeigten folgt, dass 
    die Ableitung \( \abb{Df(x)}{(\R^{d_1 + d_2}) \times \R^{d_3}}{\R^m} \) 
    existiert für alle \( x\in \tilde{U} \times U_3 = U \) und stetig 
    in \( x \in U \).
\end{bew}
\end{document}