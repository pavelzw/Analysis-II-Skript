\documentclass[../ana2.tex]{subfiles}

\begin{document}
\setcounter{section}{11}
\section{Existenz der Ableitung}
\begin{defi*}
    \(U \subset \R^n \) offen, \(\abb{f}{U}{\R^m}\) heißt
    stetig differenzierbar auf \(U\), falls \(f\) für alle
    \(x \in U\) differenzierbar ist und    
    \[ \abb{Df}{U}{\mathscr{L}(\R^n, \R^m)} \]
    eine stetige Funktion ist.
\end{defi*}
\begin{bem}
    Seien \(X, Y\) normierte Räume. 
    \( \mathscr{L}(X,Y) \) ist ein Vektorraum mit
    der Operatornorm \( (A \in \mathscr{L}(X,Y)) \)
    \[ \norm{A} = \underset{\substack{x\in X\\x\neq 0}}{\sup}
    \frac{\norm{Ax}_Y}{\norm{x}_X} 
    = \underset{\substack{u\in X\\ \norm{u}_X = 1}}{\sup}
    \norm{Au}_Y
    \quad a \in \mathscr{L}(X,Y) \]
\end{bem}
\begin{bem}
    Eine Funktion \(\abb{G}{U}{\mathscr{L}(X,Y)}\) ist 
    stetig in \(x_0 \in U\), falls 
    \[ \limesx{x}{x_0} \underbrace{\norm{G(x) - G(x_0)}}_{
        = \norm{(G(x) - G(x_0))[u]}_Y
    } = 0 \]
\end{bem}
Ist \(X=\R^m, Y = \R^m\), so ist \(\mathscr{L}(\R^n, \R^m)\)
wieder ein endlich dimensionaler reeller Vektorraum, also sind 
alle Normen auf \( \mathscr{L}(\R^n, \R^m) \) äquivalent.
Ist \( A \in \mathscr{L}(\R^n, \R^m) \), so können wir \(A\) 
auch als \( m \times n \)-Matrix darstellen.
\[ A = (a_{jk})_{\substack{ 1 \leq j \leq m\\1 \leq k \leq n }} \]
\begin{defi*}[Hilbert-Schmidt Norm]
    Wir definieren
    \[ \norm{A}_{HS} = \norm{A}_2 := \left( 
        \sum_{j=1}^m \sum_{k=1}^n \abs{a_{jk}}^2 \right)^{1/2} \]
    die sogenannte Hildert-Schmidt Norm.
\end{defi*}
Somit erhalten wir für \( \abb{G}{U}{\mathscr{L}(\R^n, \R^m)} \) 
für jedes \( x \in U \) eine Matrix
\[ G(x) = (g_{jk}(x))_{\substack{1 \leq j \leq m \\ 
1 \leq k \leq n}} \]
und \( \abb{G}{U}{\mathscr{L}(\R^n, \R^m)} \) ist stetig 
in \(x_0 \in U\), falls
\[ \limesx{x}{x_0} \underbrace{\norm{G(x) - G(x_0)}_{HS}}_{
    (= \sum_{j,k} (g_{jk}(x)-g_{jk}(x_0))^2)^{1/2}} = 0 \]
Eine wichtige Folgerung:\\
\( \abb{G}{U}{\mathscr{L}(\R^n, \R^m)}; x \mapsto G(x) 
= (g_{jk}(x))_{\substack{1 \leq j \leq m 
\\ 1 \leq k \leq n}} \)
ist stetig in \( x_0 \in U \)
\( \Leftrightarrow \) alle \( \abb{g_{j,k}}{U}{\R} \) 
sind stetig in \(x_0\) für alle 
\( 1 \leq j \leq m, 1 \leq k \leq n \).\\

Ist \(\abb{f}{U}{\R^m} \) in \(x_0\) differenzierbar, so lässt 
sich \(Df(x_0)\) durch die Jacobimatrix darstellen.
\[ \begin{pmatrix}
    \partial_1 f(x_0) & \cdots & \partial_n f(x_0)
\end{pmatrix} = \begin{pmatrix}
    \partial_1 f_1(x_0) & \cdots & \partial_n f_1(x_0) \\
    \vdots && \vdots \\
    \partial_1 f_m(x_0) & \cdots & \partial_n f_m(x_0)
\end{pmatrix}. \]
\( \Rightarrow \) die Ableitung \( Df(x_0) \) existiert 
\( \forall x_0 \in U \subset \R^n \), so ist 
\( \abb{Df}{U}{\mathscr{L}(\R^n, \R^m)} \) 
stetig in \( x_0 \in U \) \\
\( \Leftrightarrow \) alle partiellen Ableitungen 
\( \abb{\partial_k f_j}{U}{\R} \) sind stetig in 
\(x_0 \; \forall 1 \leq j \leq m, 1 \leq k \leq n \) \\
\( \Leftrightarrow \) alle partiellen Ableitungen 
\( \abb{\partial_k f}{U}{\R^m} \)
sind stetig in \(x_0 \; \forall 1 \leq k \leq n\).
Hauptbeobachtung:\\
Die Stetigkeit aller partiellen Ableitungen 
impliziert die Existenz von \(Df\).\\
Erinnerung: \\
Ist \(\abb{f}{U}{\R^m}\)
differenzierbar in \( x\in U \)\\
\( \Rightarrow \) alle partiellen Ableitungen 
\( \partial_j f(x), 1 \leq j \leq n \) existieren
und es gilt
\[ Df(x)[h] = \sum_{j=1}^n \partial_j f(x) h_j. \]
Hier \( \R^n = \R \underbrace{\times \ldots \times}_{n\text{-mal}} \R \)
Es wird nützlich sein, dies etwas flexibler zu 
gestalten.\\
Seien \( d_1, \ldots, d_l \in \N \) mit
\[ n = \sum_{j=1}^l d_j. \]
Schreiben 
\[ R^n = \R^{d_1} \times \ldots \times \R^{d_l} \]
\( \Rightarrow h \in \R^n \) lässt sich schreiben
als \( h = (h_1,\ldots, h_l), h_j \in \R^{d_j}, j=1,\ldots,l \).
\begin{defi*}
    Sei \( \R^n = \R^{d_1} \times \dots \times \R^{d_l}, U \subset \R^n \),
    \( x = (x_1, \ldots, x_l)^t \in U \).
    Eine Funktion \( \abb{f}{U}{\R^m} \) heißt verallgemeinert
    partiell differenzierbar in \(x\) in der \(j\)-ten 
    Variable (\(j = 1, \ldots, l\)), falls die Funktion 
    \[ h_j \mapsto f(x_1,\ldots, x_{j-1}, x_j + h_j, 
    x_{j+1}, \ldots, x_l) \in \R^m \] 
    in \( h_j = 0 \) differenzierbar ist.\\
    Wir schreiben \(D_jf(x)\) für diese Ableitung.\\
    Man beachte: \( \abb{D_jf(x)}{\R^{d_j}{\R^m}} \) ist linear.
\end{defi*}
\begin{bem}
    Ist \(U \subset \R^n\) offen und 
    \( x = (x_1, \ldots, x_l) 
    \in \R^{d_1} \times \ldots \times \R^{d_l} \in U \),
    so gibt es offene Mengen \( U_j \subset \R^{d_j} \) 
    mit \( x_j \in U_j, j = 1,\ldots, l \), sodass 
    der \gqq{Quader} \( \tilde{U} := U_1 \times \cdots \times U_l \subset U \) 
    ist.
\end{bem}
\begin{lem}
    Notation wie Definition 2. \( U \subset \R^n \) 
    offen, \( x \in U, x = (x_1, \ldots, x_l)^t \in U, 
    x_j \in \R^{d_j}, 1 \leq j \leq l \).
    Ist \(\abb{f}{U}{R^m}\) in \(x\) differenzierbar,
    so existieren alle (verallgemeinerten) partiellen 
    Ableitungen in \( x \) und für \( h \in \R^n, 
    h = (h_1, \ldots, h_l)^T, h_j \in \R^{d_j}, 1 \leq j \leq l \) 
    gilt 
    \[ Df(x)[h] = \sum_{j=1}^l D_j f(x)[h_j]. \]
\end{lem}
\begin{bew}
    Sei \(\R^n = \R^{d_1} \times \ldots \times \R^{d_j}\) und
    \( \abb{I_j}{\R^{d_j}}{\R^n}; h_j \mapsto (0, \ldots, 0, h_j, 0, \ldots, 0)^t \) 
    die kanonische Einbettung von \(\R^{d_j}\) in
    \[\R^n = \R^{d_1} \times \R^{d_{j-1}} \times \R^{d_j} 
    \times \R^{d_{j+1}} \times \ldots \times \R^{d_l}\]
    Die Abbildung ist stetig und linear und somit 
    differenzierbar mit Ableitung \( I_j \).\\
    Man beachte: \( \exists U_j \subset \R^{d_j} \) offen mit \(0 \in U_j\)
    und \(\abb{f_j}{U_j}{\R^m}; 
    a_j \mapsto f(x_1, \ldots, x_{j-1}, x_j + a_j, x_{j+1}, \ldots, x_l)\)
    \[ f_j(a_j) = f(x + I_j(a_j)). \]
    Nach der Kettenregel ist \( f_j \) in \( a_j = 0 \) 
    differenzierbar und somit existiert \( D_j f(x) \) 
    die gegeben ist durch 
    \[ D_j f(x) = Df(x) \circ DI_j 
    = Df(x) \circ I_j \tag{\(*\)} \]
    Da für alle \(h \in \R^n = \R^{d_1} \times \ldots \times \R^{d_j},
    h = \sum_{j=1}^l I_j(h_j) \) folgt
    \[ Df(x)[h] = Df(x)[\sum_{j=1}^l I_j(h_j)]
    = \sum_{j=1}^l 
    \underbrace{Df(x)I_j(h_j)]}_{= Df(x) \circ I_j(h_j) = D_jf(x)[h_j]} \]
\end{bew}
\begin{satz}
    Schreiben \( \R^n = \R^{d_1} \times \ldots \times \R^{d_j} \),
    \( d_j \in \N , 1 \leq j \leq l \) mit \( \sum_{j=1}^l d_j = n \).\\
    Sei \( U_j \in \R^{d_j} \) offen und 
    \( U = U_1 \times \ldots \times U_l \subset \R^n \) (auch offen).\\
    Eine Funktion \( \abb{f}{U}{\R^m} \) ist stetig differenzierbar auf 
    \( U \Leftrightarrow \) alle verallgemeinerten partiellen Ableitungen 
    \( D_jf \) sind stetig auf U.\\
    In diesem Fall gilt die Formel aus Lemma 3.
\end{satz}
In der Vorlesungs wurde die folgende Kette von Inmplikationen bewiesen:
\begin{align*}
    &\text{stetig partiell differenzierbar} \\
    &\Rightarrow \text{ Differenzierbarkeit} \\
    &\Rightarrow \text{ partiell differenzierbar}
\end{align*}
Die Umkehrung gilt im Allgemeinen nicht.
\begin{lem}
    Sei \(U \subset \R^n \) offen und wegzusammenhängend. 
    Dann gibt es zu \(x_0, x_1 \in U\) einen Polynomzug, d. h. 
    ein stückweise linearen Weg \(\abb{\gamma}{[0,1]}{U}\) mit
    \(\gamma(0) = x_0, \gamma(1) = x_1\).
\end{lem}
\begin{bew}
    Sei \(\abb{\phi}{[0,1]}{U}\) stetig mit \(\phi(0) = x_0\)
    und \(\phi(1) = x_1\). Da \(\phi\) stetig und \([0,1]\)
    kompakt ist, ist \( K:= \set{\phi(t): 0 \leq t \leq 1} \subset U \subset \R^n \)
    eine kompakte Teilmenge von \(\R^n\). Ausserdem ist 
    \(A := U^C = \R^n \setminus U\) abgeschlossen und somit 
    \(r := \dist(K, A) = 
    \underset{\substack{x \in U \\ y \in A}}{\inf} 
    \abs{x-y} > 0\). \\
    Denn angenommen \( r = 0 \), so existieren 
    Folgen \( (x_n)_n \subset K, (y_n)_n \subset A \)
    mit \( \abs{x_n - y_n} \rightarrow 0 \).
    Da \(K\) kompakt ist, existiert eine konvergente Teilfolge
    \( (x_{n_k})_k, \tilde{x} := \limes{k} x_{n_k} \in K \subset U \)    
    \( \Rightarrow \abs{ y_{n_K} - \tilde{x} }
    \leq \abs{ y_{n_k} - x_{n_k} } 
    + \abs{ x_{n_k} - \tilde{x} } \rightarrow 0 \).
    Das heißt \( y_{n_k} \rightarrow \tilde{x} \) 
    für \(k \rightarrow \infty \). Da 
    \( y_{n_k} \in A = U^C \) abgeschlossen
    \( \Rightarrow \tilde{x} \in A \text{ \Lightning} \) 
    zu \(\tilde{x} \in K \subset U\). 
    Also ist \(r > 0\).\\
    Sei \( N\in\N \). Betrachte Zerlegung
    \( t_j = \frac{j}{N} \) von \( [0,1] \).
    \( j = 0,1,\ldots,N \) und definieren
    \[ \gamma_N(t) = \frac{t_k -t}{t_k-t_{k-1}} \varphi(t_{k-1}) 
    + \frac{t-t_{k-1}}{t_k - t_{k-1}} \varphi(t_k) \]
    für \( t \in [t_{k-1}, t_k] \).\\
    \(\Rightarrow\) Polygonzug \(\abb{\gamma_N}{[0,1]}{\R^n}\)
    \begin{beh}
        \( \gamma_N([0,1]) \subset U \), 
        wenn \(N\) groß genug ist.
    \end{beh}
    \begin{bew}\renewcommand{\qedsymbol}{}
        \(\abb{\varphi}{[0,1]}{U}\) ist stetig, 
        \( [0,1] \) ist kompakt \( \Rightarrow 
        \varphi \) ist gleichmäßig stetig auf 
        \( [0,1] \).
        Setze \(\mathrm{osz}(\varphi, \delta) := \sup 
        \set{\abs{\varphi(t)- \varphi(\delta)} \vert s,t \in [0,1], \abs{t-s} \leq \delta}\)
        die \(\delta\)-Oszillation von \( \varphi \).\\
        \( \varphi \) gleichmäßig stetig 
        \( \Leftrightarrow \limesx{\delta}{0+} 
        \mathrm{osz}(\varphi, \delta) = 0 \).
        Insbesondere \(\exists \delta > 0\) mit
        \[ \mathrm{osz}(\varphi, \delta) \leq \frac{r}{2} \]
        Wähle \(N \in \N\) so, dass \(\frac{1}{N} < \delta\).\\
        \( \Rightarrow \) Auf dem Intervall 
        \( [t_{k-1}, t_k] \) ist
        \begin{align*}
            \gamma_N(t) - \varphi(t) 
            &= \frac{t_{k} - t}{t_k - t_{k-1}}\varphi(t_{k-1})
            + \frac{ t - t_k }{ t_k - t_{k-1} }\varphi(t_k) 
            - \varphi(t) \\
            &= \frac{ t_k - t }{t_k - t_{k-1}} (\varphi(t_{k-1}) - \varphi(t)) 
            + \frac{ t - t_{k-1} }{t_k - t_{k-1}}(\varphi(t_k) - \varphi(t_{k-1}))
        \end{align*}
        
        \[ \Rightarrow \abs{\gamma_N(t) - \varphi(t)} 
        \leq \mathrm{osz}(\varphi, \frac{1}{N})
        \leq \mathrm{osz}(\varphi, \delta) < \frac{r}{2} 
        \; \forall 0 \leq t \leq 1 \]
        \[ \Rightarrow \underset{t \in [0,1]}{\sup} 
        \abs{\gamma_N(t)-\varphi(t)} \leq \frac{r}{2} \]
        d. h. \( \gamma_N([0,1]) \) ist in einem \(\frac{r}{2}\)-Schlauch um
        \(\varphi([0,1]) = K\) und da \(\dist(K,A) = r > 0\) folgt
        \[ \gamma_N([0,1]) \subset U. \]        
    \end{bew}
\end{bew}
\begin{satz}
    Sei \(U \subset \R^n\) offen und wegzusammenhängend,
    \( \abb{f}{U}{\R^m} \) differenzierbar. Dann gilt
    \[ Df(x) = 0 \; \forall x \in U \Rightarrow f \text{ ist konstant} \]
\end{satz}
\begin{satz}
    Seien \(x_0, x_1 \in U\). Nach Lemma 3 existiert ein Polygonzug
    \( \abb{\gamma}{[0,1]}{U} \) mit \( \gamma(0) = x_0 \), 
    \( \gamma(1) = x_1 \).
    Da \( \gamma \) stückweise \(\mathcal{C}^1\) ist und \(f\) stetig
    differenzierbar ist, folgt aus Lemma 1 
    \[ f(x_1)-f(x_0) = \integralx{\underbrace{Df(\gamma(t)[\gamma(t)]}_{=\; 0}}{a}{b}{t} 
    = 0 \in \R^m. \]
    \( \Rightarrow f(x_1) = f(x_0) \). Da \( x_0, x_1 \) beliegbig sind, 
    folgt die Behauptung.
\end{satz}
\subsection{Zweite Ableitung}
%\(U \subset \R^n\) offen, \(\abb{f}{U}{\R^m}\) differenzierbar.
%\[ \Rightarrow \abb{Df}{U}{
%    \underbrace{\mathscr{L}(\R^n, \R^m)}_{\text{ist wieder VR } W}} \]
%\( W \) ist ein normierter Vektorraum und sei 
%\( \abb{f}{U}{W} \) ist differenzierbar
%\[ \Rightarrow \abb{Df}{U}{\underbrace{\mathscr{L}(\R^n, W)}_{= \mathscr{L}(\R^n, 
%\mathscr{L}(\R^n, \R^m))} =: \tilde{W}}. \]
%Was ist das? \\
Sei \(U \subset \R^n\) offen, \(W\) beliegbiger Vektorraum, \(\abb{f}{U}{W}\)
differenzierbar. \\
\[ \Rightarrow \abb{Df}{U}{\mathscr{L}(\R^n, W)} =: W \text{ wieder ein VR} \]
Frage: Ist \(\abb{f}{U}{\tilde{W}}\) wieder differenzierbar und falls ja,
ist die zweite Ableitung von \(f\) dann
\[ \abb{D^2f = D(Df)}{U}{\mathscr{L}(\R^n, \tilde{W})}  \]
Was dieses Objekt ist, wird in einem Handout diskutiert.\\
Im Folgenden der einfachere Fall von Richtungsableitungen.
Ist \( \abb{f}{U}{\R^m} \) und \( h\in \R^n \) (\(U \subset \R^n\) offen)
\[ \Rightarrow D_h f(x) = \limesx{t}{0} 
\frac{ f(x + th) - f(x) }{t} = Df(x)[h] \]
falls \(f\) differenzierbar ist.
Satz 12.4 sagt: \\
\(f\) ist stetig differenzierbar. \\
\(\Leftrightarrow\) alle partiellen Ableitungen 
\(\abb{\partial_j f}{U}{\R^m}\) sind stetig auf \(U\). \\
\(\Leftrightarrow \forall n \in \R^n\) sind 
alle Richtungsableitungen stetig auf \(U\).
Also \( \abb{f}{U}{\R^m}, h_1, h_2 \in \R^n \) und 
Richtungsableitungen \( \abb{D_{h_1}f}{U}{\R^m} \) 
und \( \abb{D_{h_2}f}{U}{\R^m} \) existieren.
Dann können wir die Richtungsableitungen von \(D_{h_1}f\) in Richtung
\(h_2\) und von \(D_{h_2}f\) in Richtung \(h_1\) bilden.
Also angenommen \( D_{h_1}(D_{h_2}f)(x) \) und \( D_{h_2}(D_{h_1}f(x) \)
existieren. Was haben sie dann miteinander zu tun? \\
Z. B. für partielle Ableitungen 
\begin{align*}
    \partial_j \partial_k f(x) &= D_{e_j}D_{e_k} f(x) \\
    \partial_k \partial_j f(x) &= D_{e_k} D_{e_j} f(x)
\end{align*}
\begin{satz}[Einfache Version des Satzes von Schwarz]
    Sei \( U \subset \R^n \) offen, \( \abb{f}{U}{\R^m}, 
    h_1, h_2 \in \R^n \). Ferner existieren die 
    Richtungsableitungen \( \abb{D_{h_1}f, D_{h_2}f}{U}{\R^m} \) 
    und die iterierten Richtungsableitungen 
    \( \abb{D_{h_1}D_{h_2}f}{U}{\R^m} \) und sie seinen stetig
    auf \(U\).
    \[ \Rightarrow D_{h_1}D_{h_2} f(x) = D_{h_2}D_{h_1} f(x) 
    \; \forall x \in U \]
    Das heißt die iterierten Richtungsableitungen sind symmetrisch.
\end{satz}
\begin{bew}
    Sei \(x \in U\) und \(\delta > 0\), sodass 
    \[ x + t_1 h_1 + t_2h_2 \; \forall t_1, t_2 \in [0, \delta] \]    
    Sei \( 0 \leq t \leq \delta \) und 
    \[ \varphi(s) := f(x + s h_1 + t h_2) - f(x + s h_1). \]
    \begin{align*}
        \Rightarrow \ddxpartial{s} \varphi(s) 
        &= D_{h_1} f( x + sh_1 )- D_{h_1}f(x + s h_1) \\
        &\oversett{HDI}{=} \integralx{\ddxpartial{u} 
        D_{h_1}f(x+s h_1 + u h_2) }{0}{t}{u} \\
        &= \integralx{ D_{h_2} D_{h_1}f(x+s h_1 + u h_2) }{0}{t}{u} 
    \end{align*}
    \begin{align*}
        \Rightarrow \varphi(t)-\varphi(0) &= f(x+th_1+th_2)-f(x+th_1)
        -f(x+th_2) + f(x) \\
        &= \integralx{ \ddxpartial{s} \varphi(s)}{0}{t}{s}\\
        &= \integralx{
        \integralx{D_{h_1} D_{h_1} f(x + s h_1 + u h_2)}{0}{t}{u}
        }{0}{t}{s}\\
        &= \integralx{
            \integralx{D_{h_1} D_{h_1} f(x + s h_1 + u h_2) - D_{h_2} D_{h_1}f(x)}{0}{t}{u}
        }{0}{t}{s} + D_{h_2} D_{h_1} f(x) t^2.
    \end{align*}
    Durch Vertauschen von \(h_1\) und \(h_2\) erhalten wir 
    (da \(\varphi(t)-\varphi(0)\) symmetrisch in \(h_1\) und \(h_2\) ist)
    \[ \varphi(t)-\varphi(0) = \integralx{
        \integralx{D_{h_1} D_{h_2} f(x + s h_2 + u h_1) - D_{h_1} D_{h_2}f(x)}{0}{t}{u}
    }{0}{t}{s} + D_{h_1} D_{h_2} f(x) t^2 \]
    Da \( U \in z \mapsto D_{h_2} D_{h_1} f(z) \) stetig ist, ist 
    \[ [0,\delta]\times [0,\delta] = [0,\delta]^2 \ni (t_1, t_2) 
    \mapsto D_{h_2} D_{h_1} f(x + t_1 h_1 + t_2 h_2) \] 
    stetig, sagen gleichmäßig stetig, da \( [0,\delta]^2 \) kompakt ist.
    \( \Rightarrow \forall \varepsilon > 0 \; \exists 0 < \eta_1 < \delta \) mit
    \[ \abs{D_{h_2}D_{h_1} f(x+sh_1 + uh_2) - D_{h_2}D_{h_1}f(x)} \leq \varepsilon
    \; \forall 0 < U, s < \eta_1 \]    
    \[ \Rightarrow \forall 0 < t \leq \eta_1 \text{ gilt} \]
    \[ A := \integralx{
        \integralx{\abs{ D_{h_2} D_{h_1} f(x + s h_1 + u h_2) 
        - D_{h_2} D_{h_1} f(x) }}{0}{t}{u}
    }{0}{t}{s} \leq \frac{\varepsilon}{2}t^2 \]
    Genauso \(\exists 0 < \eta_2 < \delta \) mit \(\forall 0 < t \leq \eta_2\)
    \[ B := \integralx{
        \integralx{\abs{ D_{h_1} D_{h_2} f(x + s h_1 + u h_2) 
        - D_{h_1} D_{h_2} f(x) }}{0}{t}{u}
    }{0}{t}{s} \leq \frac{\varepsilon}{2}t^2 \]
    \[ t^2(D_{h_2}D_{h_1}f(x)-D_{h_1}D_{h_2}f(x)) = B - A \]
    \[ \Rightarrow t^2 \abs{ D_{h_2} D_{h_1} f(x) - D_{h_1} D_{h_2} f(x) } \leq \varepsilon t^2 \]
    \[ \Rightarrow \abs{ D_{h_2} D_{h_1} f(x) - D_{h_1} D_{h_2} f(x) } \leq \varepsilon \]
    \(\Rightarrow\) da \(\varepsilon > 0\) beliebig ist, folgt 
    \[ D_{h_2}D_{h_1}f(x) = D_{h_1}D_{h_2}f(x). \]
\end{bew}
\end{document}