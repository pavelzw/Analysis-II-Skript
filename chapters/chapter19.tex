\documentclass[../ana2.tex]{subfiles}

\begin{document}
\setcounter{section}{18}
\section{Kuurvenintegrale}
\begin{defi}[\(\mathcal{C}^1\)-Kurve]
    Ist \( I \subset \R \) Intervall, \( \gamma \in \mathcal{C}^1(I, \R^d) \)
    so heißt \(\gamma \in \mathcal{C}^1\)-Kurve.
\end{defi}
\begin{bspe}\leavevmode
    \begin{enumerate}
    \item \( p, v \in \R^d, v \neq 0, \quad \gamma(t) = p + tv, t \in \R \).
    \item \(\abb{\gamma}{\R}{\R^2} \gamma(t) = (r(\cos t, \sin t) \)
    \item Schraubenlinie \( \abb{\gamma}{\R}{\R^3} t \mapsto \gamma(t) = (r \cos t, r \sin t, at) \).
    \item Spirale \( \abb{\gamma}{\R}{\R^3}, t \mapsto \gamma(t) = e^{-t} (\cos t, \sin t) \)
    \item \( \abb{\gamma}{I}{R^2}; t \mapsto \gamma(t) = (\cos t, \sin (2t))\)
    \end{enumerate}
\end{bspe}
\begin{defi}[Bogenlänge]
    Sei \(\abb{\gamma}{[a,b]}{\R^d}\ \mathcal{C}^1\)-Kurve
    \[ L(\gamma) := \integralx{\abs{\gamma'(t)}}{a}{b}{t} \]
    Bogenlänge (Länge der Kurve).\\
    Zerlegung \(a = t_0 < t_1 < \ldots < t_N = b\)
    \[ \sum_{j=1}^N \abs{\gamma(t_j) - \gamma(t_{j-1})} 
    = \sum_{j=1}^N \abs{\gamma'(t_j)} (t_j - t_{j-1}) 
    \approx \integralx{\abs{\gamma'(t)}}{a}{b}{t} \]
\end{defi}
\begin{bsp}
    \( \gamma(t) = r (\cos t, \sin t), \quad 0 \leq t \leq 2\pi \).
    \[ L(\gamma) = \integralx{\abs{\gamma'(t)}}{0}{2\pi}{t}
    = \integralx{\abs{r \begin{pmatrix}
        -\sin t \\ \cos t
    \end{pmatrix}}}{0}{2\pi}{t} 
    = r \integralx{}{0}{2\pi}{t} = 2\pi r. \]
\end{bsp}
\begin{bsp}
    \(\gamma(t) = (r\cos t, r\sin t, at), 0 \leq t \leq 2 \pi \)
    \[ \gamma'(t) = (-r\sin t, r\cos t, a) \]
    \[ L(\gamma) = \integralx{\abs{\gamma'(t)}}{0}{2\pi}{t} 
    = \integralx{\abs{ \begin{pmatrix}
        -r \sin t \\ r \cos t \\ a
    \end{pmatrix} }}{0}{2\pi}{t} 
    = \integralx{(r^2 + a^2)^{1/2}}{0}{2\pi}{t} 
    = 2\pi \sqrt{r^2 + a^2}. \]
\end{bsp}
\begin{defi}[Umparametrisierung]
    Seien \(\abb{\gamma_j}{I_j}{\R^d}, j=1,2 \) zwei \(\mathcal{C}^1\)-Kurven
    Dann heißt \(\gamma_2\) Umparametrisierung von \( \gamma_1 \),
    falls es eine Bijektion \(\varphi \in \mathcal{C}^1(I_2, I_1)\)
    mit \(\varphi' \neq 0 \) existiert mit 
    \[ \gamma_2 = \gamma_1 \circ \varphi. \]
    Die Bijektion \( \varphi \) heißt Parametertransformation.
\end{defi}
\begin{lem}
    Sind \(\gamma_j \in \mathcal{C}^1(I_j, \R^d), j = 1,\ldots, \gamma_2\) Umparametrisierung  
    von \( \gamma \), so ist 
    \[ L(\gamma_1) = L(\gamma_2). \]
\end{lem}
\begin{bew}
    Haben \( \gamma_2 = \gamma_1 \circ \varphi \), \( \varphi \in \mathcal{C}^1(I_2, I_1), 
    \varphi' \neq 0 \).
    \[ \gamma_2'(t) = (\gamma_1 \circ \varphi)' 
    = \gamma_1'(\varphi(t)) \cdot \varphi'(t). \]
    \[ \Rightarrow \abs{\gamma_2'(t)} 
    = \abs{\gamma'_1(\varphi(t)) \cdot \varphi'(t)} 
    = \pm \abs{\gamma'_1(\varphi(t))} \cdot \varphi'(t) \\
    +: \varphi'> 0;\forall t \in I_2 \\
    -: \varphi'< 0;\forall t \in I_2 \]
    \begin{align*}
        \Rightarrow L(\gamma_2) = \integralx{\abs{ \gamma_2'(t) }}{a_2}{b_2}{t}
        &= \pm \integralx{\abs{\gamma_2'(\varphi(t))} \varphi'(t) }{a_2}{b_2}{t} \\
        &= \pm \integralx{\abs{\gamma_1'(s)}}{\varphi(a_2)}{\varphi(b_2)}{s}
    \end{align*}
    \[ = \begin{cases}
        \integralx{\abs{ \gamma_1'(s) }}{a_1}{b_1}{s}, & \varphi' > 0 \\
        - \integralx{\abs{\gamma_1'(s)}}{b_1}{a_1}{s} 
        = \integralx{\abs{\gamma_1'(s)}}{a_1}{b_1}{s}, & \varphi < 0 
    \end{cases}\]
\end{bew}
\begin{defi}[Parametrisierung nach Bogenlänge]
    Eine Kurve \(c \in \mathcal{C}^1(I, \R^d)\) heißt nach Bogenlänge
    parametrisiert, falls
    \[ \abs{c'(s)} = 1 \; \forall s \in I. \]
    (Einheitsgeschwindigkeit)
    \[ [a,b] \subset I: L(c \vert_{[a,b]}) = \integralx{\abs{c'(s)}}{a}{b}{s}
    = b - a \]
    d. h. \(c\) bildet \([a,b]\) längentreu nach \(\R^d\) ab.
\end{defi}
\begin{satz}[Parametrisierung nach Bogenlänge]
    Sei \( \gamma \in \mathcal{C}^1([a,b],\R^d) \) eine Kurve 
    mit Länge \(L = L(\gamma)\) und \(\gamma'(t) \neq 0 \; \forall t \in [a,b] \)
    Dann gibt es eine \(\mathcal{C}^1\)-Bijektion \(\varphi\) 
    mit \( \varphi' > 0 \) so, dass \(c = \gamma \circ \varphi\) nach 
    Bogenlängeparametrisierung ist.
\end{satz}
\begin{bew}
    Bogenlängefunktion
    \(\abb{\sigma}{[a,b]}{0, L}; t \mapsto \integralx{\abs{\gamma'(s)}}{a}{t}{s}\)
    \[ \Rightarrow \sigma'(t) = \abs{\gamma'(t)} > 0 \;\forall t. \]
    \( \Rightarrow \sigma \) ist eine Bijektion der Klasse 
    \(\mathcal{C}^1\).\\
    Sei \(\varphi\) die Umkehrfunktion von \(\sigma\) (siehe Ana 1).
    \[\abs{(\gamma \circ \varphi)'(s)} = \abs{\gamma'(\varphi(s))\varphi'(s)} 
    = \abs{\gamma'(\varphi(s))} \varphi'(s) 
    = \abs{\gamma'(\varphi(s))} \frac{1}{\sigma'(\varphi(s))} 
    =  \abs{\gamma'(\varphi(s))} \frac{1}{\gamma'(\varphi(s))} = 1. \]
\end{bew}
\begin{bem}
    \(\mathcal{C}^1\)-Kurven mit \(\gamma'(s) \neq 0 \; \forall s\)
    heißen regulär.\\
    Neilsche Parabel 
    \( \abb{\gamma}{\R}{\R^2}; t \mapsto (t^2, t^3)\)\\
    \(\gamma'(t) = (2t, 3t^2), \gamma(0) = (0, 0)\)
\end{bem}
\begin{defi}[Stückweise \(\mathcal{C}^1\)-Kurve]
    \(\gamma \in \mathcal{C}([a,b], \R^d)\) heißt stückweise
    \(\mathcal{C}^1\)-Kurve, falls eine Zerlegung
    \(a = t_1 < t_1 < \ldots < t_N = b\) existiert so, dass
    für \( I_k = [t_{k-1}, t_k]\)
    \[ \gamma \vert_{I_k} \in \mathcal{C}^1(I_k, \R^d) \; \forall k = 1,\ldots,N \]
\end{defi}
\begin{notation}
    \( \gamma \in P \mathcal{C}^1([a,b], \R^d)\)
    piecewise continously diffentiable
\end{notation}
Setzen wir \(\gamma'(t_k):= 0, k = 1 \ldots N\) \\
\( \Rightarrow \abb{\gamma'}{[a.b]}{\R^d} \) stückweise stetig 
und somit (Riemann-)integrierbar.
Und dasselbe für \( \abb{\abs{\gamma'} }{[a,b]}{\R} \).
\(\Rightarrow \) Bogenlänge \(L(\gamma) = \integralx{\abs{\gamma'(s)}}{0}{b}{s}\)
auch für \(\gamma \in P \mathcal{C}'([a,b], \R^d)\) erklärt
und hängt nicht von Unterteilung ab.
\begin{defi}[Linienintegral]
    Ist \( F \in \mathcal{C}(U, \R^d) \), \(U \subset \R^d\) offen. 
    \( \gamma \in P\mathcal{C}'([a,b], U) \), so heißt 
    \[ \int_\gamma F\cdot d \vec{x} 
    := \integralx{\scalarprod{ F(\gamma(t)) }{ \gamma'(t) } }{a}{b}{t} \]
\end{defi}
\(F\) Kraftfeld \(\integral{F}{\gamma}{1} = \) verrichtete Arbeit.
\(\integralx{F(\gamma(t))\cdot \gamma'(t)}{0}{b}{t}\) Merkregel
\begin{bsp}\leavevmode
    \begin{enumerate}
        \setcounter{enumi}{5}
        \item \( \abb{F}{\R^3 \setminus \set{0}}{\R^3}, x \mapsto F(x) = -c \frac{x}{\abs{x}^3} \)
        \item Winkelfeld in \( \R^2 \): 
        \[ \abb{W}{\R^2 \setminus \set{0}}{\R^2}, W(x,y) 
        := (\frac{-y}{x^2 + y^2} , \frac{x}{x^2 + y^2}) 
        = \frac{1}{r^2} (-y, x) \]
        \[ \abb{\gamma}{[a,b]}{\R^2 \setminus \set{0}} \quad 
        \gamma(t) = r(t) (\cos(\theta(t)), \sin(\theta(t))), r, d \in \mathcal{C}^1([a,b]) \]
        \[ \int_\gamma w d \vec{x} = \integralx{\scalarprod{w(\gamma{t})}{\gamma'(t)}}{a}{b}{t} \]
        \[ = \integralx{\scalarprod{\frac{1}{r^3(t)} \begin{pmatrix}
            -r(t) \sin \theta(t) \\
            r(t) \cos \theta(t)
        \end{pmatrix}}{r(t)\theta'(t) \begin{pmatrix}
            -\sin \theta(t) \\
            \cos \theta(t)
        \end{pmatrix} + r'(t) \begin{pmatrix}
            \cos \theta(t) \\
            \sin \theta(t)
        \end{pmatrix}}}{a}{b}{t} \]
        \[ = \integralx{\theta'(t)}{a}{b}{t} = \theta(b) - \theta(a). \]
    \end{enumerate}
\end{bsp}
\end{document}