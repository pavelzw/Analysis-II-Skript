\documentclass[../ana2.tex]{subfiles}

\begin{document}
\setcounter{section}{18}
\section{Kuurvenintegrale}
\begin{defi}[\(\mathcal{C}^1\)-Kurve]
    Ist \( I \subset \R \) Intervall, \( \gamma \in \mathcal{C}^1(I, \R^d) \)
    so heißt \(\gamma \in \mathcal{C}^1\)-Kurve.
\end{defi}
\begin{bspe}\leavevmode
    \begin{enumerate}
    \item \( p, v \in \R^d, v \neq 0, \quad \gamma(t) = p + tv, t \in \R \).
    \item \(\abb{\gamma}{\R}{\R^2} \gamma(t) = (r(\cos t, \sin t) \)
    \item Schraubenlinie \( \abb{\gamma}{\R}{\R^3} t \mapsto \gamma(t) = (r \cos t, r \sin t, at) \).
    \item Spirale \( \abb{\gamma}{\R}{\R^3}, t \mapsto \gamma(t) = e^{-t} (\cos t, \sin t) \)
    \item \( \abb{\gamma}{I}{R^2}; t \mapsto \gamma(t) = (\cos t, \sin (2t))\)
    \end{enumerate}
\end{bspe}
\begin{defi}[Bogenlänge]
    Sei \(\abb{\gamma}{[a,b]}{\R^d}\ \mathcal{C}^1\)-Kurve
    \[ L(\gamma) := \integralx{\abs{\gamma'(t)}}{a}{b}{t} \]
    Bogenlänge (Länge der Kurve).\\
    Zerlegung \(a = t_0 < t_1 < \ldots < t_N = b\)
    \[ \sum_{j=1}^N \abs{\gamma(t_j) - \gamma(t_{j-1})} 
    = \sum_{j=1}^N \abs{\gamma'(t_j)} (t_j - t_{j-1}) 
    \approx \integralx{\abs{\gamma'(t)}}{a}{b}{t} \]
\end{defi}
\begin{bsp}
    \( \gamma(t) = r (\cos t, \sin t), \quad 0 \leq t \leq 2\pi \).
    \[ L(\gamma) = \integralx{\abs{\gamma'(t)}}{0}{2\pi}{t}
    = \integralx{\abs{r \begin{pmatrix}
        -\sin t \\ \cos t
    \end{pmatrix}}}{0}{2\pi}{t} 
    = r \integralx{}{0}{2\pi}{t} = 2\pi r. \]
\end{bsp}
\begin{bsp}
    \(\gamma(t) = (r\cos t, r\sin t, at), 0 \leq t \leq 2 \pi \)
    \[ \gamma'(t) = (-r\sin t, r\cos t, a) \]
    \[ L(\gamma) = \integralx{\abs{\gamma'(t)}}{0}{2\pi}{t} 
    = \integralx{\abs{ \begin{pmatrix}
        -r \sin t \\ r \cos t \\ a
    \end{pmatrix} }}{0}{2\pi}{t} 
    = \integralx{(r^2 + a^2)^{1/2}}{0}{2\pi}{t} 
    = 2\pi \sqrt{r^2 + a^2}. \]
\end{bsp}
\begin{defi}[Umparametrisierung]
    Seien \(\abb{\gamma_j}{I_j}{\R^d}, j=1,2 \) zwei \(\mathcal{C}^1\)-Kurven
    Dann heißt \(\gamma_2\) Umparametrisierung von \( \gamma_1 \),
    falls es eine Bijektion \(\varphi \in \mathcal{C}^1(I_2, I_1)\)
    mit \(\varphi' \neq 0 \) existiert mit 
    \[ \gamma_2 = \gamma_1 \circ \varphi. \]
    Die Bijektion \( \varphi \) heißt Parametertransformation.
\end{defi}
\begin{lem}
    Sind \(\gamma_j \in \mathcal{C}^1(I_j, \R^d), j = 1,\ldots, \gamma_2\) Umparametrisierung  
    von \( \gamma \), so ist 
    \[ L(\gamma_1) = L(\gamma_2). \]
\end{lem}
\begin{bew}
    Haben \( \gamma_2 = \gamma_1 \circ \varphi \), \( \varphi \in \mathcal{C}^1(I_2, I_1), 
    \varphi' \neq 0 \).
    \[ \gamma_2'(t) = (\gamma_1 \circ \varphi)' 
    = \gamma_1'(\varphi(t)) \cdot \varphi'(t). \]
    \[ \Rightarrow \abs{\gamma_2'(t)} 
    = \abs{\gamma'_1(\varphi(t)) \cdot \varphi'(t)} 
    = \pm \abs{\gamma'_1(\varphi(t))} \cdot \varphi'(t) \\
    +: \varphi'> 0;\forall t \in I_2 \\
    -: \varphi'< 0;\forall t \in I_2 \]
    \begin{align*}
        \Rightarrow L(\gamma_2) = \integralx{\abs{ \gamma_2'(t) }}{a_2}{b_2}{t}
        &= \pm \integralx{\abs{\gamma_2'(\varphi(t))} \varphi'(t) }{a_2}{b_2}{t} \\
        &= \pm \integralx{\abs{\gamma_1'(s)}}{\varphi(a_2)}{\varphi(b_2)}{s}
    \end{align*}
    \[ = \begin{cases}
        \integralx{\abs{ \gamma_1'(s) }}{a_1}{b_1}{s}, & \varphi' > 0 \\
        - \integralx{\abs{\gamma_1'(s)}}{b_1}{a_1}{s} 
        = \integralx{\abs{\gamma_1'(s)}}{a_1}{b_1}{s}, & \varphi < 0 
    \end{cases}\]
\end{bew}
\begin{defi}[Parametrisierung nach Bogenlänge]
    Eine Kurve \(c \in \mathcal{C}^1(I, \R^d)\) heißt nach Bogenlänge
    parametrisiert, falls
    \[ \abs{c'(s)} = 1 \; \forall s \in I. \]
    (Einheitsgeschwindigkeit)
    \[ [a,b] \subset I: L(c \vert_{[a,b]}) = \integralx{\abs{c'(s)}}{a}{b}{s}
    = b - a \]
    d. h. \(c\) bildet \([a,b]\) längentreu nach \(\R^d\) ab.
\end{defi}
\begin{satz}[Parametrisierung nach Bogenlänge]
    Sei \( \gamma \in \mathcal{C}^1([a,b],\R^d) \) eine Kurve 
    mit Länge \(L = L(\gamma)\) und \(\gamma'(t) \neq 0 \; \forall t \in [a,b] \)
    Dann gibt es eine \(\mathcal{C}^1\)-Bijektion \(\varphi\) 
    mit \( \varphi' > 0 \) so, dass \(c = \gamma \circ \varphi\) nach 
    Bogenlängeparametrisierung ist.
\end{satz}
\begin{bew}
    Bogenlängefunktion
    \(\abb{\sigma}{[a,b]}{0, L}; t \mapsto \integralx{\abs{\gamma'(s)}}{a}{t}{s}\)
    \[ \Rightarrow \sigma'(t) = \abs{\gamma'(t)} > 0 \;\forall t. \]
    \( \Rightarrow \sigma \) ist eine Bijektion der Klasse 
    \(\mathcal{C}^1\).\\
    Sei \(\varphi\) die Umkehrfunktion von \(\sigma\) (siehe Ana 1).
    \[\abs{(\gamma \circ \varphi)'(s)} = \abs{\gamma'(\varphi(s))\varphi'(s)} 
    = \abs{\gamma'(\varphi(s))} \varphi'(s) 
    = \abs{\gamma'(\varphi(s))} \frac{1}{\sigma'(\varphi(s))} 
    =  \abs{\gamma'(\varphi(s))} \frac{1}{\gamma'(\varphi(s))} = 1. \]
\end{bew}
\begin{bem}
    \(\mathcal{C}^1\)-Kurven mit \(\gamma'(s) \neq 0 \; \forall s\)
    heißen regulär.\\
    Neilsche Parabel 
    \( \abb{\gamma}{\R}{\R^2}; t \mapsto (t^2, t^3)\)\\
    \(\gamma'(t) = (2t, 3t^2), \gamma(0) = (0, 0)\)
\end{bem}
\begin{defi}[Stückweise \(\mathcal{C}^1\)-Kurve]
    \(\gamma \in \mathcal{C}([a,b], \R^d)\) heißt stückweise
    \(\mathcal{C}^1\)-Kurve, falls eine Zerlegung
    \(a = t_1 < t_1 < \ldots < t_N = b\) existiert so, dass
    für \( I_k = [t_{k-1}, t_k]\)
    \[ \gamma \vert_{I_k} \in \mathcal{C}^1(I_k, \R^d) \; \forall k = 1,\ldots,N \]
\end{defi}
\begin{notation}
    \( \gamma \in P \mathcal{C}^1([a,b], \R^d)\)
    piecewise continously diffentiable
\end{notation}
Setzen wir \(\gamma'(t_k):= 0, k = 1 \ldots N\) \\
\( \Rightarrow \abb{\gamma'}{[a.b]}{\R^d} \) stückweise stetig 
und somit (Riemann-)integrierbar.
Und dasselbe für \( \abb{\abs{\gamma'} }{[a,b]}{\R} \).
\(\Rightarrow \) Bogenlänge \(L(\gamma) = \integralx{\abs{\gamma'(s)}}{0}{b}{s}\)
auch für \(\gamma \in P \mathcal{C}'([a,b], \R^d)\) erklärt
und hängt nicht von Unterteilung ab.
\begin{defi}[Linienintegral]
    Ist \( F \in \mathcal{C}(U, \R^d) \), \(U \subset \R^d\) offen. 
    \( \gamma \in P\mathcal{C}'([a,b], U) \), so heißt 
    \[ \int_\gamma F\cdot d \vec{x} 
    := \integralx{\scalarprod{ F(\gamma(t)) }{ \gamma'(t) } }{a}{b}{t} \]
\end{defi}
\(F\) Kraftfeld \(\integral{F}{\gamma}{1} = \) verrichtete Arbeit.
\(\integralx{F(\gamma(t))\cdot \gamma'(t)}{0}{b}{t}\) Merkregel
\begin{bsp}\leavevmode
    \begin{enumerate}
        \setcounter{enumi}{5}
        \item \( \abb{F}{\R^3 \setminus \set{0}}{\R^3}, x \mapsto F(x) = -c \frac{x}{\abs{x}^3} \)
        \item Winkelfeld in \( \R^2 \): 
        \[ \abb{W}{\R^2 \setminus \set{0}}{\R^2}, W(x,y) 
        := (\frac{-y}{x^2 + y^2} , \frac{x}{x^2 + y^2}) 
        = \frac{1}{r^2} (-y, x) \]
        \[ \abb{\gamma}{[a,b]}{\R^2 \setminus \set{0}} \quad 
        \gamma(t) = r(t) (\cos(\theta(t)), \sin(\theta(t))), r, d \in \mathcal{C}^1([a,b]) \]
        \[ \int_\gamma w d \vec{x} = \integralx{\scalarprod{w(\gamma{t})}{\gamma'(t)}}{a}{b}{t} \]
        \[ = \integralx{\scalarprod{\frac{1}{r^3(t)} \begin{pmatrix}
            -r(t) \sin \theta(t) \\
            r(t) \cos \theta(t)
        \end{pmatrix}}{r(t)\theta'(t) \begin{pmatrix}
            -\sin \theta(t) \\
            \cos \theta(t)
        \end{pmatrix} + r'(t) \begin{pmatrix}
            \cos \theta(t) \\
            \sin \theta(t)
        \end{pmatrix}}}{a}{b}{t} \]
        \[ = \integralx{\theta'(t)}{a}{b}{t} = \theta(b) - \theta(a). \]
    \end{enumerate}
\end{bsp}
\( \gamma \ \mathcal{C}^1 \)-Kurve \( F \in \mathcal{C}(U, \R^d) \)
in \(U \subset \R^d\) offen. \( \abb{\gamma}{[a,b]}{U} \).
\[\integralx{F}{\gamma}{}{\vec{x}} 
:= \integralx{\scalarprod{F(\gamma(t))}{\gamma'(t)}}{a}{b}{t}\]
\begin{lem}[Eigenschaften des Kurvenintegrals]
    a): linear: \\
    \( F_j \in \mathcal{C}(U, \R^d), \lambda_j \in \R, j = 1,2 \)
    \[ \gamma \in P\mathcal{C}^1([a,b], U) \Rightarrow 
    \int_\gamma (\lambda_1 F_1 + \lambda_2 F_2)\; d\vec{x} 
    = \lambda_1 \int_\gamma F_1\; d\vec{x} + \lambda_2 \int_\gamma F_2\; d\vec{x}. \]   
    b): Additivität bei Zerlegung:\\
    Ist \(\gamma \leftarrow P\mathcal{C}^1([a,b], U), F \leftarrow \mathcal{C}^1(U, \R^d) \)
    und \(a = t_0 < \ldots < t_N = b\) Zerlegung von \([a,b]\)
    \[\gamma_k := \gamma\vert_{[t_k1, t_k]}
    \Rightarrow \integralx{F}{\gamma}{}{\vec{x}}
    = \sum_{k=1}^N\integralx{F}{\gamma}{}{\vec{x}}\]    
    c) Invarianz unter Umparametrisierung:\\
    Sind \(\gamma \in P\mathcal{C}^1(I,\R^d), \varphi \in \mathcal{C}^1(I_2, I_1)\)
    eine Parametertransformation, so gilt, je nach Vorzeichen von \(\varphi'\)
    \[ \int_{\gamma \circ \varphi} F \; d\vec{x} = \pm \int_\gamma F \; d\vec{x} \]
    \(+: \varphi' > 0, -: \varphi' < 0\)
\end{lem}
\begin{bew}
    a) und b) folgen aus alten Eigenschaften des Integrals!
    c): Sei \(I_1 = [a_1,b_1], I_2 = [a_2,b_2]\)
    \begin{align*}
        \integralx{F}{\gamma}{}{\vec{x}} 
        &= \integralx{\scalarprod{F(\gamma \circ \varphi(t))}{(\gamma \circ \varphi)'(t)}}{a_2}{b_2}{t} \\
        &= \integralx{ \scalarprod{ F(\gamma(s)) }{ \gamma'(s) } }{\varphi(a_2)}{\varphi(b_2)}{s}
    \end{align*}
    Ist \(\varphi' > 0 \Rightarrow \varphi(a_2) = a_1 \wedge \varphi(b_2) = b_1\)\\
    Ist \(\varphi' < 0 \Rightarrow \varphi(a_2) = b_1 \wedge \varphi(b_2) = a_1\)
    \[ = \begin{cases}
        \integralx{\scalarprod{F(\gamma(s))}{\gamma'(s)}}{a_1}{b_1}{s} & \varphi' > 0 \\
    \integralx{\scalarprod{F(\gamma(s))}{\gamma'(s)}}{b_1}{a_1}{s} & \varphi' < 0 \\
    \end{cases} = \begin{cases}
        \int_\gamma F \; d\vec{x} & \varphi' > 0 \\
        - \int_\gamma F \; d\vec{x} & \varphi' < 0
    \end{cases} \]    
\end{bew}
\begin{lem}
    Sei \(U \subset \R^d\) offen, \(F \in \mathcal{C}^1(U, \R^d), \gamma \in P\mathcal{C}^1([a,b], U)\)
    \[ \abs{\int_\gamma F d\vec{x}} \leq \norm{F \circ \gamma}_{\infty,[a,b]} L(\gamma) 
    = \underset{t \in [a,b]}{\sup} \abs{F(\gamma(t))} L(\gamma)\]
\end{lem}
\begin{bew}
    \begin{align*}
        \abs{\int_\gamma F d\vec{x}} &= \abs{\integralx{\scalarprod{F(\gamma(t))}{\gamma'(t)}}{a}{b}{t}} \\
        &\leq \integralx{\abs{ \scalarprod{F(\gamma(t))}{\gamma'(t)}}}{a}{b}{t} \\
        &\leq \norm{F \circ \gamma}_{\infty, [a,b]} \integralx{\abs{\gamma'(t)}}{a}{b}{t}.
    \end{align*}
\end{bew}
\begin{defi}[Gradientenfeld]
    Sei \(U \subset \R^d\) offen. Das Vektorfeld \(F \subset \mathcal{C}(U,\R^d)\)
    heißt Gradientenfeld (bzw. Konservativ),
    falls ein \(\varphi \in \mathcal{C}^1(U, \R)\) mit 
    \[ \nabla \varphi = F. \] 
    \(\varphi\) heißt Stammfunktion oder Potential von \(F\).
\end{defi}
\begin{lem}[Stammfunktion ist eindeutig bis auf Konstante]
    Ist \( U \subset \R^d \) wegweise zusammenhängend, so 
    ist eine Stammfunktion \( \varphi \) von \( F \in \mathcal{C}(U, \R) \) 
    bis auf eine additive Konstante eindeutig bestimmt.
\end{lem}
\begin{bew}
    Sind \(\varphi_1, \varphi_2 \) Stammfunktionen zu \(F\)
    \[ \Rightarrow \nabla(\varphi_1-\varphi_2) = \nabla \varphi_1 - \nabla \varphi_2 
    = F-F = 0 \]
    \( \overundersett{alter}{Satz}{\Rightarrow} a := \varphi_1 + \varphi_2 \) ist 
    konstant auf \( U \). 
    \[ \Rightarrow \varphi_1 = \varphi_2 + a \text{ auf } U. \]
\end{bew}
\begin{defi}
    Eine Kurve \(\abb{\gamma}{[a,b]}{\R^d}\)
    heißt geschlossen, falls \(\gamma(a) = \gamma(b)\).
\end{defi}
\begin{satz}[Wegunabhängigkeit der Kurvenintegrale]
    Sei \( U \subset \R^d\) offen und wegweise zusammenhängend.
    Für \( F \in \mathcal{C}(U, \R^d) \) sind äquivalent
    \begin{enumerate}
        \item \( F \) ist Gradientenfeld.
        \item \(\gamma \in P\mathcal{C}^1([a,b], U)\) ist 
        \[ \int_\gamma F d\vec{x} = 0 \]
        \item  Für je zwei Kurven \(\gamma_0 \circ \gamma_1 \in P\mathcal{C}^1([a,b],U)\)
        mit \(\gamma_0(a) = \gamma_1(a), \gamma_0(b) = \gamma_1(b)\) ist    
        \[ \int_{\gamma_1} F d\vec{x} = \int_{\gamma_2} F d\vec{x} \]
    \end{enumerate}
\end{satz}
\begin{bew}
    Zeigen \(a) \Rightarrow b) \Rightarrow c) \Rightarrow a) \)\\
    \(a) \Rightarrow b) \):\\
    Ist \(F = \nabla \varphi, \varphi in \mathcal{C}^1(U,\R)\),
    \(\gamma in P \mathcal{C}^1([a,b], U)\) abgeschlossen
    \[ \ddxpartial{t}(\varphi(\gamma(t)) = D\varphi(\gamma(t))[\gamma'(t)]
    = \scalarprod{\nabla \varphi(\gamma(t))}{\gamma'(t)} \]
    \begin{align*}
        \int_\gamma F d \vec{x} &= \integralx{\scalarprod{F(\gamma(t))}{\gamma'(t)}}{a}{b}{t}\\
        &= \integralx{\ddx{t}(\varphi(\gamma(t)))}{a}{b}{t}\\
        &= \varphi(\gamma(b)) - \varphi(\gamma(a))\\
        &= 0
    \end{align*}
    da \(\gamma(a) = \gamma(b)\).\\    
    \(b) \Rightarrow c) \):\\
    z.B. sind \(\abb{\gamma_1, \gamma_2}{[a,b]}{U} \)
    \(\gamma_1(a) = \gamma_2(a), \gamma_1(b) = \gamma_2(b)\)
    \(\gamma(t) := \begin{cases}
        \gamma_1(t), a \leq t \leq b\\
    \gamma_2(2b - t), b \leq t \leq 2b - a 
    \end{cases} \)    
    Sind \[\gamma_1, \gamma_2 \in P \mathcal{C}^1([a,b], U)
    \Rightarrow \gamma \in P \mathcal{C}^1([a,b], U)\]
    \[0 = \int_gamma F ; d\vec{x} = \int_gamma_1 F ; d\vec{x} - \int_gamma_2 F ; d\vec{x} 
    \Rightarrow c)\]    
    \(c) \Rightarrow a)\):\\
     Sei \( x_0 \in U \) beliebig, aber fest, \( x \in U \)
    \[ \Rightarrow \exists \gamma_x \in P \mathcal{C}^1([0,1], U) \]
    \[ \gamma_x(0) = x_0, \gamma_x(1) = x. \]
    Setzen \( \varphi(x) := \int_{\gamma_x} F \; d\vec{x} \) 
    wohldefiniert.\\
    \( h \in \R, \abs{h} \) klein, \( x + h e_j \).
    Verbinde \( x \) mit \( x + h e_j \) durch Pfad \( \abb{\tilde{\gamma}}{[0,1]}{U}, 
    \tilde{\gamma}(s) = x + h s e_j \).
    \[ \varphi(x + e_j) - \varphi(x) 
    = \int_{\gamma_x} f \; d\vec{x} + \int_{\tilde{\gamma}} F \; d\vec{x} - \int_{\gamma_x} F \; d \vec{x} \]
    \[ = \int_{\tilde{\gamma}} F \; d\vec{x} 
    = \integralx{\scalarprod{ F(\tilde{\gamma}(s) }{ \tilde{\gamma}'(s) }}{0}{1}{s} 
    = F_j (x+sh e_j) \]
    \[ = \integralx{\scalarprod{F(x + h s e_j)}{ h e_j }}{0}{1}{s} 
    = h \integralx{ \scalarprod{ F(x + h s e_j) }{ e_j } }{0}{1}{s} 
    = h \integralx{F_j (x + hs e_j)}{0}{1}{s}. \]
    \[ \Rightarrow \frac{ \varphi(x + h e_j) - \varphi(x) }{h} 
    = \integralx{F_j(x+hse_j)}{0}{1}{s}  \overset{h \rightarrow 0}{\longrightarrow} 
    \integralx{F(x)}{0}{1}{s} 
    = F(x) \]
    \[ \Rightarrow \varphi \in \mathcal{C}^1 (U, \R) \]
    \[ \Rightarrow \partial_j \varphi(x) = \limesx{h}{0} \varphi(x) 
    = \limesx{h}{0} \frac{\varphi(x + he_j) - \varphi(x)}{h} \]
    existiert \(F_j(x)\) stetig.
    \[ \Rightarrow \nabla \varphi(x) = F(x) \;\forall x \in U. \]
\end{bew}
\begin{satz}[Rotationsfreiheit]
    \( U \subset \R^d \) ist \( F \in \mathcal{C}(U, \R^d) \)
    ein Gradientenfeld.
    \[ \Rightarrow \partial_l F_j = \partial_j F_l \text{ in } U \;\forall 1 \leq l,j \leq d. \]
\end{satz}
\begin{bew}
    \( F = \nabla \varphi \Rightarrow \varphi \in \mathcal{C}^2(U, \R) \)
    \[ \oversett{Schwarz}{\Rightarrow} \partial_l \partial_j \varphi = \partial_j \partial_l \varphi \]
    \[ \Leftrightarrow \partial_l F_j = \partial_j F_l. \]
\end{bew}
\( d = 3 \): Können es schreiben als 
\[ 0 = \mathrm{rot} F = \nabla x F  = \begin{pmatrix}
    \partial_2 F_3 - \partial_3 F_2 \\
    -(\partial_1 F_3 - \partial_3 F_1) \\
    \partial_1 F_2 - \partial_2 F_1 
\end{pmatrix} \]
\begin{bsp}
    \(\abb{F}{R^2}{R^2}  F(x,y) = (-y, x)\)
    hat keine Stammfunktion, denn \(\partial_1 F_2 = 1 \neq \partial_2 F_1 = -1\)
\end{bsp}
\begin{bsp}
    Winkelvektorfeld.
    \[ \abb{W}{\R^2 \setminus \set{0}}{\R} \]
    \[ W(x,y) = \begin{pmatrix}
        \frac{-y}{x^2 + y^2} & \frac{x}{x^2 + y^2}
    \end{pmatrix} \Rightarrow \partial_1 W_2 = \partial_2 W_1 \]
    aber es hat keine Stammfunktion.\\
    Nehme Pfad \( \gamma(t) = (\cos t, \sin t), 0 \leq t \leq 2\pi \)
    \[ \Rightarrow \int_\gamma w \; d\vec{x} = 2\pi \neq 0. \]
\end{bsp}
Haben gesehen: \(F\) ist Gradientenfeld. 
\( F = \nabla \varphi, \abb{\varphi}{U}{\R}, U \subset \R^d \) 
offen 
\[ \Rightarrow \partial_j \partial_l \varphi = \partial_j F_l = 
\partial_l F_j = \partial_l \partial_j \varphi. \] 
wenn \( F \in \mathcal{C}^1, \varphi \in \mathcal{C}^2 \)\\
Winkelvektorfeld \(\abb{W}{\R^2 \setminus \set{0}}{\R^2}; 
(x,y) \mapsto (-\frac{y}{x^2+y^2}, \frac{x}{x^2+y^2})\)\\
ist kein Gradientenfeld.
\begin{bsp}[Aharonov-Bohm Vektorpotenzial]
    \(\varphi(x,y) = \arccos \frac{x}{\sqrt{x^2+y^2}}\)
    auf oberer Halbebene \(\set{(x,y): x \in \R, y > 0}\)
\end{bsp}
\begin{defi}[Homotopie (verbiegen von Pfaden)]
    Sei \(U \subset \R^d\) offen. Eine Homotopie in \(U\)
    zwischen Kurven \(\gamma_j \in \mathcal{C}([a,b], U), j=1,2,\ldots\)
    ist eine Abbildung \(\gamma \in \mathcal{C}([a,b]\times [0,1], U)\) mit
    \[ \gamma(\cdot, 0) = \gamma_0, \gamma(\cdot, 1) = \gamma_1 \].
    Gilt \( \gamma_0(a) = \gamma_1(a) = p, 
    \gamma_0(b) = \gamma_1(b) = q \) und gibt es eine Homotopie mit    
    \(\gamma(a,t) = p \; \forall t \in [0,1]\) mit 
    \(\gamma(b,t) = q \; \forall t \in [0,1]\), so heißen \(\gamma_0, \gamma_1\)
    homotop mit festen Endpunkten. Gilt \(\gamma_0(a) = \gamma_0(b), \gamma_1(a) = \gamma_1(b)\)
    und gibt es eine Homotopie mit \(\gamma(a, t) = \gamma(b,t) = \; \forall t \in [0,1]\)
    so heißen \(\gamma_0, \gamma_1\) geschlossen homotop in \(U\).
\end{defi}
\begin{lem}
    Sei \( U \subset \R^d, F \in \mathcal{C}^2(U, \R^d), 
    \gamma \in \mathcal{C}^1([a,b] \times [0,1], U) \) und 
    \( \partial_s \partial_t \gamma \in \mathcal{C}([a,b] \times [0,1], \R^2) \).
    \[ \int_{\gamma(\cdot, 1)} F \; d \vec{x}
    - \int_{\gamma(\cdot, 0)} 
    = \int_{\gamma(b, \cdot)} F \; d \vec{x} - \int_{\gamma(a, \cdot)} \; d\vec{x} 
    + \int_0^1 \int_a^b (\scalarprod{Df \circ \gamma[\ddxpartialboth{\gamma}{t}] }{ \ddxpartialboth{\gamma}{s} } 
    - \scalarprod{ DF \circ \gamma[\ddxpartialboth{\gamma}{s}] }{ \ddxpartialboth{\gamma}{t} } ) \; ds\; dt \]
\end{lem}
\begin{bew}
    Schwarz \( \partial_s \partial_t \gamma = \partial_t \partial_s \gamma \)
    \begin{align*}
        \ddxpartial{t} \int_{\gamma(\cdot, t)} F d\vec{x} 
        &= \ddxpartial{t} \integralx{\scalarprod{F(\gamma(s,t))}{\ddxpartialboth{\gamma}{s}(s,t)}}{a}{b}{s}\\
        &= \int_a^b \ddxpartial{t} \scalarprod{Df \circ \gamma[\ddxpartialboth{\gamma}{t}] }{ \ddxpartialboth{\gamma}{s}(s,t) } \; ds\\
        &= \int_a^b \scalarprod{DF \circ \gamma [\ddxpartialboth{\gamma}{t}]}{\ddxpartialboth{\gamma}{s}} \; ds + [\scalarprod{F \circ \gamma}{\ddxpartialboth{\gamma}{t}}]_{s=a}^{s=b} 
        - \int_a^b \scalarprod{ \ddxpartial{s} F(\gamma(s,t))}{\ddxpartialboth{\gamma}{t} } \; ds
    \end{align*}
    Integration bzgl. \(t \in [0,1]\) liefert die Formel.
\end{bew}
\begin{bem}
    Bei festen Endpunkten: \(\Rightarrow \gamma(a,t), \gamma(b,t) \) sind konstant (in \(t\)).\\
    Ist \(\gamma(a, \cdot) = \gamma(b, \cdot)\) Homotopie geschlossen\\
    \(\Rightarrow\) aller wegintegrale heben sich gegenseitig weg.
\end{bem}
\begin{satz}[Fundamentalsatz der Algebra]
    Jedes kopmlexe Polynom vom Grad \( n \geq 1 \) hat 
    mindestens eine komplexe Nullstelle.
\end{satz}
\begin{bew}
    Wir betrachten Winkelvektorfeld \( \abb{W}{\R^2 \setminus \set{0}}{\R^2} \) 
    identifizieren \( \R^2 \) mit \( \C \).
    \[ p(z) = z^n + \cdots a_{n-1} z^{n-1} + \cdots + a_1 z + a_0 \; a_j \in \C \]
    \[ = z^n + q(z), \grad q \leq n - 1 \Rightarrow z^{-n} q(z) \rightarrow 0, \abs{z} \rightarrow \infty. \]
    \(\Rightarrow R > 0\) groß genug ist 
    \[ \abs{q(Re^{i \theta}) } leq \frac{1}{2} R^n, \theta \in [0, 2\pi] \]
    \[ \abb{\gamma_0}{[0,2\pi]}{{p(Re^{i\theta})}} = (Re^{i\theta})^n + q(Re^{i\theta}) \]
    Homotopie: \( \gamma(\theta, t) := (R e^{i\theta})^n + (1-t) q (R e^{i \theta}) \)
    \[ \abs{ \gamma(\theta, t) } \geq \abs{ R e^{i\theta} }^n - (1-t) \abs{ q (R e^{i\theta}) } 
    \geq R^n - \frac{1}{2} R^n \geq \frac{1}{2} \R^n > 0. \]
    \( \oversett{Lemma 17}{\Rightarrow} \partial_j w_e = \partial_e w_j \oversett{!}{=} R^n \)
    \[ \int_{\gamma_0} w d\vec{x} = \int_{\gamma(\cdot, 1)} w d\vec{x} = 2\pi n \]
    Ang. \( p \) hat keine Nullstelle.
    Homotopie: \( \tilde{\gamma} := p(p e^{i\theta}), 0 \leq p \leq R, \theta \in [0,2\pi] \)
    \[ \tilde{\gamma}(\theta, 0) = p(0) = a_0 \]
    \( \Rightarrow \int_{\tilde{\gamma(\cdot, R)}} w d\vec{x} 
    = \int_{\tilde{\gamma}(\cdot, 0) } w d\vec{x} = 0 \) \Lightning.
\end{bew}
\begin{lem}[Affine Homotopie]
    Sei \(U \subset \R^d \) offen, \( F \in \mathcal{C}^1(U, \R^d), \partial_j F_l = \partial_l F_j \)\\
    Kennen \( \gamma_0, \gamma_1 \in P\mathcal{C}^1([a,b], U) \) affine Homotopie 
    \[ \abb{\gamma}{[a,b] \times [0,1]}{\R^n}, \gamma(s,t) := (1-t)\gamma_0(s) + t \gamma_1(s) \]
    Haben \(\gamma_0,\gamma_1\) dieselben Endpunkte oder sind geschlossen und 
    \(\gamma([a,b]\times[0,1]) \subset U \)
    \[ \Rightarrow \int_{\gamma_0} F d\vec{x} = \int_{\gamma_1} F d\vec{x} \]
\end{lem}
\begin{bew}
    Sind \(\gamma_0, \gamma_1 \mathcal{C}^1 \Rightarrow\) folgt aus Lemma 17.\\
    \(\gamma_0,\gamma_1 P \mathcal{C}^1([a,b], U)\)\\    
    Beweis durch Bild!
\end{bew}
\begin{satz}[Homotopieinvarianz]
    Sei \( U \subset \R^n \) offen, \( F \in \mathcal{C}^1(U, \R^d), 
    \partial_j F_l = \partial_l F_j, \gamma_0, \gamma_1 \in P\mathcal{C}^1([a,b], U) \)
    Homotopie in \( U \) mit festen Endpunkten (oder geschlossene Homotopie)
    \[ \Rightarrow \int_{\gamma_0} F \; d\vec{x} = \int_{\gamma_1} F \; d\vec{x}. \]
\end{satz}
\begin{bew}
    Skript. Ist \( \gamma \) genügend glatt \( \Rightarrow \) Lemma 19.
\end{bew}
\begin{defi}
    Eine Menge \(U \subset \R^d\) heißt einfach zusammenhängend, wenn jede
    geschlossene Kurve \(\gamma \in \mathcal{C}([a,b], U)\) in \(U\)
    geschlossen homotop zu einer konstanten Kurve ist.
\end{defi}
\begin{bsp}
    \(U\) ist sternförmig, wenn 
    \( x_0 \in U: \forall x \in U, [x_0, x] \subset U \)
    \[ [x_0, x] = (1-t) x + tx_0 \; 0 \leq t \leq 1 \]
    \[ \gamma(s,t) = (1-t) \gamma(s) + t x_0 \]
\end{bsp}
\begin{satz}
    Sei \( U \subset \R^d \) offen und einfach zusammenhängend. Dann sind für ein Vektorfeld 
    \( F \in \mathcal{C}^1(U, \R^d) \) äquivalent:
    \begin{enumerate}
        \item \( \partial_j F_l = \partial_l F_j,\; l,j = 1,\ldots, d \).
        \item \( F \) hat eine Stammfunktion.
    \end{enumerate}
\end{satz}
\begin{bew}
    b) \(\Rightarrow\) a) Satz 15\\
    a) \(\Rightarrow\) b) Satz 20: \( \int_\gamma F \; d \vec{x} = 0 \) 
    für geschlossenen Weg \( \gamma \in P \mathcal{C}^1([a,b], U) \)\\
    \( \oversett{Satz 14}{\Rightarrow} F \) hat eine Stammfunktion \( \varphi \).
\end{bew}
\begin{bsp}
    \( U \) ist sternförmig: \\
    \( \exists x_0 \in U: \forall x \in U: [x_0, x] 
    = \set{ (1-t)x_0 + tx, 0 \leq t \leq 1 } \subset U \)
    \( \gamma \in P \mathcal{C}^1([a,b], U) \) geschlossen.\\
    \( \gamma(s,t) := (1-t)\gamma(s) + tx_0 \)\\
    Sei \( F \in \mathcal{C}^1(U, \R^d), \partial_jF_l = \partial_l F_j 
    \; \forall j, l = 1,\ldots, d\)\\
    \(\oversett{Satz 22}{\Rightarrow} F\) hat Stammfunktion (d. h. es ist ein Gradientenfeld)\\
    \(F\) in \(x \in U\), nehme \(\abb{\gamma_x}{[0,1]}{U}; t \mapsto (1-t)x_0 + tx.\)\\
    Dann ist \( \varphi(x) := \int_{\gamma_x} F \; d\vec{x} 
    = \integralx{\scalarprod{F(\gamma(t))}{\gamma'(t)}}{0}{1}{t} 
    = \integralx{\scalarprod{F((1-t)x_0 + tx)}{x - x_0}}{0}{1}{t} \)\\
    Tabelle: Eigenschaften von Kurvenintegralen\\
    \(F\) Gradientenfeld \(\oversett{Satz 14}{\Leftrightarrow} \int_\gamma F dx\)
    wegunabhängig \(\Leftrightarrow \int_{\tilde{\gamma}} F dx = 0 \forall \) geschlossene Wege \(\tilde{\gamma}\)\\
    \( \oversett{Satz 15}{\Rightarrow} \partial_j F_l = \partial_l F_j 
    \oversett{Satz 17}{\Leftrightarrow} \int_\gamma F d\vec{x} \) ist 
    homotopieinvariant.
\end{bsp}
\end{document}