\documentclass[../ana2.tex]{subfiles}

\begin{document}
\setcounter{section}{10}
\section{Der mehrdimensionale Mittelwertsatz}
\begin{satz}
    Sei \( I = [a,b] \subset \R \) ein 
    kompaktes Intervall 
    und \(\abb{f}{I}{\R^m} \) stetig und 
    differenzierbar in \( (a,b) \) mit 
    \[ \norm{Df(t)} 
    = \underset{\substack{h \in \R\\\abs{h} = 1}}{\sup} 
    \abs{Df(t)[h]}
    = \abs{f'(t)} \leq M \; \forall t \in (a,b) \]
    \[ \Rightarrow \abs{f(b) - f(a)} \leq M(b-a). \]
\end{satz}
\begin{bew}
    Wir geben uns etwas mehr Freiraum und zeigen, dass 
    \( \forall \varepsilon > 0 \ 
    a < \tilde{a} < \tilde{b} < b \)
    \[ \abs{f(\tilde{b}) - f(\tilde{a})} 
    \leq (M + \varepsilon)(\tilde{b} - \tilde{a}) \tag{\(**\)} \]
    Die Behauptung folgt dann im Grenzwert
    \( \tilde{b} \rightarrow b, \tilde{a} \rightarrow a \)
    (da \(f\) auf \( [a,b] \) stetig ist) 
    und \( \varepsilon \rightarrow 0 \).
    Zum Beweis von \((**)\): 
    \[ A := \set{t \in [\tilde{a}, \tilde{b}]:
    \abs{f(t)-f(\tilde{a})} \leq (M+\varepsilon)(t-\tilde{a})
    \; \forall \tilde{a} < t < \tilde{b}} \]
    Beachte: 
    \begin{enumerate}
        \item \( A \neq \emptyset \), da \( \tilde{a} \in A \).
        \item \( A \leq \tilde{b} \), also ist \( A \) 
        nach oben beschränkt.
    \end{enumerate}
    \( \overundersett{Vollst.}{von \(\R\)}{\Rightarrow} 
    c := \sup A \) existiert.\\
    Müssen somit zeigen, dass \( c = \tilde{b} \).\\
    Beachte: nach Konstruktion gilt \( A = [\tilde{a},c] \).\\
    Angenommen \( \tilde{a} \leq c < \tilde{b} \). \\
    Da \(f\) in \(c\) differenzierbar ist, folgt 
    \( \exists \abb{\eta}{[\tilde{a}, \tilde{b}]}{\R^m} \) 
    stetig in \(c\) mit \( \eta(c) = 0 \) 
    und 
    \[ f(c + h) = f(c) + Df(c)[h] + \abs{h} \eta(h) \tag{\(***\)} \]
    Da \( \limesx{h}{0} \eta(h) = 0 \Rightarrow \exists \delta > 0: 
    \abs{\eta(h)} < \varepsilon \; \forall \abs{h} < \delta \). \\
    Sei \( c \leq x \leq c + \delta \):
    \begin{align*}
        \Rightarrow \abs{ f(x) - f(\tilde{a}) } 
        &= \abs{ f(x) - f(c) + f(c) - f(\tilde{a})} \\
        &\leq \abs{f(x) - f(c)} 
        + \underbrace{\abs{f(c) - f(\tilde{a})}}_{\leq (M+\varepsilon)(c - \tilde{a})}
    \end{align*}
    und aus \( (***) \) folgt 
    \begin{align*} 
        \abs{ f(x) - f(c) } &= \abs{Df(x)[x - c]} + \abs{x - c}\eta(x-c) \\
        &\leq \abs{Df(c)[x - c]} + \abs{ \abs{x-c}\eta(x-c) }\\
        &\leq \underbrace{\norm{ Df(c) }}_{\leq M} \abs{x-c}
        + \abs{x - c} \underbrace{\abs{ \eta(x-c) }}_{<\varepsilon} \\
        &\leq (M+\varepsilon) \abs{x-c}
    \end{align*}
    \begin{align*}
        \Rightarrow \abs{f(x) - f(\tilde{a})} 
        &\leq (M+\varepsilon)(x - c + c - \tilde{a}) \\
        &= (M+\varepsilon)(x - \tilde{a}) \;\forall c \leq x < c + \delta
    \end{align*}
    Also ist z. B. \( c < c + \frac{1}{2} \delta =: x \in A \)
    \Lightning{} zu \(c = \sup A\).
    \[ \Rightarrow c =: \tilde{b}. \]
\end{bew}
\begin{kor}
    Sei \( U \subset \R^n \) offen, \(x,y \in U\) mit 
    \[ [x,y] := \set{x+t(y-x): 0 \leq t \leq 1} \subset U \]
    (das Liniensegment, welches \(x\) und \(y\) verbindet).
    Ferner sei \( \abb{f}{U}{\R^m} \) stetig auf \( [x,y] \)
    und differenzierbar in \((x,y) := \set{x+t(y-x): 0 < t < 1} \)
    mit \( \norm{Df(z)} \leq M \;\forall z \in (x,y) \).\\
    Dann ist 
    \[ \abs{f(y) - f(x)} \leq M \abs{y - x}. \]
\end{kor}
\begin{bew}
    Sei \( I = [0,1] \subset \R \) und 
    \( \abb{g}{I}{\R^m} \)
    \[ g(t) := f(x + t(y - x)). \]
    Da \(f\) differenzierbar auf \((x,y)\) ist, ist \(g\)
    differenzierbar in \(0,1\) und
    \[ g'(t) = Df(x+t(y-x))[y-x] (\text{Kettenregel}) \]
    Aus Satz 1 folgt die Behauptung.
\end{bew}
\begin{kor}
    Sei \(U \subset \R^n\) offen, \(x, y \in U\)
    mit \([x,y] \subset U, \abb{f}{U}{\R^m}\) differenzierbar
    auf ganz \(U\). \(\Rightarrow \forall z \in U\)
    \[\abs{f(y)-f(x) - Df(z)[y-x]} \leq \abs{y-x} 
    \underset{\rho \in U}{\sup} \norm{Df(\rho)- Df(z)} \]
\end{kor}
\begin{bew}
    Sei \(\abb{g}{U}{\R^m}\) definiert durch 
    \[ g(\rho) := f(\rho) - Df(z)[\rho]. \]
    Dann ist \( g \) differenzierbar mit 
    Ableitung 
    \[ D_g(\rho) = Df(\rho) - Df(z). \]
    Aus Korollar 2 folgt die Behauptung.
\end{bew}
\end{document}