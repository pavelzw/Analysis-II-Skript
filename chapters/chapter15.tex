\documentclass[../ana2.tex]{subfiles}

\begin{document}
\setcounter{section}{14}
\section{Taylor}

Motivation: Sei \( \abb{f}{(a,b)}{\R} \ k \)-mal stetig 
differenzierbar.
\( x \in (a,b). x + h \in (a,b). \)
\[ f(x + h) = f(x) + \dots \]
\[ \abb{g}{[0,1]}{\R}, t \mapsto f(x+th) = g(t) \]

\begin{align*}
    \Rightarrow f(x + h) - f(x) &= g(1) - g(0) 
    = \integralx{g'(s)}{0}{1}{s} \\
    &= \integralx{1 \cdot g'(s)}{0}{1}{s} = 
    \integralx{ -\ddx{s} (1-s) g'(s) }{0}{1}{s}\\
    &= \left[ -(1-s) g'(s) \right]_0^1 + \integralx{(1-s)g''(s)}{0}{1}{s}\\
    &= g'(0) + \integralx{ (1-s)g''(s) }{0}{1}{s} \\    
    &= g'(0) + \integralx{(\ddx{s}(-\frac{1}{2}(1-s)^2))g''(s)}{0}{1}{s}\\
    &= g'(0) + \left[- \frac{1}{2}(1-s)^2 g''(s) \right]_0^1 
    + \frac{1}{2}\integralx{ (1-s)^2 g'''(s)}{0}{1}{s} \\
    &= g'(0) + \frac{1}{2} g''(0) + \frac{1}{2} 
    \integralx{(1-s)^2 g'''(s)}{0}{1}{s}.
\end{align*}
\[ (1-s)^2 = -\frac{1}{3} \ddx{s}((1-s)^2) \]
\begin{align*}
    \integralx{(1-s)^2 g'''(s)}{0}{1}{s}{s} &=
    -\frac{1}{3} \integralx{\ddx{s} ((1-s)^3)g'''(s)}{0}{1}{s}\\
    &= \underbrace{\left[-\frac{1}{3}(1-s)^3g'''(s)\right]_0^1 }_{=\frac{1}{3}g'''(0)}
    + \frac{1}{3} \integralx{(1-s)^3 g^{(4)}(s)}{0}{1}{s}
\end{align*}
Weitermachen mit Induktion. 
\[ g(1) = g(0) + g'(0) + \frac{1}{2}g''(0) + \frac{1}{2 \cdot 3} g'''(0)
+ \cdots + \frac{1}{(k-1)!} g^{(k-1)}(0) 
+ \frac{1}{(k-1)!} \integralx{(1-s)^{k-1} g^{(k)}(s) }{0}{1}{s} \]
\[ \integralx{(1-s)^{k-1} g^{(k)}(s) }{0}{1}{s} 
= \integralx{(1-s)^{k-1} g^{(k)}(s) - g^{(k)}() }{0}{1}{s} 
+ \underbrace{\integralx{ (1-s)^{k-1} }{0}{1}{s}}_{=\frac{1}{k}} g^{(k)}(0). \]
\[ \frac{1}{(k-1)!} \integralx{(1-s)^{k-1} g^{(k)}(s) }{0}{1}{s} 
= \frac{1}{k!} g^{(k)}(0) + 
\frac{1}{(k-1)!} \integralx{(1-s)^{k-1} (g^{(k)}(s) - g^{(k)}(0))}{0}{1}{s} \]
\begin{lem}
    Ist \( \abb{g}{[0,1]}{\R} (\text{ oder } \R^m) k\)-mal stetig differenzierbar.    
    \[ \Rightarrow g(1) = 
    g(0) + g^{(1)}(0) + \frac{1}{2} g^{(2)}(0) 
    + \frac{1}{2 \cdot 3}g^{(3)}(0) + \cdots 
    + \frac{1}{k!} g^{(k)}(0) + R_k(g) \]
    mit der Integraldarstellung des Restglieds
    \[ R_k(g) = \frac{1}{(k-1)!} \integralx{(1-s)^{k-1} 
    (g^{(k)}(s) - g^{(k)}(0)) }{0}{1}{s}. \] 
\end{lem}
\begin{kor}
    Sei \(\abb{f}{(a,b)}{\R} (\text{ oder } \R^m) \in \mathcal{C}^k \)\\
    \( x+h \in (a,b), x \in (a,b) \)
    \[ \Rightarrow f(x+h) = f(x) + f(x)h + \frac{1}{2}f^{(2)}h^2
    + \frac{1}{2 \cdot 3}f^{(3)}h^3 + \ldots + \frac{1}{k!}f^{(k)}h^k + \text{Rest} \]
    mit 
    \[ \text{Rest } = \frac{h^k}{(k-1)!} 
    \integralx{(1-s)^{k-1}(f^{(k)}(x + sh) - f^{(k)}(x))}{0}{1}{s} \]
\end{kor}
\begin{bew}
    Setzen \( g(t) = f(x + th) \). \( g'(t) = f'(x + th) h \), 
    \begin{align*}
        g'(t) &= f'(x+th)h \\
        g''(t) &= f''(x + th)h^2 \\
        g^{(l)}(t) &= f^{(l)}(x + th)h^l
    \end{align*}
    \[ \oversett{Lem. 1}{\Rightarrow} f(x+th) = g(1) = f(x) + f'(x)h
    + \frac{1}{2}f^{(2)}(x)h^2 + \ldots + \frac{1}{k!}f^{(k)}(x)h^k\]
    \[ + \frac{1}{(k-1)!} \integralx{(1-s)^{(k-1)} 
    (f^{(k)}(x+sh)h^k - f^{(k)}(x)h^k)}{0}{1}{s} \]
\end{bew}
Mehrere Dimensionen können verwirrend sein:
\( U \subset R^n \) offen, \(x \in U: h \in \R^n, 
\underbrace{[x,x+h]}_{\set{x+th, 0 \leq t \leq 1}} \subset U\)\\
\(f \in \mathcal{C}^k(U, \R)\)\\
\( \abb{g}{[0,1]}{\R} (\text{oder } \R^m), t \mapsto f(x+th) \)
wie oben folgt:
\[ \Rightarrow f(x+h) = g(0) + g'(0) + \frac{1}{2}g''(0) + \frac{1}{2 \cdot 3}g^{(3)}(0)
+ \ldots + \frac{1}{k!} \integralx{(1-s)^{k-1} (g^{(k)}(s)-g^{k}(0))}{0}{1}{s} \]
Die wichtige Frage ist: \( g^{(l)}(0) =  \) ?
\begin{align*}
    g'(0) &= \ddx{t} f(x + th) \vert_{t=0} \\
    &= D_h f(x) = Df(x)[h] \\
    &= \sum_{l=1}^n \partial_l f(x) h_l.
\end{align*}
\begin{align*}
    g'(t) &= \ddx{t} f(x + th) \\
    &= D_h f(x + th) \\
    &= Df(x + th)[h] \\
    &= \sum_{l=1}^n \partial_l f(x + th) h_l.
\end{align*}
Sei \( \partial_l f =: \tilde{f}_l \).
\( \Rightarrow g'(t) = \sum_{l=1}^n \tilde{f}_l (x + th) h_l \).
\[ \Rightarrow g''(t) = \ddx{t} g'(t) 
= \sum_{l=1}^n(\ddx{t} \tilde{f}_l(x+th))h_l
= D_h\tilde{f}_l(x+th) = D\tilde{f}_l(x+th)[h]
= \sum_{j=1}^n \underbrace{\partial_j \tilde{f}_l(x+th)}
_{= \partial_j\partial_lf(x+th)} h_j \]
\[ g''(t) = \sum_{j_2=1}^n \sum_{j_1=1}^n 
\partial_{j_2} \partial_{j_1} f(x + th) h_{j_1} h_{j_2}. \]
\[ \Rightarrow g'''(t) = \sum_{j_1,j_2 = 1}^n 
\ddx{t} \partial_{j_2} \partial_{j_1} f(x + th) h_{j_1} h_{j_2} \]
\[ = \sum_{j_3 = 1}^n\sum_{j_2 = 1}^n \sum_{j_1=1}^n \partial_{j_3} \partial_{j_2} \partial_{j_1}
f(x + th) h_{j_1} h_{j_2} h_{j_3}. \]
\[ g^{(k)}(t) = \sum_{j_k=1}^n \sum_{j_{k-1}=1}^n \ldots \sum_{j_{1}=1}^n 
\partial_{j_k}\partial_{j_{k-1}}\ldots\partial_{j_1} f(x+th)h_{j_1}h_{j_2}\ldots h_{j_k} \]
Beobachtung: 
\[ \partial_5 \partial_3 \partial_1f  = \partial_1 \partial_3 \partial_5 f
= \partial_3 \partial_5 \partial_1 f \text{ usw. (Schwarz)} \]
Multiindexnotation:
\[ N_0^n = \N_0 \times \ldots \times N_0 \text{ k-mal} \]
\( \alpha \in \N_0^n, \alpha =(\alpha_1, \ldots, \alpha_n), \alpha_j \in \N_0 \)
\[ \abs{\alpha} := \sum_{j=1}^n \alpha_j \]
\[ \alpha! := (\alpha_1!) \cdot (\alpha_1!) \cdots (\alpha_n!) \]
\[ h \in \R^n: h^\alpha := h_1^{\alpha_1} \cdots h_n^{\alpha_n} \text{ (Monom)} \]
\[ \partial^\alpha := \partial_1^{\alpha_1}\ldots \partial_n^{\alpha_n}
= \partial_n^{\alpha_n}\ldots\partial_1^{\alpha_1} \]
\[ h_j^0 = 1, \partial_j^0 = 1 \]
\begin{satz}[Taylor in Multiindexnotation]
    Sei \( U \subset \R^n \) offen, \( f \in \mathcal{C}^k(U), 
    x \in U, [x,x+h] \subset U, h \in \R^n \).
    \[ \Rightarrow f(x + h) = \sum_{\substack{\alpha \in \N_0^n \\ \abs{\alpha} \leq k}} 
    \frac{ \partial^\alpha f(x) }{\alpha!} h^\alpha 
    + R_k(f, x)(h) \]
    mit 
    \[R_k(f,x)(h) := \integralx{\frac{(1-t)^k}{(k-1)!} 
    \sum_{\abs{\alpha} = k} 
    (\partial^\alpha f(x+th)-\partial^\alpha f(x))h^\alpha}{0}{1}{t}\]   
\end{satz}
\begin{bew}[doch nicht so trivial]
    \[ g^{(k)}(t) = \sum_{j_1,\ldots, j_k = 1}^n 
    \partial_{j_k}\cdots \partial_{j_1} f(x+th)h_1^{j_1}\ldots h_k^{j_k} \]
    Anzahl der Tupel \( (j_1, \ldots, j_k) \) 
    in denen nach jeder der Koordinaten 
    \( x_\nu \) genau \( \alpha_\nu \) mal differenzieren, 
    \( \alpha = (\alpha_1, \ldots, \alpha_n) \) \\
    \[ \text{Anzahl } = \binom{k}{\alpha_1} \binom{k - \alpha_1}{\alpha_2} \cdots 
    \binom{k - (\alpha_1 + \cdots + \alpha_{n-1})}{\alpha_n}
    = \frac{ k! }{\alpha_1! \cdots \alpha_n!} 
    = \frac{k!}{\alpha!}. \]
    \[ \Rightarrow g^{(k)}(t) = \sum_{j_1,\ldots,j_k = 1}^n 
    \partial_{j_k}\cdots\partial_{j_1} f(x+th) 
    \underbrace{h_{j_1}\cdots h_{j_k}}_{h_1^{\alpha_1}\cdots h_n^{\alpha_n} = h^\alpha} \]
    \[ = \sum_{\substack{\alpha \in \N_0^n\\ \abs{\alpha}=k}} \frac{k!}{\alpha!}
    \partial^\alpha f(x+th) h^\alpha. \]
    \[ f(x + th) = \sum_{l=0}^k \frac{1}{k!} g^{(k)}(0) + \text{ Rest} \]
    \[ = \sum_{l=0}^k \frac{1}{\alpha!} \partial^\alpha f(x) h^\alpha \]
\end{bew}
Noch einmal Taylor: 
\( f \in \mathcal{C}^k(U, \R) \) (oder \( \R^m \)) \\
\(\abb{g}{[0,1]}{\R} (\text{auch }\R^m)\ k\)-mal stetig differenzierbar auf \((-\delta, 1+\delta)\)
\[ g(1) = g(0)+\integralx{g'(s)}{0}{1}{s} = \underbrace{\ldots}_{k\text{-mal partiell int.}} \]
\[ = g(0)+g'(0)+\frac{1}{2}g''(0) + \ldots + \frac{1}{k!} g^{(k)}(0) + \frac{1}{(k-1)!}
\integralx{(1-s)^{k-1} (g^{(k)}(s)-g^{(k)}(0))}{0}{1}{s} \]
Anwenden:
Setzen \( g(t) := f(x + th) \)
\[ \Rightarrow f(x+h) = \sum_{l=0}^k \frac{1}{l!} g^{(l)}(0) 
+ \frac{1}{(k-1)!} \integralx{(1-t)^{k-1} (g^{(k)}(s) - g^{(k)}(0)) }{0}{1}{s} \]
\[ g^{(l)}(0) = \frac{d^l}{dt^l}g(0) = \frac{d^l}{dt^l} f(x+th) \vert_{t=0} \]
\[ g^{(1)}(t) = \sum_{j=1}^n \partial_j f(x+th) h_j \]
\[ g^{(2)}(t) = \sum_{j=1}^n \ddx{t} (\partial_j f(x+th)) h_j 
= \sum_{j_1=1}^n \sum_{j_2 = 1}^n \partial_{j_2} 
\partial_{j_1} f(x + th) h_{j_2} h_{j_1} \]
\[ \Rightarrow \ldots g'^{(l)}(t) = \sum_{j_1=1,\ldots, j_l =1}^n 
\partial_{j_l} \ldots \partial_{j_1} f(x+th) h_{j_1}\cdots h_{j_l} \]
\[ \partial_j = \ddxpartial{x_j} \]
Multiindizes: \(\alpha \in \N_0^n, \alpha = (\alpha_1, \ldots, \alpha_n)\)\\
\( h \in \R^n, h^\alpha := h_1^{\alpha_1} \cdots h_n^{\alpha_n} \)\\
\( \partial^\alpha := \partial_1^{\alpha_1} \cdots \partial_n^{\alpha_n} \)\\
\[ \abs{\alpha} = \sum_{m=1}^n \alpha_m \]
Angenommen \(\abs{\alpha} = l\) und \(\sum_{m=1}^n \alpha_m = l\)
\[ = \sum_{\abs{\alpha} = l} \text{Faktor} \cdot \partial^\alpha f(x+th)h^\alpha \]
Wie mache ich diesen Faktor?\\
Anzahl der Tupel \((j_1, \ldots, j_n)\) (oder der Tupel \((\partial_{j_1}, \ldots, \partial_{j_n})\))
in denen nach jeder der Koordinaten \(x_j\) genau \(\alpha_j-\) mal partiell differenzierbar wird.  
\[ \binom{l}{\alpha_1}\binom{l-\alpha_1}{\alpha_2}\binom{l-\alpha_1-\alpha_2}{\alpha_3}
\cdot \binom{l-(\alpha_1+\ldots+\alpha_{n-1})}{\alpha_n} \]
\[ = \frac{l!}{\alpha_1!\cdots \alpha_n!} = \frac{l!}{\alpha!} \]
\[ \alpha! = \alpha_1!\cdots \alpha_n!  \]
\[ \Rightarrow g^{(l)}(0) = \sum_{\substack{\alpha \in \N_0^n \\ \abs{\alpha} = l}}
\frac{l!}{\alpha!} \partial^\alpha f(x) h^\alpha \]
\[ \Rightarrow f(x+h) = \sum_{l=0}^k \frac{1}{l!} g^{(l)}(0) 
+ \frac{1}{(k-1)!} \integralx{ (1-t)^{k-1} (g^{(k)}(s) - g^{(k)}(0)) }{0}{1}{s} \]
\[ = \sum_{l=0}^k \sum_{\substack{\alpha \in \N_0^n \\ \abs{\alpha} = l}}
\frac{\partial^l f(x)}{\alpha!}h^\alpha + \frac{1}{(k-1)!} k!
\sum_{l=0}^k \sum_{\substack{\alpha \in \N_0^n \\ \abs{\alpha} = l}}
\integralx{(1-t)^{k-1} (\partial^\alpha f(x+th)-\partial^\alpha f(x))(0))}{0}{1}{s} \]
\[ \Rightarrow f(x + h) = \sum_{\substack{\alpha \in \N_0^n \\ \abs{\alpha} \leq k}} 
\frac{ \partial^\alpha f(x) }{\alpha!} h^\alpha 
+ k \sum_{\abs{\alpha} = k} \frac{h^\alpha}{\alpha!} 
\integralx{(1-s)^{k-1} (\partial^\alpha f(x + sh) - \partial^\alpha f(x)) }{0}{1}{s} \]
\begin{bem}
    Anzahl der Multiindizes mit \(\abs{\alpha} = l\) ist 
    \[ \binom{l+n-1}{n-1} \]
\end{bem}
\end{document}