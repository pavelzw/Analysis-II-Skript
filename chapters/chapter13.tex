\documentclass[../ana2.tex]{subfiles}
\begin{document}
\setcounter{section}{12}
\section{Anwendungen}
\begin{notation}
    \( U \subset \R^n \) offen, 
    dann schreiben wir \(f \in \mathcal{C}^1(U, \R^m)\), 
    falls alle partiellen Ableitungen \( \abb{\partial_jf}{U}{\R^m}, 
    j = 1,\ldots, n \) stetig sind.\\
    \( \oversett{Satz 12.4}{\Leftrightarrow} 
    \abb{Df}{U}{\mathscr{L}(\R^n, \R^m)} \) 
    ist stetig (auf \(U\))
\end{notation}
Erinnerung: HDI: \\
Ist \( \abb{f}{[a,b]}{\R} \) stetig und stetig differenzierbar
auf \((a,b)\).
\[ \Rightarrow f(b)-f(a) = \integralx{f'(t)}{a}{b}{t} \]
Dasselbe gilt, wenn \(f\) stückweise \(\mathcal{C}^1\)
auf \([a,b]\), d.h. \(f\) ist stetig auf \([a,b]\) und
\( \exists N \in \N, a = t_1 < t_2 < \ldots < t_N = b \),
sodass \( \abb{f}{(t_j, t_{j+1})}{\R} \) ist stetig 
differenzierbar.\\
\( \Rightarrow f(b)- f(a) = \sum_{j=2}^N (f(t_j) - f(t_{j-1}))
= \sum_{j=2}^N \integralx{f'(s)}{t_{j-1}}{t_j}{s} 
=: \integralx{f'(s)}{a}{b}{s} \)\\
Verallgemeinerung auf mehrere Dimensionen
\begin{lem}
    Sei \(U \subset \R^n\) offen, \( f \in \mathcal{C}^1(U, \R^m) \) 
    und \( \abb{\gamma}{[a,b]}{U} \) stückweise \(\mathcal{C}^1\)
    \[ \Rightarrow f(\gamma(b)) - f(\gamma(a)) 
    = \integralx{Df(\gamma(t))[\dot{\gamma}(t)]}{a}{b}{t} \]
    mit \(\dot{\gamma}(t) = \ddx{t} \gamma(t)\)
\end{lem}
\begin{bew}
    Sei \( a = t_1 < t_2 < \dots < t_N = b \) Zerlegung von 
    \( [a,b] \) mit \( \abb{\gamma}{(t_j, t_{j+1})}{U} \) 
    ist \( \mathcal{C}^1 \)
    \[ \Rightarrow D \gamma(t)[1] = \ddx{t} \gamma(t) 
    = \dot{\gamma}(t) \]
    ist stetig auf \( [t_j, t_{j+1}] \)
    \[ \oversett{Kettenregel}{=} D(f\circ \gamma)(t)
    = Df(\gamma(t))D\gamma(t) \]
    d. h. 
    \[ \ddx{t} f(\gamma(t)) = D(f\circ\gamma)(t)[1] 
    = Df(\gamma(t))[D\gamma(t)[1]] = (**) \] 

    \( = Df(\gamma(t))[\dot{\gamma}(t)] \) ist stetig auf 
    \( (t_j t_{j+1}) \)
    \begin{align*}
        &\oversett{HDI}{\Rightarrow} f(\gamma(t_{j+1})) 
        - f(\gamma(t_j)) \\
        &= \integralx{\ddx{s}f(\gamma(s))}{t_j}{t_{j+1}}{s} \\
        &= \integralx{D f(\gamma(s))
        [\dot{\gamma}(s)]}{t_j}{t_{j+1}}{s} 
    \end{align*}
    \begin{align*}
        &\Rightarrow f(\gamma(b)) - f(\gamma(a)) \\
        &= \sum_{j=1}^{N-1} f(t_{j+1}) - f(t_j) \\
        &= \sum_{j=1}^{N-1} \integralx{Df(\gamma(s))[\dot{\gamma}(s)]}{t_j}{t_{j+1}}{s} \\
        &= \integralx{Df(\gamma(s))[\dot{\gamma}(s)]}{a}{b}{s}.
    \end{align*}
\end{bew}
\begin{bem}
    Im Beweis kommen Integrale einer vektorwertigen Funktion
    vor. Man macht entweder Integrationstherorie für Vektorfunktionen
    oder erklärt es komponentenweise.\\
    Ist \( v \in \mathcal{C}([a,b], \R^m), 
    b_1, \ldots, b_m \) Basis von \( \R^m \).
    \[ v(t) = \sum_{j=1}^m v_j(t) b_j. \]

    \[ \Rightarrow v_j \in \mathcal{C}([a,b], \R) \]
    und
    \[ \integralx{v(s)}{a}{b}{s} 
    = \sum_{j=1}^m (\integralx{v_j(s)}{a}{b}{s}b_j) \]
\end{bem}
\begin{defi}
    Ein metrischer Raum \(X\) heißt wegzusammenhängend, falls 
    es zu je zwei Punkten \( x_0, x_1 \in X \) eine stetige 
    Funktion \( \abb{\varphi}{[0,1]}{X} \) gibt mit 
    \( \varphi(0) = x_0, \varphi(1) = x_1 \).\\
    Da stetige Wege kompliziert sein können, z. B. 
    kann die Peano-Kurve ein Quadrat überdecken (Peano 1890).\\
    Deswegen benötigen wir stückweise stetige \( \mathcal{C}^1 \) 
    Wege.
\end{defi}
Bild: 
%Bild

In der Vorlesungs wurde die folgende Kette von Implikationen bewiesen:
\begin{align*}
    &\text{stetig partiell differenzierbar} \\
    &\Rightarrow \text{ Differenzierbarkeit} \\
    &\Rightarrow \text{ partiell differenzierbar}
\end{align*}
Die Umkehrung gilt im Allgemeinen nicht.
\begin{lem}
    Sei \(U \subset \R^n \) offen und wegzusammenhängend. 
    Dann gibt es zu \(x_0, x_1 \in U\) einen Polygonzug, d. h. 
    ein stückweise linearen Weg \(\abb{\gamma}{[0,1]}{U}\) mit
    \(\gamma(0) = x_0, \gamma(1) = x_1\).
\end{lem}
\begin{bew}
    Sei \(\abb{\phi}{[0,1]}{U}\) stetig mit \(\phi(0) = x_0\)
    und \(\phi(1) = x_1\). Da \(\phi\) stetig und \([0,1]\)
    kompakt ist, ist \( K:= \set{\phi(t): 0 \leq t \leq 1} \subset U \subset \R^n \)
    eine kompakte Teilmenge von \(\R^n\). Ausserdem ist 
    \(A := U^C = \R^n \setminus U\) abgeschlossen und somit 
    \(r := \dist(K, A) = 
    \underset{\substack{x \in U \\ y \in A}}{\inf} 
    \abs{x-y} > 0\). \\
    Denn angenommen \( r = 0 \), so existieren 
    Folgen \( (x_n)_n \subset K, (y_n)_n \subset A \)
    mit \( \abs{x_n - y_n} \rightarrow 0 \).
    Da \(K\) kompakt ist, existiert eine konvergente Teilfolge
    \( (x_{n_k})_k, \tilde{x} := \limes{k} x_{n_k} \in K \subset U \)    
    \( \Rightarrow \abs{ y_{n_K} - \tilde{x} }
    \leq \abs{ y_{n_k} - x_{n_k} } 
    + \abs{ x_{n_k} - \tilde{x} } \rightarrow 0 \).
    Das heißt \( y_{n_k} \rightarrow \tilde{x} \) 
    für \(k \rightarrow \infty \). Da 
    \( y_{n_k} \in A = U^C \) abgeschlossen
    \( \Rightarrow \tilde{x} \in A \text{ \Lightning} \) 
    zu \(\tilde{x} \in K \subset U\). 
    Also ist \(r > 0\).\\
    Sei \( N\in\N \). Betrachte Zerlegung
    \( t_j = \frac{j}{N} \) von \( [0,1] \).
    \( j = 0,1,\ldots,N \) und definieren
    \[ \gamma_N(t) = \frac{t_k -t}{t_k-t_{k-1}} \varphi(t_{k-1}) 
    + \frac{t-t_{k-1}}{t_k - t_{k-1}} \varphi(t_k) \]
    für \( t \in [t_{k-1}, t_k] \).\\
    \(\Rightarrow\) Polygonzug \(\abb{\gamma_N}{[0,1]}{\R^n}\)
    \begin{beh}
        \( \gamma_N([0,1]) \subset U \), 
        wenn \(N\) groß genug ist.
    \end{beh}
    \begin{bew}\renewcommand{\qedsymbol}{}
        \(\abb{\varphi}{[0,1]}{U}\) ist stetig, 
        \( [0,1] \) ist kompakt \( \Rightarrow 
        \varphi \) ist gleichmäßig stetig auf 
        \( [0,1] \).
        Setze \(\mathrm{osz}(\varphi, \delta) := \sup 
        \set{\abs{\varphi(t)- \varphi(\delta)} \vert s,t \in [0,1], \abs{t-s} \leq \delta}\)
        die \(\delta\)-Oszillation von \( \varphi \).\\
        \( \varphi \) gleichmäßig stetig 
        \( \Leftrightarrow \limesx{\delta}{0+} 
        \mathrm{osz}(\varphi, \delta) = 0 \).
        Insbesondere \(\exists \delta > 0\) mit
        \[ \mathrm{osz}(\varphi, \delta) \leq \frac{r}{2} \]
        Wähle \(N \in \N\) so, dass \(\frac{1}{N} < \delta\).\\
        \( \Rightarrow \) Auf dem Intervall 
        \( [t_{k-1}, t_k] \) ist
        \begin{align*}
            \gamma_N(t) - \varphi(t) 
            &= \frac{t_{k} - t}{t_k - t_{k-1}}\varphi(t_{k-1})
            + \frac{ t - t_k }{ t_k - t_{k-1} }\varphi(t_k) 
            - \varphi(t) \\
            &= \frac{ t_k - t }{t_k - t_{k-1}} (\varphi(t_{k-1}) - \varphi(t)) 
            + \frac{ t - t_{k-1} }{t_k - t_{k-1}}(\varphi(t_k) - \varphi(t_{k-1}))
        \end{align*}
        
        \[ \Rightarrow \abs{\gamma_N(t) - \varphi(t)} 
        \leq \mathrm{osz}(\varphi, \frac{1}{N})
        \leq \mathrm{osz}(\varphi, \delta) < \frac{r}{2} 
        \; \forall 0 \leq t \leq 1 \]
        \[ \Rightarrow \underset{t \in [0,1]}{\sup} 
        \abs{\gamma_N(t)-\varphi(t)} \leq \frac{r}{2} \]
        d. h. \( \gamma_N([0,1]) \) ist in einem \(\frac{r}{2}\)-Schlauch um
        \(\varphi([0,1]) = K\) und da \(\dist(K,A) = r > 0\) folgt
        \[ \gamma_N([0,1]) \subset U. \]        
    \end{bew}
\end{bew}
\begin{satz}
    Sei \(U \subset \R^n\) offen und wegzusammenhängend,
    \( \abb{f}{U}{\R^m} \) differenzierbar. Dann gilt
    \[ Df(x) = 0 \; \forall x \in U \Rightarrow f \text{ ist konstant} \]
\end{satz}
\begin{bew}
    Seien \(x_0, x_1 \in U\). Nach Lemma 3 existiert ein Polygonzug
    \( \abb{\gamma}{[0,1]}{U} \) mit \( \gamma(0) = x_0 \), 
    \( \gamma(1) = x_1 \).
    Da \( \gamma \) stückweise \(\mathcal{C}^1\) ist und \(f\) stetig
    differenzierbar ist, folgt aus Lemma 1 
    \[ f(x_1)-f(x_0) = \integralx{\underbrace{Df(\gamma(t)[\gamma(t)]}_{=\; 0}}{a}{b}{t} 
    = 0 \in \R^m. \]
    \( \Rightarrow f(x_1) = f(x_0) \). Da \( x_0, x_1 \) beliegbig sind, 
    folgt die Behauptung.
\end{bew}
\subsection{Zweite Ableitung}
%\(U \subset \R^n\) offen, \(\abb{f}{U}{\R^m}\) differenzierbar.
%\[ \Rightarrow \abb{Df}{U}{
%    \underbrace{\mathscr{L}(\R^n, \R^m)}_{\text{ist wieder VR } W}} \]
%\( W \) ist ein normierter Vektorraum und sei 
%\( \abb{f}{U}{W} \) ist differenzierbar
%\[ \Rightarrow \abb{Df}{U}{\underbrace{\mathscr{L}(\R^n, W)}_{= \mathscr{L}(\R^n, 
%\mathscr{L}(\R^n, \R^m))} =: \tilde{W}}. \]
%Was ist das? \\
Sei \(U \subset \R^n\) offen, \(W\) beliegbiger Vektorraum, \(\abb{f}{U}{W}\)
differenzierbar. \\
\[ \Rightarrow \abb{Df}{U}{\mathscr{L}(\R^n, W)} =: W \text{ wieder ein VR} \]
Frage: Ist \(\abb{f}{U}{\tilde{W}}\) wieder differenzierbar und falls ja,
ist die zweite Ableitung von \(f\) dann
\[ \abb{D^2f = D(Df)}{U}{\mathscr{L}(\R^n, \tilde{W})}  \]
Was dieses Objekt ist, wird in einem Handout diskutiert.\\
Im Folgenden der einfachere Fall von Richtungsableitungen.
Ist \( \abb{f}{U}{\R^m} \) und \( h\in \R^n \) (\(U \subset \R^n\) offen)
\[ \Rightarrow D_h f(x) = \limesx{t}{0} 
\frac{ f(x + th) - f(x) }{t} = Df(x)[h] \]
falls \(f\) differenzierbar ist.
Satz 12.4 sagt: \\
\(f\) ist stetig differenzierbar. \\
\(\Leftrightarrow\) alle partiellen Ableitungen 
\(\abb{\partial_j f}{U}{\R^m}\) sind stetig auf \(U\). \\
\(\Leftrightarrow \forall n \in \R^n\) sind 
alle Richtungsableitungen stetig auf \(U\).
Also \( \abb{f}{U}{\R^m}, h_1, h_2 \in \R^n \) und 
Richtungsableitungen \( \abb{D_{h_1}f}{U}{\R^m} \) 
und \( \abb{D_{h_2}f}{U}{\R^m} \) existieren.
Dann können wir die Richtungsableitungen von \(D_{h_1}f\) in Richtung
\(h_2\) und von \(D_{h_2}f\) in Richtung \(h_1\) bilden.
Also angenommen \( D_{h_1}(D_{h_2}f)(x) \) und \( D_{h_2}(D_{h_1}f(x) \)
existieren. Was haben sie dann miteinander zu tun? \\
Z. B. für partielle Ableitungen 
\begin{align*}
    \partial_j \partial_k f(x) &= D_{e_j}D_{e_k} f(x) \\
    \partial_k \partial_j f(x) &= D_{e_k} D_{e_j} f(x)
\end{align*}
\begin{satz}[Einfache Version des Satzes von Schwarz]
    Sei \( U \subset \R^n \) offen, \( \abb{f}{U}{\R^m}, 
    h_1, h_2 \in \R^n \). Ferner existieren die 
    Richtungsableitungen \( \abb{D_{h_1}f, D_{h_2}f}{U}{\R^m} \) 
    und die iterierten Richtungsableitungen 
    \( \abb{D_{h_1}D_{h_2}f}{U}{\R^m} \) und sie seinen stetig
    auf \(U\).
    \[ \Rightarrow D_{h_1}D_{h_2} f(x) = D_{h_2}D_{h_1} f(x) 
    \; \forall x \in U \]
    Das heißt die iterierten Richtungsableitungen sind symmetrisch.
\end{satz}
\begin{bew}
    Sei \(x \in U\) und \(\delta > 0\), sodass 
    \[ x + t_1 h_1 + t_2h_2 \; \forall t_1, t_2 \in [0, \delta] \]    
    Sei \( 0 \leq t \leq \delta \) und 
    \[ \varphi(s) := f(x + s h_1 + t h_2) - f(x + s h_1). \]
    \begin{align*}
        \Rightarrow \ddxpartial{s} \varphi(s) 
        &= D_{h_1} f( x + sh_1 )- D_{h_1}f(x + s h_1) \\
        &\oversett{HDI}{=} \integralx{\ddxpartial{u} 
        D_{h_1}f(x+s h_1 + u h_2) }{0}{t}{u} \\
        &= \integralx{ D_{h_2} D_{h_1}f(x+s h_1 + u h_2) }{0}{t}{u} 
    \end{align*}
    \begin{align*}
        \Rightarrow \varphi(t)-\varphi(0) &= f(x+th_1+th_2)-f(x+th_1)
        -f(x+th_2) + f(x) \\
        &= \integralx{ \ddxpartial{s} \varphi(s)}{0}{t}{s}\\
        &= \integralx{
        \integralx{D_{h_1} D_{h_1} f(x + s h_1 + u h_2)}{0}{t}{u}
        }{0}{t}{s}\\
        &= \integralx{
            \integralx{D_{h_1} D_{h_1} f(x + s h_1 + u h_2) - D_{h_2} D_{h_1}f(x)}{0}{t}{u}
        }{0}{t}{s} + D_{h_2} D_{h_1} f(x) t^2.
    \end{align*}
    Durch Vertauschen von \(h_1\) und \(h_2\) erhalten wir 
    (da \(\varphi(t)-\varphi(0)\) symmetrisch in \(h_1\) und \(h_2\) ist)
    \[ \varphi(t)-\varphi(0) = \integralx{
        \integralx{D_{h_1} D_{h_2} f(x + s h_2 + u h_1) - D_{h_1} D_{h_2}f(x)}{0}{t}{u}
    }{0}{t}{s} + D_{h_1} D_{h_2} f(x) t^2 \]
    Da \( U \in z \mapsto D_{h_2} D_{h_1} f(z) \) stetig ist, ist 
    \[ [0,\delta]\times [0,\delta] = [0,\delta]^2 \ni (t_1, t_2) 
    \mapsto D_{h_2} D_{h_1} f(x + t_1 h_1 + t_2 h_2) \] 
    stetig, sagen gleichmäßig stetig, da \( [0,\delta]^2 \) kompakt ist.
    \( \Rightarrow \forall \varepsilon > 0 \; \exists 0 < \eta_1 < \delta \) mit
    \[ \abs{D_{h_2}D_{h_1} f(x+sh_1 + uh_2) - D_{h_2}D_{h_1}f(x)} \leq \varepsilon
    \; \forall 0 < U, s < \eta_1 \]    
    \[ \Rightarrow \forall 0 < t \leq \eta_1 \text{ gilt} \]
    \[ A := \integralx{
        \integralx{\abs{ D_{h_2} D_{h_1} f(x + s h_1 + u h_2) 
        - D_{h_2} D_{h_1} f(x) }}{0}{t}{u}
    }{0}{t}{s} \leq \frac{\varepsilon}{2}t^2 \]
    Genauso \(\exists 0 < \eta_2 < \delta \) mit \(\forall 0 < t \leq \eta_2\)
    \[ B := \integralx{
        \integralx{\abs{ D_{h_1} D_{h_2} f(x + s h_1 + u h_2) 
        - D_{h_1} D_{h_2} f(x) }}{0}{t}{u}
    }{0}{t}{s} \leq \frac{\varepsilon}{2}t^2 \]
    \[ t^2(D_{h_2}D_{h_1}f(x)-D_{h_1}D_{h_2}f(x)) = B - A \]
    \[ \Rightarrow t^2 \abs{ D_{h_2} D_{h_1} f(x) - D_{h_1} D_{h_2} f(x) } \leq \varepsilon t^2 \]
    \[ \Rightarrow \abs{ D_{h_2} D_{h_1} f(x) - D_{h_1} D_{h_2} f(x) } \leq \varepsilon \]
    \(\Rightarrow\) da \(\varepsilon > 0\) beliebig ist, folgt 
    \[ D_{h_2}D_{h_1}f(x) = D_{h_1}D_{h_2}f(x). \]
\end{bew}
\end{document} 