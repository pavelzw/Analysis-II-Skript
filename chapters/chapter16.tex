\documentclass[../ana2.tex]{subfiles}

\begin{document}
\setcounter{section}{15}
\section{Banachscher Fixpunktsatz}
\( \abb{T}{M}{M} \) ist \(x\in M\) Fixpunkt (von \(T\)),
falls \(x = T(x)\)
\begin{defi}
    Sei \(M\) ein metrischer Raum mit 
    Metrik \(d\).\\
    \( \abb{T}{M}{M}\) ist eine \( \alpha \)-Kontraktion, 
    falls \( d(T(x), T(y)) \leq \alpha d(x, y) \;\forall x, y \in M \).
\end{defi}
\begin{satz}[Banachscher Fixpunktsatz]
    Sei \( M \) ein vollständiger metrischer Raum, \( \abb{T}{M}{M} \) 
    eine \( \alpha \)-Kontraktion (\( 0 \leq \alpha < 1 \)). Dann gibt 
    es genau einen Fixpunkt \( x^* \) und man nehme irgendein \( x \in M \) 
    \[ \Rightarrow x^* = \limes{n} T^n(x). \]
    mit \( T^1(x) = T(x), T^{n+1}(x) = T(T^n(x)) \)
    und a priori Fehlerabschätzung: 
    \[ d(x^*, T^n(x)) \leq \frac{1}{1-\alpha} \alpha^n d(x, T(x)). \]
\end{satz}
\begin{bew}
    1. Höchstens einen Fixpunkt:\\
    Seien \( x_1^* = T(x_1^*), x_2^* = T(x_2^*) \) Fixpunkte. 
    \[ \Rightarrow d(x_1^*, x_2^*) = d(T(x_1^*),T(x_2^*)) 
    \leq \alpha d(x_1^*, x_2^*). \]
    \[ \Rightarrow (1-\alpha) d(x_1^*, x_2^*) = 0 
    \Rightarrow d(x_1^*, x_2^*) = 0 \Rightarrow x_1^* = x_2^*. \]
    2. Es gibt einen Fixpunkt:\\
    \(T\) ist Lipschitzstetig mit Konstante \(\alpha 
    \Rightarrow T\) ist stetig.\\
    Nimm \( x \in M \). \( x_0 = x, x_1 = T(x), x_{n+1} = T(T^n(x)) 
    = T(x_n) \).
    \(\Rightarrow\) Folge \((x_n)_n\) in \(M\).
    Angenommen: Die Folge konvergiert, d. h. \( x^* := \limes{n} x_n \in M \) 
    existiert.\\
    \( x^* \leftarrow x_{n+1} = T(x_n) \Rightarrow x^* = x_{n+1} 
    = \limes{n} T(x_n) = T(\limes{n}x_n) = T(x^*) \).\\
    \( M \) vollständig. \( \Rightarrow (x_n)_n \) konvergiert 
    \( \Leftrightarrow (x_n)_n \) Cauchyfolge.
    \[ d(x_n, x_m) \leq d(x_n, x_{n+1}) + d(x_{n+1}, x_m) 
    \leq d(x_n, x_{n+1}) + d(x_{n+1}, x_{m+1}) 
    + d(x_{m+1}, x_m) \]
    \[ \Rightarrow (1-\alpha) d(x_n, x_m) 
    \leq d(x_{n+1}, x_n) + d(x_{m+1}, x_m) \]
    \[ \Rightarrow d(x_n, x_m) 
    \leq \frac{1}{1-\alpha} (d(x_{n+1}, x_n) + d(x_{m+1}, x_m)) \]   
    \( \Rightarrow \) Brauche obere Schranke von 
    \[d(x_{n+1}, x_n) = d(T(x_n), T(x_{n-1})) 
    \leq \alpha d(x_n, x_{n-1}) \]
    \[ \leq \alpha^2 d(x_{n-1}, x_{n-2}) 
    = d(T(x_{n-1}), T(x_{n-2})) \]
    \[ \leq \alpha^n d(x_1, x_0) = \alpha^n d(T(x), x) \]
    \[ \Rightarrow d(x_n, x_m) \leq \frac{1}{1-\alpha}(\alpha^n + \alpha^m)d(T(x), x) \]
    Wähle \( N \), sodass \( \frac{1}{1-\alpha}\alpha^N d(T(x), x) < \forall{\varepsilon}{2} \)
    \[ \Rightarrow \frac{1}{1-\alpha} \alpha^n d(T(x), x) < \frac{\varepsilon}{2} \; \forall n \geq N. \]    
\end{bew}
\subsection*{Fehlerabschätzung}
Sei \( x^* = T(x^*) \). 
\begin{align*}
    \Rightarrow d(x^*, x_{n+1}) &= d(T(x^*), T(x_n)) \\
    &\leq \alpha d(x^*, x_n) \\
    &\leq \alpha^2 d(x^*, x_{n-1}) \\
    &\leq \ldots \leq \alpha^{n+1} \underbrace{d(x^*, x_0)}_{\text{nicht bekannt}}
\end{align*}
Etwas anders: 
\begin{align*}
    d(x^*, x_n) &\leq d(x^*, x_{n+1}) + d(x_{n+1}, x_n) \\
    &= d(T(x^*), T(x_n)) + d(x_{n+1}, x_n)\\
    &\leq \alpha d(x^*, x_n) + d(x_{n+1}, x_n).
\end{align*}
Da \(1- \alpha > 0\) ist, folgt
\[ d(x^*, x_n) \leq \frac{1}{1-\alpha} d(x_{n+1}, x_n) \]
A-posteriori Fehlerabschätzung.
Da
\begin{align*}
    d(x_{n+1}, x_n) *\leq \alpha d(x_1, x_0) \\
    &\leq \alpha^n d(T(x_0), x_0),
\end{align*}
folgt 
\[ d(x^*, x_n) \leq \frac{\alpha^n}{1-\alpha} d(T(x_0),x_0) \]
A-priori Fehlerabschätzung.
\begin{bem}
    Manchmal muss man \(\abb{T}{M}{M}\) auf die Teilmenge \(A \subset M\)
    einschränken, um zu erreichen, das \(T\) eine Kontraktion ist. 
    Aber es muss nicht gelten 
    \[ \abb{T}{A}{A} \]    
\end{bem}
Frage: Ist \(a \in M\) und \(\overline{B_r}(a) = \set{x \in M: d(x,a) \leq r}\)
Wann ist dann 
\[ \abb{T}{\overline{B_r}(a)}{\overline{B_r}(a)} \text{?}\]
Brauchen: \(\forall x \in \overline{B}_r(a) \) ist \(T(x) \in \overline{B}_r(a)\),
d. h. \(d(T(x), a) \leq r\).
\[ d(T(x), a) \leq \underbrace{d(T(x), T(a))}_{
    \substack{\leq \alpha d(x, a) \\ \leq \alpha r}} + d(T(a), a) \]
Also falls \( d(T(a), a) \leq (1 - \alpha)r \), folgt 
\[ d(T(x), a) \leq \alpha r + (1-\alpha) r \leq r \]
für alle \( x\in B_r(a) \).
\subsection*{Anwendung}
Sei \( \abb{f}{\R^m}{\R^n} \). Gibt es ein \( x^*: f(x^*) = 0 \)?\\
Angenommen \( x \in \R: f(x) \neq 0 \). Wir suchen \( h\in\R^n \) 
mit 
\[ 0 \overset{!}{=} f(x+h) = f(x) + Df(x)[h] + o(\abs{h}) \]
\[ \Rightarrow A(x)h = -f(x)  + o(\abs{h})\]
Ist \(A\) invertierbare \(n \times n\)-Matrix, so folgt
\[ h = -A(x)^{-1} f(x) + o(\abs{h}) \]
\( \Rightarrow \) neuer Punkt 
\[ x + h = x - A(x)^{-1} f(x) =: T(x) \]
Iteriere. Sei \( x_0 \in \R^n, x_{n+1} := T(x_n) = x_n - A(x_0)^{-1} f(x_n) \).\\
Brauchen \( A(x_n) = \abb{Df(x_n)}{\R^n}{\R^n} \) ist invertierbar.
\subsection*{Vereinfachtes Newton-Verfahren}
Sei \(A\) irgendwie invertierbare lineare Abbildung von 
\( \R^n \rightarrow \R^n \).\\
Setze \( T(x) := x - A^{-1} f(x) \) \((*)\) 
und nehmen \( x_0 \in \R^n, x_{n+1} := T(x_n), n\in \N_0 \).
\begin{satz}
    Genügt die Funktion \(T\) einer Lipschitzbedingung
    in einer offenen Kugel \( B_r(a) \) mit 
    Lipschitzkonstante \( \alpha = \frac{1}{2} \) 
    und ist \(\abs{A^{-1} f(a)} < \frac{r}{2}\), so hat \(f\)
    in \(B_r(a)\) genau eine Nullstelle \(\xi\).
\end{satz}
Nehmen \( x_0 \in B_r(a) \) und induktiv 
\( x_{n+1} = T(x_n) \), dann ist für alle \( n \in \N_0 \ x_n \in B_r(a) \) 
und \( \xi = \limes{n} x_n \) erfüllt \( f(\xi) = 0 \).
\begin{bew}
    Wir wenden Kontraktionsprinzip auf \( \abb{T}{\overline{B_s}(a)}{\overline{B_s}(a)} \) 
    für ein \( 0 < s < r \), sodass \( \abs{x_0 - a} \leq s \) 
    und \( \abs{ T(a) - a } = \abs{ A^{-1} f(a) \leq \frac{1}{2}s } \).
    Dann ist für alle \(x \in \overline{B}_s(a)\) auch
    \begin{align*}
        \abs{T(x)-a} \leq \abs{T(x)-T(a)} + \abs{T(a)-a}\\
        \leq \frac{1}{2} \abs{x-a} + \frac{1}{2}s \leq s
    \end{align*}
    \[ \Rightarrow T(x) \in \overline{B}_s(a). \]
\end{bew}
\end{document}