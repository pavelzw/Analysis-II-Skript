
\documentclass[../ana2.tex]{subfiles}
\begin{document}
\setcounter{section}{8}
\section{Topologie des \( \R^d (\C^d, \ldots) \)}
Frage: Abstand, Konvergenz in \gqq{komplizierteren} 
Räumen.
Z. B. 
\[ \R^d = \set{x = (x_1, \ldots, x_d) : 
x_j \in \R \forall 1 \leq j \leq d} \]
\[ \C^d = \set{z = (z_1, \ldots, z_d) : 
z_j \in \R \forall 1 \leq j \leq d} \]
\begin{defi}[Norm auf einen Vektorraum]
    Eine Norm auf enen reellen oder komplexen Vektorraum
    ist eine Funktion: \( \abb{\norm{\cdot}}{X}{\R} \) mit
    \begin{description}
        \item[Positivität] \( \norm{x} \geq 0 \forall x\in X \) 
        und \( \norm{x} = 0 \Leftrightarrow x = 0 \).
        \item[Homogenität] \( \norm{\lambda x} = \abs{\lambda} \norm{x}
        \; \forall \lambda \in \R (\C), x\in X \).
        \item[\( \Delta \)-Ungleichung] \( \norm{x+y}
        \leq \norm{x} + \norm{y} \; \forall x,y \in X \)
    \end{description}
\end{defi}
\begin{bspe}
    \[ \norm{x}_1 = \sum_{j=1}^d \abs{x_j} \]
    \[ \norm{x}_\infty 
    = \underset{1 \leq j \leq d}{\max} \abs{x_j} \]
\end{bspe}
\begin{defi}[Metrik und metrischer Raum]
    Menge \(X\) mit einer Funktion \( \abb{d}{X\times X}{\R} \)
    mit den Eigenschaften
    \begin{description}
        \item[Positivität] \( d(x,y) \geq 0, d(x,y) = 0 \Leftrightarrow x=y
        \; \forall x,y \in X \)       
        \item[Symmetrie] \( d(x,y) = d(y,x) \; \forall x,y \in X \)
        \item[\( \Delta \)-Ungleichung] 
        \( d(x,y) \leq d(x,z) + d(z, y) 
        \; \forall x,y,z \in X \).
    \end{description}
    \(d\) heißt Metrik und \( (X, d) \) heißt metrischer 
    Raum.
\end{defi}
\begin{bsp}
    \( \R^d \) oder \( \C^d\) mit \(\norm{\cdot}_2\)
    \[ d(x,y) := \norm{x-y}_2 \]
    oder ein allgemeiner Vektorraum \(X\) mit Norm 
    \(\norm{\cdot}\)
    \[ d(x,y) := \norm{x-y} \]
    Z. B. 
    \begin{align*}
        d(y,x) &= \norm{y - x} = \norm{-(x-y)} \\
        &= \abs{-1}\norm{x-y} = \norm{x-y} = d(x, y).
    \end{align*}
    \begin{align*}
        d(x,y) &= \norm{x-y} = \norm{x-z + (z-y)} \\
        &\leq \norm{x-z} + \norm{z-y} \\
        &= d(x,z) + d(z,y)
    \end{align*}
    \( \R^d \) mit \( \norm{\cdot}_2 \) ist ein metrischer 
    Raum.
\end{bsp}
\begin{bsp}[beschränkte Funktionen von \( X \) nach \( \R \)]
    \[ L^\infty(x) := \set{\abb{f}{X}{\R} 
    \; \vert \; \underset{x\in X}{\sup} \abs{f(x)} < \infty} \]
    \[ d_\infty(f, g) := \underset{x\in X}{\sup} 
    \abs{ f(x) - g(x) } = \norm{f-g}_\infty \]
    Angenommen: \(X\) ist ein metrischer Raum, 
    \( \abb{d}{X \times X}{\R} \). \\
    Nehme \( a \in X \), setze 
    \( f_x(y) := -d(y,a) + d(x,y), y \in X \) \\
    \( \Rightarrow \) Die Abbildung \( X \rightarrow L^\infty \)
    \( x \mapsto f_x \) (\( f_x(y) := 
    -d(y,a) + d(x,y), a \in X \))
    ist injektiv.
\end{bsp}
\begin{bsp}
    Norm \( \norm{\cdot} \) auf VR \( V, 
    d(x,y) := \norm{x-y} \) ist Metrik auf 
    \(V\).
\end{bsp}
\begin{defi}
    Sei \(X\) ein metrischer Raum mit Metrik \(d\).
    \[ x_0 \in X, r > 0 \quad B_r(x_0) 
    := \set{ x\in X: d(x, x_0) < r } \] 
    ist die offnee Kugel mit Radius \(r\) um \(x_0\).
    \[ \R^d: B_r(x_0) = \set{x \in \R^d: 
    \norm{x - x_0}_2 < r}. \]
\end{defi}
\begin{defi}[offene Mengen]
    Sei \( (X,d) \) ein metrischer Raum. Eine Menge 
    \( U \subset X \) heißt offen, falls 
    \[ \forall x\in U \exists r = r_x > 0 \]
    mit 
    \[ B_r(x) \subset U. \]
\end{defi}
\begin{bsp}
    \( B_r(x_0) \) ist offen: 
    \( x \in B_r(x_0) \).
    \( \varepsilon := r - d(x, x_0) \)
    Ist \( y \in B_\varepsilon(x): d(y, x_0) 
    \leq d(y, x) + d(x, x_0) < \varepsilon 
    + d(x, x_0) = r \).
\end{bsp}
\begin{satz}[Topologie]
    Man nehme metrischen Raum \( (X, d) \).
    Die Menge aller offenen Teilmengen von \(X\) 
    ist eine Topologie, das heißt
    \begin{enumerate}[label=(\alph*)]
        \item \( \emptyset, X \) sind offen.
        \item Der Durchschnitt endlich vieler
        offener Mengen ist offen.
        \item Die Vereinigung beliebig vielen offenen
        Mengen ist offen. 
    \end{enumerate}    
\end{satz}
\begin{bew}
    Etwas später.
\end{bew}
\begin{defi}
    Sei \( (X, d) \) ein metrischer Raum, \( M \subset X \).
    Dann ist 
    \[ \inner M = M^o 
    := \set{x \in M: \exists r_x > 0: B_r(x)\subset M}\]
    \[= \set{x \in M: \exists \varepsilon > 0: 
    B_\varepsilon(x) \subset M }\]
    Abschluss:
    \[ \overline{M} := 
    \set{x \in X: \; \forall \varepsilon > 0: 
    B_\varepsilon(x) \cap M \neq \emptyset} \]
    Rand: 
    \[ \delta M := \set{ x \in X: \; \forall \varepsilon > 0: 
    B_\varepsilon(x) \cap M \neq \emptyset, 
    B_\varepsilon(x) \cap M^C \neq \emptyset },
    M^C = X \setminus M. \]
    \[ \inner M \subset M \subset \overline{M}. \]
\end{defi}
\begin{lem}[Hausdorff-Trennungseigenschaft]
    Sei \( X \) ein metrischer Raum, 
    \( x,y \in X, x\neq y \). 
    \[ \Rightarrow \exists \varepsilon > 0: 
    B_\varepsilon(x) \cap B_\varepsilon(y) = \emptyset \]
\end{lem}
\begin{bew}
    Ist \(x \neq y \Rightarrow L = d(x,y) > 0 \) \\
    Sei \( \varepsilon := \frac{L}{2} \) und 
    \( B_\varepsilon(x) := \set{z \in X: d(z,x) < \varepsilon} \)
    Angenommen \( z \in B_\varepsilon(x) \cap 
    B_\varepsilon(y) \)
    \[ \Rightarrow L = d(x,y) 
    \leq d(x,z) + d(z,y) 
    \leq \varepsilon + \varepsilon = 2\varepsilon = L. \]
\end{bew}
\begin{defi}[Konvergenz]
    Sei \( X \) ein metrischer Raum. Die Folge 
    \( (x_n)_n \) von Punkten \( x_n \in X \; \forall n\in\N \)
    (schreiben \( (x_n)_n \subset X \)) konvergiert 
    gegen \( x \in X \), falls 
    \begin{align*}
        &\forall \varepsilon > 0 \; \exists K \in \N: 
        d(x_n, x) < \varepsilon \;\forall n \geq K. \\
        &\Leftrightarrow \; \forall \varepsilon > 0 
        \; \exists K \in \N: x_n \in B_\varepsilon(x) 
        \; \forall n \geq K 
    \end{align*}
    \( \Leftrightarrow \forall \varepsilon > 0  \) 
    ist \( x_n \in B_\varepsilon(x) \) für fast alle 
    \( n\in\N \)
    \[ \Leftrightarrow \limes{n} d(x_n, x) = 0 \]
    Schreiben \( x_n \rightarrow x \), oder 
    \( \limes{n} x_n = x \).
\end{defi}
\begin{bem}
    Der Grenzwert \( x \) ist eindeutig. (wg. Hausdorff)
\end{bem}
\begin{defi}[Abgeschlossene Menge]
    Sei \( X \) ein metrischer Raum und \( A \subset X \) 
    heißt abgeschlossen, falls 
    \[ (x_n)_n \subset A, x_n \rightarrow x 
    \Rightarrow x \in A. \]
\end{defi}
\begin{bem}
    \(\emptyset, X \) sind abgeschlossen.
\end{bem}
\begin{satz}
    In einem metrischen Raum 
    \( X \) gilt für alle \( A \subset X \):
    \[ A \text{ ist offen } 
    \Leftrightarrow A^C = X \setminus A \text{ ist 
    abgeschlossen}. \]
\end{satz}
\begin{bew}
    \gqq{\( \Rightarrow \)}: Sei \(M\) offen.
    \( M^C = X \setminus M \). \( (x_n)_n 
    \in M^C, x_n \rightarrow x \).

    Angenommen \( x \notin M^C\), d. h. \(x \in M \).
    \(M\) offen \(\Rightarrow \exists \varepsilon > 0:
    B_\varepsilon(x) \subset M. \) \\
    \( \Rightarrow \) ist \(y \in M^C: d(y, x) 
    \geq \varepsilon \) \\
    Andererseits \( d(x_n, x) \rightarrow 0 
    (n \rightarrow \infty)  \) \Lightning{} \\
    \( \Rightarrow x \in M^C \)
    \gqq{\( \Leftarrow \)}:
    Annahme: \( M^C \) ist abgeschlossen\\
    Angenommen, \(M\) wäre nicht offen.
    \[ \Rightarrow \exists x\in M: 
    \forall \varepsilon > 0 : B_\varepsilon(x) \cap 
    M^C \neq \emptyset. \]    
    Wähle \( \varepsilon = \frac{1}{n} 
    \Rightarrow \exists x_n \in B_{\frac{1}{n}}(x) \cap 
    M^C \).
    \begin{align*}
        &\Rightarrow (x_n)_n \subset M^C, x_n \rightarrow x 
        \overunderset{M^C}{\text{abgeschlossen}}{\Rightarrow} 
        x \in M^C \\
        &\Rightarrow x \in M \cap M^C = \emptyset \text{ \Lightning} 
    \end{align*}
\end{bew}
\begin{kor}
    Für Teilmengen eines metrischen Raums \(X\) gilt: 
    \begin{enumerate}
        \item \( \emptyset, X \) abgeschlossen.
        \item Die Vereinigung endlich vieler abgeschlossen 
        Mengen ist abgeschlossen.
        \item Der Durchschnitt beliebig vieler abgeschlossener 
        Mengen ist abgeschlossen.
    \end{enumerate}
\end{kor}
\begin{bew}
    Satz 10 und de Morgan Regeln.
\end{bew}
\begin{bem}
    \( I \) Indexmenge, \( j \in I \) ist 
    \( U_j \subset X \) offen.
    \[ U = \bigcup_{j\in I} U_j \]
    \[ x \in U \Leftrightarrow \exists j_0 \in I: 
    x \in U_{j_0}. \]
\end{bem}
\begin{satz}
    Sei \( M \) Teilmenge eines metrischen 
    Raumes \( X \).
    \begin{enumerate}
        \item \( \inner M \) ist offen und \( U \) offen 
        und \( U \subset M \Rightarrow U \subset \inner M \).
        (d. h. \( \inner M \) ist die größte offene Teilmenge 
        von \(M\))
        \item \( \overline{M} \) ist abgeschlossen und ist 
        \(A \supset M \), \( A \) abgeschlossen 
        \[ \Rightarrow A \supset \overline{M}. \]
        \[ \text{d. h. } \overline{M} 
        = \bigcap_{\substack{A \supset M\\A \text{ abg.}}} 
        U \subset M \]
        \item \( \delta M = \overline{M} \setminus (\inner M) \).
    \end{enumerate}
\end{satz}
\begin{bew}
    1.\\
    \( x \in \inner M \). Das heißt 
    \[ \exists r > 0: B_r(x) \subset M. \]
    Sei \( y \in B_r(x) \) und \( \varepsilon := d(y,x) > 0 \).
    \[ \Rightarrow B_\varepsilon(y) \subset B_r(x) 
    \subset M. \]
    \[ \Rightarrow y \in \inner M \]
    \[ \Rightarrow B_r(x) \subset \inner M \]
    \[ \Rightarrow \inner M \text{ ist offen}. \]
    Sei \( U \subset M \) offen. 
    \( \Rightarrow \forall x \in U 
    \exists \varepsilon = \varepsilon_x > 0: 
    B_\varepsilon(x) \subset U \subset M \).

\end{bew}
\end{document}