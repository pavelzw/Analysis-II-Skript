%26.04.2019
\documentclass[../ana2.tex]{subfiles}
\begin{document}

\setcounter{section}{2}
\section{Ein paar Eigenschaften elementarer Funktionen}
\( e^z = \exp z := \sum_{n=0}^\infty \frac{z^n}{n!} \), 
konvergiert absolut \( \forall z\in\C \). \( z = x + iy, x,y \in \R \). \\
\( \exp(z + w) = \exp z \exp w \) (schon bekannt).
\[ \exp(it) = \cos(t) + i\sin(t) \]
\[ \cos t = \sum_{n=0}^{\infty} \frac{(-1)^n}{(2n)!} \cdot t^{2n} \]
\[ \sin t = \sum_{k=0}^\infty \frac{(-1)^k}{(2k+1)!} t^{2k+1}. \]
Erinnerung: Cauchy-Produktformel: 
\begin{align*}
    e^z e^w &= \sum_{k=0}^\infty \frac{z^k}{k!} 
    \sum_{k=0}^\infty \frac{w^m}{m!} \\
    &= \sum_{k=0}^\infty \frac{1}{n!} 
    \sum_{k=0}^n \frac{z^k}{k!} \frac{w^{n-k}}{(n-k)!} \\
    &= \sum_{k=0}^\infty \frac{1}{n!} 
    \sum_{k=0}^n \binom{n}{k} z^k w^{n-k} \\
    &= \sum_{k=0}^\infty \frac{1}{n!} (z + w)^n
\end{align*}
\begin{satz}\leavevmode
    \begin{enumerate}[label=(\alph*)]
        \item \( \forall z \in \C: e^z \neq 0 \).
        \item \( \exp(z)' = \frac{d}{dz} \exp(z) = \exp(z) 
        := \limesx{h}{0} \frac{\exp(z+h) - \exp(z)}{h} \), erlauben 
        \( z,h \in \C \).
        \item \( \abb{\exp}{\R}{\R} \) strikt wachsend. 
        \( e^x \rightarrow \infty, x \rightarrow \infty, \quad
        e^x \rightarrow 0, x\rightarrow -\infty \). \\
        \( \abb{\exp}{\R}{(0,\infty)} \) ist bijektiv.
        \item \( \exists \pi > 0: e^{\frac{\pi}{2}i} = i \) und
        \( e^{iz} = 1 \Leftrightarrow \frac{z}{2\pi i} \in \Z \).
        \item \( \abb{\exp}{\C}{\C} \) ist periodisch mit 
        der Periode \( 2\pi i \). 
        \( \exp(z + 2\pi i) = \exp(z) \;\forall z\in\C \).
        \item \( \abb{e^i}{\R}{S = \set{z \in \C: \abs{z} = 1}}, 
        t \mapsto e^{it} \) ist surjektiv.
        \item Ist \( w \in \C \setminus \set{0} \Rightarrow 
        \exists z \in \C: w = e^z \).
    \end{enumerate}
\end{satz}
\begin{bew}\leavevmode
    \begin{enumerate}[label=(\alph*)]
        \item \( e^z \cdot e^{-z} = e^{z-z} = e^0 = 1 
        \Rightarrow e^z \neq 0 \;\forall z\in\C, 
        (e^z)^{-1} = e^{-z} \).
        \item \( \frac{\exp(h) - \exp(0)}{h} = \frac{\exp(h) - 1}{h},
        \frac{\exp(h) - 1}{h} - 1 = \frac{\exp(h) -1-h}{h} \)
        \begin{align*}
            &\Rightarrow \abs{\frac{\exp(h) - 1}{h} - 1} 
            = \abs{\sum_{n=1}^{\infty} \frac{h^n}{(n+1)!}} \\
            &= \abs{h}\abs{\frac{h^{n-1}}{(n+1)!}} 
            = \frac{\sum_{n=1}^{\infty} \frac{h^n}{n!} -1 -h }{h} \\
            &= \frac{\sum_{n=2}^{\infty} \frac{h^n}{n!}}{h} 
            = \sum_{n=1}^{\infty} \frac{h^n}{(n+1)!} \\
            &\leq \abs{h} \sum_{n=0}^{\infty} 
            \underbrace{\frac{\abs{h}^n}{(n+2)!}}_{\leq \frac{\abs{h}^n}{n!}} 
            \leq \abs{h} \sum_{n=0}^{\infty} \frac{\abs{h}^n}{n!} \\
            &= \abs{h} \exp(\abs{h}) \overset{h \rightarrow 0}{\rightarrow} 0.
        \end{align*}
        \[ \exp'(0) = 1. \]
        \begin{align*}
            \frac{ \exp(z + h) - \exp z }{h} 
            &= \frac{ \exp(z)\exp(h) - \exp(h) }{h} \\
            &= \exp(z) \frac{\exp(h) - 1}{h} \\
            &\Rightarrow \limesx{h}{0} 
            \frac{\exp(z + h) - \exp(z)}{h} \\
            &= \exp(z) \limesx{h}{0} \frac{\exp(h) - 1}{h} \\
            &= \exp(z) \cdot 1 = \exp(z) \\
            &\Rightarrow \exp'(z) = \frac{d}{dz} \exp(z) = \exp(z).
        \end{align*}
        \item \( 0 \leq x \mapsto \exp(x) 
        = \sum_{n=0}^\infty \frac{x^n}{n!} \) 
        ist strikt wachsend.
        \[ \exp(x) \geq 1 + x \rightarrow \infty, 
        x\rightarrow \infty \] 
        \[ \limesx{x}{-\infty} \exp(x) 
        = \limes{x} \exp(-x) = \limes{x} \frac{1}{\exp(x)} = 0. \]
    
Beobachtung: \( t \in \R \)
\[ \overline{e^{it}} = \overline{\sum_{n=0}^{\infty} \frac{(it)^n}{n!}}
= \sum_{n=0}^{\infty} \frac{(-it)^n}{n!} = e^{-it} \]
\[ \Rightarrow \abs{e^{it}}^z = \overline{e^{it}}e^{it} = e^{-it} e^{it}
= e^0 = 1 \Rightarrow e^{it} \in S = \set{z \in \C: \abs{z} = 1} \forall t \in \R \]
Auch: Euler:
\[ e^{it} = \cos t + i\sin t \]
\begin{align*}
    &= \sum_{n=0}^{\infty} \frac{(it)^n}{n!} \\
    &\overundersett{abs.}{konv.}{=}
    \sum_{n \text{ ungerade}} \frac{(it)^n}{n!} + \sum_{n \text{ ungerade}} \frac{(it)^n}{n!} \\
    &= \sum_{k=0}^{\infty} \frac{(it)^{2k}}{(2k)!} 
    + \sum_{k=0}^{\infty} \frac{(it)^{2k + 1}}{(2k + 1)!} \\
    &= \sum_{k=0}^{\infty} \frac{(-1)^k t^{2k}}{(2k)!} 
    + i \sum_{k=0}^{\infty} \frac{(-1)^k}{(2k+1)!}t^{2k+1}
\end{align*}
\begin{align*}
    \frac{d}{dt} (\cos t + i \sin t) &= \cos'(t) + i \sin'(t) \\
    &= \frac{d}{dt} e^{it} = e^{it} i(\cos t + i \sin t) \\
    &= -\sin t + i \cos t \\
    &= \cos'(t) + i \sin'(t)
\end{align*}
\[ \Rightarrow \cos' = -\sin, \sin' = \cos. \]
Auch 
\begin{align*}
    1 &= \abs{ e^{it} }^2 = \overline{e^{it}} e^{it} \\
    &= (\cos t - i \sin t)(\cos t + i \sin t) \\
    &= \cos^2 t + \sin^2 t.
\end{align*}
% Bild
Einschließungen von \( \cos, \sin \):
\[ 0 \leq t \leq 2: \cos t = \sum_{k=0}^{\infty} \frac{(-1)^k}{(2k)!}t^{2k}
= 1 - t^2 + \frac{t^4}{4!} -+ \ = \sum_{n=0}^{\infty} (-1)^n a_n \]
beachte: Ist \( 0 \leq t \leq 2 \): 
\[ a_n := \frac{t^{2n}}{(2n)!} n\in\N, a_0 = 1. \]
\[ \cos t = \sum_{n=0}^\infty (-1)^n a_n 
= 1 - a_1 + \sum_{n=2}^\infty (-1)^n a_n 
\leq 1 - a_1 + a_2 = 1 - \frac{t^2}{2} + \frac{t^4}{4!}. \]
und für \( t\in [0,2] \) gilt: 
\[ \cos t \geq 1 - \frac{t^2}{2}. \] 
Genauso gilt: 
\[ t-\frac{t^3}{3!} \leq \sin t \leq t. \]
Bild:\\
% Bild
\( \cos(2) \leq 1 - \frac{2^2}{2} + \frac{2^4}{4!} 
= -1 + \frac{16}{24} < 0 \) \\
Zwischenwertsatz: \( \exists \) kleinste \( t_0 > 0 \): 
\[ \cos(t_0) = 0. \]
\( \sin \) Bild:
% Bild
\( \Rightarrow \sin t > 0 \) auf \( (0,2] 
\Rightarrow \cos \) ist auf \( [0,2] \) strikt fallend. \\
\( \Rightarrow \existse 0 < t_0 < 2: \cos t_0 = 0 \).\\
\item Folgt in e)
\item Wir definieren:
\[ \pi := 2 t_0. \]
\( \frac{\pi}{2} \) ist die einzige Nullstelle 
von \( \cos \) auf \( [0,2] \).\\
\[ \Rightarrow \cos^2(t) + \sin^2(t) = 1 
\Rightarrow \sin(\frac{\pi}{2}) \in \set{-1, 1}. \]
Da \( \sin > 0 \) auf \( (0, 2] \Rightarrow \sin(\frac{\pi}{2}) = 1 \)
\[ \Rightarrow e^{\frac{pi}{2} i} 
= \cos(\frac{\pi}{2}) + i \sin(\frac{\pi}{2}) = i \tag{4} \]
\begin{align*}
    \Rightarrow e^{i\pi} &= (e^{\frac{\pi}{2}i})^2 \\
    &= i^2 = -1 = \cos(\pi) + i\sin (\pi) \\
    &\Leftrightarrow \cos(\pi) = -1, \sin(\pi) = 0
\end{align*}
\begin{align*}
    \Rightarrow e^{2i\pi} &= (e^{\pi i})^2 \\
    &= (-1)^2 = 1 = \cos(2\pi) + i\sin(2\pi) \\ 
    &\Leftrightarrow \cos(2\pi) = 1, \sin(2\pi) = 0
\end{align*}
\begin{align*}
    &\oversett{Induktion}{\Rightarrow} e^{2 \pi i n} = 1^n 
    = 1 \forall n\in\N_0 \\
    &\Rightarrow e^{-2\pi i n} = \frac{1}{e^{2 \pi i n}} 
    = 1 \forall n\in\N \\
    &\Rightarrow e^{2 \pi i n} = 1 \forall n \in \Z \\
    &\Rightarrow e^{z + 2 \pi i} = e^z e^{2\pi i} 
    = e^z \forall z\in\C \tag{5} \\
    &\Rightarrow \text{ (e)}
\end{align*}
Sei \( z = x + iy, x,y \in\R \).
\[ e^z = e^{x+iy} = e^x e^{iy} = e^x(\cos y + i \sin y) \]
\[ \Rightarrow \abs{e^z} = e^x \abs{e^{iy}} = e^x. \]
Also: \( e^z = 1 \Rightarrow x = \Re(z) = 0 \).\\
\Dphp{} aus \( e^z = 1 \) folgt \(x = \Re(z) = 0 \).\\
Sei \( e^{iy} = 1 \). Wollen \( \frac{y}{2\pi} \in \Z \).\\
Zu zeigen: Aus \( 0 < y < 2\pi \) folgt \( e^{iy} \neq 1 \).\\
Ang. \( 0 < y < 2\pi \). \\
Schreibe \( e^{i \frac{y}{4}} = u + iv, \quad u,v\in\R \).\\
\[ 0 < \frac{y}{4} < \frac{\pi}{2} \]
\[ \Rightarrow u,v > 0, u^2 + v^2 \neq 1. \]
Angenommen \( e^{iy} = 1 \) (oder nur \( e^{iy} \in \R \))
\[ e^{iy} = \left( e^{i \frac{y}{4}} \right)^4 
= (u+iv)^4 = \underbrace{u^4 - 6u^2 v^2 + v^4 + 4iuv(u^2 - v^2)}_{
    \in \R \Leftrightarrow u^2 = v^2 
} \tag{7} \]
\[ \Rightarrow u^2 = v^2 \text{ und } u^2+v^2 
= 1 \Rightarrow u^2 = v^2 = \frac{1}{2} \]
\[ \Rightarrow e^{iy} = u^4 - 6u^2v^2+v^4 
= \left(\frac{1}{2}\right)^2 - 6 \cdot \frac{1}{2} \cdot \frac{1}{2}
+ \left(\frac{1}{2}\right)^2 = -1 \neq 1 \]
\Dphp{} ist \( 0 < y < 2\pi \) und \( e^{iy} \in \R \), 
dann ist \( e^{iy} = -1 \). Dann ist aber \( y = \pi \).
\item Wir wissen schon \( \abs{e^{it}} = 1 \;\forall t \in\R \).\\
Sei \( w\in\C, \abs{w} = 1 \), Frage: \( \exists t \in \R: e^{it} = w \)?\\
\[ w = u + iv, u^2 + v^2 = 1. \]
1. Fall: 
\( u, v \geq 0 \). Da \( u \leq 1 \Rightarrow 
\exists 0 \leq t \leq \frac{\pi}{2}: \cos t = u. \)
\[ \sin^2 t = 1 - \cos^2 t = 1 - u^2 = v^2, v \geq 0 
\Rightarrow \sin t = v. \]
\[ \Rightarrow e^{it} = \cos t + i \sin t = u + iv = w. \]
2. Fall: 
\( u < 0, v \geq 0 \). 
\[ \Rightarrow -i w = -i (u + iv) = v + i(-u). \]
\[ \oversett{1. Fall}{\Rightarrow} \exists t \in \R: 
-i w = e^{it} \Rightarrow w = i e^{it} < e^{i \frac{\pi}{2}} e^{it} 
= e^{i(t + \frac{\pi}{2})} \]
3. Fall: \( v < 0 \)
\[ \overundersett{1. und 2.}{Fall}{\Rightarrow} \exists t \in \R: -w = e^{it} \]
\[ \Rightarrow w = -e^{it} = e^{i\pi + it} = e^{i(t+\pi)} \]

\item Sei \( w \in \C \setminus \set{0} \).
\[ \Rightarrow w = \abs{w} \underbrace{\frac{w}{\abs{w}}}_{
    =e^{iy}, y\in\R}, 
\abs{\frac{w}{\abs{w}}} = 1 \]
\[ \Rightarrow w = \abs{w} e^{iy} \]
\[ e^{x_1} < \abs{w} < e^{x_2} \]
\[ \abs{w} > 0 \oversett{ZWS}{\Rightarrow} 
\exists x \in \R, \abs{w} = e^x \]
\[ \Rightarrow w = e^x e^{iy} = e^{x + iy} = e^z, 
z = x + iy. \]
\end{enumerate}
\end{bew}
\begin{bem}
    \( \R \ni t \mapsto e^{it} = \cos t + i \sin t \).\\
    \( e^{i t_1} = e^{i t_2} 
    \Leftrightarrow e^{i(t_1 - t_2)} = 1 \)
    \[ \overundersett{Satz 1}{d)}{\Leftrightarrow} 
    i (t_1 - t_2) = 2 \pi i k, k\in\Z \]
    \[ t_1 = t_2 + 2\pi k, k\in\Z. \]
\end{bem}
\begin{kor}
    Gleichung \( w^n = 1, n \in \N \) hat 
    genau \(n\) Lösungen.
    \[ \zeta_k = e^{\frac{ik2\pi}{n}} 
    = \cos(\frac{2\pi k}{n}) + i \sin(\frac{2\pi k}{n}) \]
    (Einheitswurzeln).
\end{kor}
\begin{bew}
    \gqq{\(\Leftarrow \)}: 
    \[ (e^{\frac{i k 2 \pi}{n}})^n 
    = (e^{\frac{i k 2 \pi}{n} \cdot n}) 
    = (e^{i k 2 \pi} \oversett{Satz 1 (e)}{=} = e^0 = 1 \]
    \gqq{\( \Rightarrow \)}: \\
    Sei \( w^n = 1 \Rightarrow w \neq 0 \)
    \[ \overundersett{Satz 1}{g)}{\Rightarrow} \exists z \in \C: w = e^z. \]
    \[ \Rightarrow 1 = w^n = (e^z)^n = e^{nz} \]
    \[ \oversett{Satz 1 (d)}{\Rightarrow} n z = 2 \pi i k, k\in\Z \]
    \[ \Rightarrow z = \frac{2 \pi i k}{n}, k\in\Z. \]

    \begin{align*}
        e^{ \frac{i 2 \pi (n + 1)}{n} } 
        &= e^{i 2\pi + i \frac{2\pi}{n}} \\
        &= e^{i 2\pi} e^{i \frac{2\pi}{n}} \\
        &= e^{i \frac{2\pi}{n}}.
    \end{align*}
    \( \Rightarrow \) Nur die Terme 
    \( \zeta_k = e^{\frac{2\pi ik}{n}}, k=1,2,\dots,n \)
    sind verschieden.
    Außerdem gilt \( \zeta_k = \zeta_1^k \), da
    \[ \zeta_k = e^{\frac{2\pi ik}{n}} = \left(e^{\frac{2\pi i}{n}}\right)^k 
    = \zeta_1^k \]
\end{bew}
\subsection*{Polarkoordinaten}
Sei \( w\in\C \setminus \set{0} \). 
\[ \oversett{Satz 1}{\Rightarrow} \exists \varphi \in \R: 
\frac{w}{\abs{w}} = e^{i \varphi}. \]
\[ \Rightarrow w = \abs{w} e^{i\varphi} \]
D.\ h.\ \( w = r e^{i\varphi}, r = \abs{w} \), 
\[ w = x +iy = r(e^{i\varphi}) = r(\cos \varphi + i\sin \varphi) \]
%Bild
\[ \Rightarrow x = r\cos \varphi, y = r\sin \varphi \]
Winkel \( \varphi \) bis auf ganzzahlige Vielfache von 
\( 2\pi \) bestimmt.
\subsection*{Cosinus und Sinus}
\[ \text{Euler: } e^{it} = \cos t + i \sin t, t\in\R. \]
\[ \Rightarrow \cos t = \frac{1}{2}(e^{it} + e^{-it}) \]
\[ \Rightarrow \sin t = \frac{1}{2i}(e^{it} - e^{-it}) \]
\begin{defi}
    \( \forall z\in\C \) setzen wir 
    \[ \cos z = \frac{1}{2} (e^{iz} + e^{-iz}). \]
    \[ \sin z = \frac{1}{2i} (e^{iz} - e^{-iz}). \]
    Insbesondere ist auch 
    \[ e^{iz} = \cos z + i \sin z \]
\end{defi}
\begin{satz}[Additionstheoreme]
    \[ \cos(z+w) = \cos z \cos w - \sin z \sin w \]
    \[ \sin(z+w) = \cos z \sin w + \sin z \cos w \]
\end{satz}
\begin{bew}
    \begin{align*}
        e^{i(z+w)} &= e^{iz} e^{iw} = (\cos z + i \sin z)(\cos w + i \sin w) \\
        &= \cos (z+w) + i \sin(z+w) \\
        &= \cos z \cos w - \sin z \sin w + i(\cos z \sin w + \sin z \cos w) \\
    \end{align*}

    Analog für \( e^{-i (z + w)} = \ldots \) 
    dann addieren und subtrahieren.
\end{bew}
\begin{bem}
    \begin{enumerate}
        \item Satz 4 \( \Rightarrow \cos(2z) 
        = \cos^2 z - \sin^2 z \) \\
        \( \sin(2z) = 2 \sin z \cos z \).
        \item Haben \( \limesx{z}{0} \frac{e^z - 1}{z} = 1 
        \Rightarrow \limesx{z}{0} \frac{\sin z}{z} = 1 \).
    \end{enumerate}
\end{bem}
\begin{kor}
    \( \forall z \in \C \) gilt 
    \[ \cos \left( z+\frac{\pi}{2}\right) = -\sin z, \cos (z + \pi) 
    = \cos z, \cos(z + 2\pi) = \cos z \]
    
    \[ \sin\left(z + \frac{\pi}{2}\right) = \cos z, 
    \sin(z + \pi) = -\sin z, \sin(z + 2\pi) = \sin z \]
\end{kor}
\begin{bem}
    Also haben \( \cos, \sin \) die reelle Periode \( 2\pi \).
\end{bem}
\begin{bew}
    Haben \( e^{i\frac{\pi}{2}} = i \)
    %Tabelle
    \[ \cos \left(z + \frac{\pi}{2}\right) 
    = \cos z \cos \frac{\pi}{2} - \sin z \sin \frac{\pi}{2} = -\sin z \]
    u.s.w.
\end{bew}
\begin{satz}
    \( \cos \) hat auf \( \C \) genau die Nullstellen 
    \( \frac{\pi}{2} + k \pi, k\in\Z \).\\
    \( \sin \) hat auf \( \C \) genau die Nullstellen 
    \( k \pi, k\in\Z \).
\end{satz}
\begin{bew}
    \begin{align*}
        \sin z &= 0 \Leftrightarrow e^{iz} = e^{-iz} \\
        &\Leftrightarrow e^{2iz} = 1 \\
        &\oversett{Satz 1}{\Leftrightarrow} 2iz = 2\pi ik, k \in \Z 
        \Leftrightarrow z = k\pi, k \in \Z
    \end{align*}
    \begin{align*}
        \cos z = 0 \Leftrightarrow e^{iz}
        &= -e^{-iz} \Leftrightarrow e^{2iz} = -1 = e^{i\pi} \\
        \Leftrightarrow e^{2\pi iz - i\pi} 
        &= 1 \Leftrightarrow 2\pi iz - i\pi \\
        &= 2\pi ik, k \in \Z \\
        &\Leftrightarrow z = \Leftrightarrow + k \pi, k \in \Z
    \end{align*}
\end{bew}
\subsection*{Logarithmus}
Wir hatten \( \abb{\exp}{\R}{(0,\infty)} \) ist strikt 
wachsend und bijektiv.
\[ \Rightarrow \forall x > 0 \existse y \in \R: e^y = x. \]
Nennen \( y = \ln x, \abb{\ln}{(0,\infty)}{\R}, 
y = \ln x \Leftrightarrow e^y = x \).
\( \ln \) ist die Umkehrfunktion von \( \exp \). \\
\( \ln \) ist strikt wachsend auf \( (0,\infty) \).\\
\( \ln \) ist stetig und differenzierbar. \\
Satz von der Umkehrfunktion: 
\[ (\ln x)' = \frac{d}{dx} \ln x = \frac{1}{\exp'(\ln x)} 
= \frac{1}{\exp(\ln x)} = \frac{1}{x}. \]
\begin{satz}[Funktionalgleichung des natürlichen Logarithmus]
    \( \forall x, y > 0: \)
    \[ \ln(x \cdot y) = \ln x + \ln y. \]
\end{satz}
\begin{bew}
    \( \exp(\ln(x y)) = x y = \exp(\ln x) \exp(\ln y) 
    = \exp(\ln x + \ln y) \\
    \Leftrightarrow \ln(x \cdot y) = \ln x + \ln y \).
\end{bew}
%Bild 
\begin{satz}
    \( \limes{x} \ln x = \infty, \limesx{x}{0+} \ln x 
    = -\infty \Rightarrow \text{Bild} \) \\
    Sei \( n \in \N \)
    \[ \limes{x} \frac{\ln x}{\sqrt[n]{x}} = 0 \]
\end{satz}
\begin{bew}
    Erste zwei: siehe Bild.\\
    Letzte Behauptung: setzen \( x = e^{nt} \).
    \[ \Rightarrow \limes{x} \frac{\ln x}{\sqrt[n]{x}} 
    = \limes{t} \frac{\ln (e^t)}{\sqrt[n]{e^{nt}}} 
    = \limes{t} \frac{t}{e^t} = 0. \]
\end{bew}
\subsection*{Allgemeine Basen}
Sei \( a > 0, x\in\R \). Was ist \( a^x \)?
Sei \( n \in \N: a^n \) ist definiert:
\[ a^0 = 1, a^1 = a, a^{n+1} = aa^n, n \in \N \text{ (induktiv)} \]
Sei \( n \in -\N \). 
\[ a^n := \frac{1}{a^{-n}}. \]
\( \Rightarrow a^n \) ist definiert \( \forall n\in\Z \).\\
\( a^{n_1 + n_2} = a^{n_1} a^{n_2} \;\forall n_1, n_2 \in\Z \).\\
Wir wollen: \( (a^\frac{1}{m})^m = a \).\\
Definieren \( a^{\frac{1}{m}} = \sqrt[m]{a} \).
\[ \Rightarrow a^{\frac{n}{m}} = \sqrt[m]{a^n} 
= (\sqrt[m]{a})^n. \]
\( a^r \) ist definiert \( \forall r \in \Q \) konsistent, da
\[ r = \frac{n_1}{m_1} = \frac{n_2}{m_2}
, n_1,n_2 \in \Z, m_1,m_2 \in \N \]
\[ \Rightarrow a^{\frac{n_1}{m_1}} = a^{\frac{n_2}{m_2}} 
\text{ (nachrechnen)} \]
Was ist für \( r\in \R \setminus \Q \)?\\
Beachte: \( a = e^{\ln a} \). 
\[ \Rightarrow a^2 = e^{\ln a} e^{\ln a} = e^{2\ln a}
\Rightarrow a^n = (e^{\ln a})^n = e^{n\ln a} \;\forall n\in\N. \]
\[ \Rightarrow a^{-n} = \frac{1}{a^n} = \frac{1}{(e^{\ln a})^n}
= \frac{1}{e^{n\ln a}} = e^{-n\ln a} \]
\[ \Rightarrow \forall n\in\Z: a^n = e^{n \ln a}. \]
Sei \( m \in \N \):
\[ \left( e^{\frac{1}{m} \ln a} \right)^m = e^{\frac{m}{m} \ln a}
= e^{\ln a} = a \]
\[ \Rightarrow e^{\frac{1}{m} \ln a = \sqrt[m]{a}} = a^{\frac{1}{m}} \]
\[ \Rightarrow \forall n\in\N_0: 
\left( e^{\frac{1}{m} \ln a} \right)^n 
= e^{\frac{n}{m} \ln a} = (a^{\frac{1}{m}})^n = a^{\frac{n}{m}} \]
\[ \Rightarrow e^{\frac{n}{m} \ln a} = a^{\frac{n}{m}} \;\forall n \in \N_0, m \in \N \]
Genauso: \( -n \in \N \):
\[ e^{\frac{n}{m} \ln a} = \frac{1}{e^{-\frac{n}{m} \ln a}}
= \frac{1}{a^{-\frac{n}{m}}} = a^{\frac{n}{m}} \]
\[ \Rightarrow a^r = \exp(r \ln a) \forall r \in \Q \]
Sei \( x\in\R \). \( \Q \) ist dicht in \( \R \). 
\( \exists \) Folge \( (r_n)_n \in \Q, r_n \rightarrow x \).\\
\[ a^{r_n} = \exp(r_n \ln a) \rightarrow \exp(x\ln a) \]
da \( \exp \) stetig.
\begin{defi}[Exponentialfunktion zur Basis \(a\)]
    Sei \( a > 0, z\in\C \).
    \[ a^z := \exp(z \ln a). \]
\end{defi}
Es gilt: \( f_a(x) = \exp(x\ln a) \)
\begin{enumerate}
    \item \( a^{z+w} = a^z \cdot a^w \;\forall z,w \in \C \).
    \item \( \limesx{z}{a} \frac{a^z -1}{z} = \ln a \).
    \item \( \C \ni z \mapsto a^z \) ist stetig.
    \item \( f_a(x) := a^x, x\in \R, f_a : \R \rightarrow (0,\infty) \)
    ist strikt wachsend bzw.\ fallend, falls
    \( a > 1 \) bzw.\  \( 0 < a < 1 \)
    ist.
    \item Ist \(a \neq 1 \), so nimmt \( f_a \) auf \( \R \) jeden Wert aus
    \( (0, \infty) \) an.
\end{enumerate}
%Bild
\begin{defi}
    \( \ln_a x = y \) genau dann wenn 
    \( a^y = x \ (\Leftrightarrow \exp(y\ln a) = x) \)
    \[ \Rightarrow \forall x > 0 \existse y \in \R: a^y = x. \]
\end{defi}
\begin{satz}[Rechenregeln für \( \ln \)]\leavevmode
    \begin{enumerate}
        \item \( \ln (a^x) = x \ln a \;\forall x \in \R \).
        \item \( (a^x)^y = a^{xy} \;\forall x,y \in \R \).
        \item \( a^z b^z = (ab)^z \;\forall z\in\C \).
    \end{enumerate}
\end{satz}
\begin{bew}\leavevmode
    \begin{enumerate}
        \item \( a^x = \exp(x\ln a) \Rightarrow \)
        \item \( (a^x)^y = \exp(y\ln (a^x)) = \exp (xy \ln a)
        =a^{xy} \)
        \item \( a^z b^z = \exp(z \ln a) \exp(z \ln b) \\
        = \exp(z (\ln a + \ln b)) = \exp(z \ln(a\cdot b)) 
        = (ab)^z. \)
    \end{enumerate}
\end{bew}
\subsection*{Die Arcus Funktionen}
\begin{align*}
    \tan z &= \frac{\sin z}{\cos z}, 
    z \in \C \setminus \set{\frac{\pi}{2}, k\pi, k \in \Z} \\
    \cot z &= \frac{\cos z}{\sin z}, 
    z \in \C \setminus \set{k\pi, k \in \Z}
\end{align*}
\( \tan \): auf \( \R \) definiert 
\( x \neq \frac{\pi}{2} + k \pi, k \in \Z \).\\
\( \abb{\tan}{(- \frac{\pi}{2}, \frac{\pi}{2})}{\R}\) ist stetig.\\
\( \limesx{x}{\frac{\pi}{2}-} \tan x 
= \limesx{x}{\frac{\pi}{2}-} \frac{\sin x}{\cos x} 
= +\infty \).\\
Analog: \( \limesx{x}{\frac{\pi}{2}+} \tan x = -\infty \).\\
Auf \( [0,\frac{\pi}{2}] \) ist \( \sin \) strikt wachsend, 
\( \cos \) strikt fallend.\\
\( \Rightarrow \tan \) ist strikt wachsend auf 
\( [0,\frac{\pi}{2}) \).\\
\( \tan \) ist antisymmetrisch: \( \tan x = -\tan(-x) \)
\( \Rightarrow \tan \) ist auf 
\( (-\frac{\pi}{2},\frac{\pi}{2}) \) strikt wachsend.
%Bild tangens mit Hauptzweig
Bild:
\begin{tikzpicture}
    
\end{tikzpicture}
Umkehrfunktion von 
\( \abb{\tan}{\left(-\frac{\pi}{2}, \frac{\pi}{2}\right)}{\R}\) existiert.
\[ \abb{\arctan}{\R}{\left(-\frac{\pi}{2}, \frac{\pi}{2}\right)} \]

Graph vom Arcustangens:\\
%Bild
\begin{tikzpicture}
    
\end{tikzpicture}\\
Analog: 
\begin{align*}
    \abb{\cos}{[0,\pi]}{[-1,1]} \\
    \abb{\sin}{\left[-\frac{\pi}{2}, \frac{\pi}{2}\right]}{[-1,1]}
\end{align*}
strikt monoton und stetig.\\
\( \oversett{ZWS}{\Rightarrow} \) sie sind surjektiv.
Sie sind bijektiv aufgrund der strikten Monotonie.
\( \Rightarrow \) Umkehrfunktionen:
\begin{align*}
    \abb{\arccos}{[-1,1]}{[0,\pi]} \\
    \abb{\arcsin}{[-1,1]}{\left[-\frac{\pi}{2},\frac{\pi}{2}\right]}
\end{align*}
Bilder:\\
%Bild
\( \cos \)
\begin{tikzpicture}
    
\end{tikzpicture}
\( \sin \)
\begin{tikzpicture}
    
\end{tikzpicture}
\( \arccos \)
\begin{tikzpicture}
    
\end{tikzpicture}
\( \arcsin \)
\begin{tikzpicture}
    
\end{tikzpicture}

\subsubsection*{Ableitungen}
\begin{align*}
    (\tan x)' &= \ddx{x} \tan x \\
    &= \ddx{x} \frac{\sin x}{\cos x}\\
    &= \frac{\cos x \sin'x - \sin x \cos'x}{\cos^2 x}\\
    &= \frac{\cos^2 x + \sin^2 x}{\cos^2 x} \\
    &= \frac{1}{\cos^2 x}.
\end{align*}

\[ (\arctan x)' = \frac{1}{\tan'(\arctan x)} 
= \cos^2(\arctan x) \]

Nebenrechnung:
\[ \tan^2 x = \frac{\sin^2 x}{\cos^2 x} 
= \frac{\sin^2 x + \cos^2 x - \cos^2 x}{\cos^2 x}
= \frac{1}{\cos^2 x} - 1 \]
\[ \Rightarrow \cos^2 x = \frac{1}{1 + \tan^2 x} \]
\[ \Rightarrow (\arctan x)' = \cos^2(\arctan x) 
= \frac{1}{1 + \tan^2(\arctan x)} = \frac{1}{1 + x^2}. \]

\[ (\arccos x)' = \frac{1}{\cos'(\arccos x)} 
= \frac{1}{-\sin(\arccos x)}
= -\frac{1}{\sqrt{1-\cos^2 (\arccos x)}} 
= -\frac{1}{\sqrt{1-x^2}} \]
Ähnlich: \( (\arcsin x)' = \frac{1}{\sqrt{1 - x^2}} \).
\subsection*{Hyperbolische Funktionen}
Es existieren neben \( \cos, \sin, \tan \) und \( \cot \) 
noch der \( \cosh, \sinh, \tanh \) und der \( \coth \).
\begin{align*}
    \cosh x &= \frac{1}{2} (e^x + e^{-x}) \geq 1\\
    \sinh x &= \frac{1}{2}(e^x-e^{-x}) \\
    \tanh x &= \frac{\sinh x}{\cosh x} \;\forall x\in\R \\
    \coth x &= \frac{\cosh x}{\sinh x}, \;x \neq 0
\end{align*}
Bild:
%Bild
\( \abb{\cosh}{\R}{[1,\infty)} \) ist stetig, 
strikt wachsend auf \( [0,\infty) \).
\begin{tikzpicture}

\end{tikzpicture}
\( \abb{\sinh}{\R}{\R} \) ist stetig und strikt
wachsend.
\begin{tikzpicture}
    
\end{tikzpicture}
\( \Rightarrow \) Umkehrfunktionen existieren.
\begin{align*}
    &\abb{\arcosh}{[1, \infty)}{[0, \infty)} \\
    &\abb{\arsinh}{\R}{\R}
\end{align*}
\end{document}