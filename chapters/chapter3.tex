%26.04.2019
\documentclass[../ana2.tex]{subfiles}
\begin{document}

\setcounter{section}{1}
\section{Ein paar Eigenschaften elementarer Funktionen}
\( e^z = \exp z := \sum_{n=0}^\infty \frac{z^n}{n!} \), 
konvergiert absolut \( \forall z\in\C \). \( z = x + iy, x,y \in \R \). \\
\( \exp(z + w) = \exp z \exp w \) (schon bekannt).
\[ \exp(it) = \cos(t) + i\sin(t) \]
\[ \cos t = \sum_{n=0}^{\infty} \frac{(-1)^n}{(2n)!} \cdot t^{2n} \]
\[ \sin t = \sum_{k=0}^\infty \frac{(-1)^k}{(2k+1)!} t^{2k+1}. \]
\begin{satz}
    \begin{enumerate}%[label=(\alph*)]
        \item \( \forall z \in \C \) ist \( \exp(z) \neq 0 \)
        \item \( \frac{d}{dx} \exp(ax) = a \exp(ax) 
        \;\forall a\in \C \).\\ \( \exp(t)^i = \exp(t) \).
        \item \( \abb{\exp}{\R}{(0,\infty)} \) strikt wachsend, \\
            \( e^x \rightarrow \infty, x \rightarrow \infty \) \\
            \( e^x \rightarrow 0, x \rightarrow -\infty \)
        \item \( \exists \) pos. Zahl \( \pi \): 
        \( e^{\frac{i\pi}{2}} = i \) und 
        \[ e^z = 1 \Leftrightarrow \frac{z}{2\pi i} \in \Z. \]
        \item \( \abb{\exp}{\C}{\C} \) ist periodisch mit 
        Periode \( 2\pi i \). \\
        \Dphp{} \( \exp(z +2\pi i) = \exp(z) \; \forall z\in\C \)
        und es gibt kein \( 0 < d < 2\pi \): 
        \[ \exp(z + di) = \exp(z)\; \forall z \in \C. \]
        \item \( \R \ni t \rightarrow \exp(it) = e^{it} \) bildet 
        \( \R \) auf den Einheitskreis in \( \C \) ab.
        \item Ist \( w \in \C, w \neq 0 \) dann gibt es ein \( z \in \C \)
        mit \( \exp(z) = w \)
        \begin{align*}
            \exp(\underbrace{it + 2\pi i}_{i(t+2\pi)}) &= \exp(it) = \cos(t) + i\sin(t) \\
            &= \exp(i(t+2\pi)) = \cos(t+2\pi) + i\sin(t+2\pi)
            &\Rightarrow \cos(t + 2\pi) = \cos(t), \sin(t + 2\pi) = \sin(t)
        \end{align*}
    \end{enumerate}
\end{satz}
\end{document}