\documentclass[../ana2.tex]{subfiles}
\begin{document}
\setcounter{section}{4}
\section{Hauptsatz der Differential- und Integralrechnung}

\begin{defi}
    Gegeben \( \abb{f}{I}{\C} \) (oder \( \R^n \)), 
    \(I\) Intervall    
    \( \abb{F}{I}{C} \) heißt Stammfunktion (von \(f\)), falls
    \begin{enumerate}
        \item \(F\) stetig ist.
        \item \(F\) außerhalb einer abzählbaren Teilmenge 
        \( A \) von \( I \) differenzierbar ist und 
        \[ F'(x) = f(x) \; \forall x\in I \setminus A. \]
    \end{enumerate}
\end{defi} 
\begin{satz}
    Sei \( \abb{f}{I}{\C} \) eine Regelfunktion 
    auf Intervall \(I\). Sei \(a \in I\) fest und 
    für \( x\in I \) setzen wir 
    \[ F(x) := \int_a^x f(t) dt. \]
    Dann gilt: 
    \begin{enumerate}
        \item \(F\) ist eine Stammfunktion zu \(f\).
        Genauer: \( F \) ist für jeden Punkt 
        \( x_0 \in I \) sowohl rechts- als auch linksseitig 
        differenzierbar mit 
        \[ F'_-(x_0) = \limesx{t}{0-} 
        \frac{F(x_0 + t) - F(x_0)}{t} 
        = f_-(x_0) = \limesx{x}{x_0-} f(x). \]
        \[ F'_+(x_0) = \limesx{t}{0+} 
        \frac{F(x_0 + t) - F(x_0)}{t} 
        = f_+(x_0) = \limesx{x}{x_0+} f(x). \]
        
        Insbesondere ist \(F\) an jeder Stetigkeitsstelle 
        \( x_0 \) von \(F\) differenzierbar mit 
        \[ F'(x_0) = f(x_0). \]
        \item Für jede beliebige Stammfunktion \(\phi \) 
        zu \(f\) auf \(I\) gilt \( \forall a,b \in I \)
        \[ \integralx{f}{a}{b}{t} = \phi(b)-\phi(a) 
        =: \left[ \phi \right]^b_a = \phi \vert_a^b. \]
    \end{enumerate}
\end{satz}
\begin{bew}
    1.\\
    1. Schritt: Sei \( J \subset I \) ein kompaktes 
    Teilintervall 
    \[ ||f||_J := \underset{x\in J}{\sup} \abs{f(x)} \]
    Dann gilt für alle \( x_1, x_2 \in J: F(x_2)-F(x_1)
    = \integralx{f(t)}{a}{b}{t} \). \\
    Somit ist \(F\) auf \(J\) Lipschitzstetig.\\
    \(\Rightarrow F\) ist stetig auf \(I\) 
    (lokal Lipschitzstetig).
    
    2. Schritt: Sei \(x_0 \in I\) mit 
    \(x_0 + \delta \in I \)
    für \(\delta > 0\)
    \begin{beh}
        \(F_+'(x_0) = f_+(x_0) \).
    \end{beh}
    \begin{bew}
        Sei \(\varepsilon > 0 \) und \(\delta > 0\):
        \[ \abs{f(x)-f(x_0)} < \varepsilon 
        \; \forall x \in (x_0, x_0 + \delta) \]
        \[ \Rightarrow \frac{F(x) - F(x_0)}{x - x_0} 
        - f_+(x_0) 
        = \frac{1}{x - x_0} 
        \integralx{f(t) - f_+(x_0)}{x_0}{x}{t}, 
        \; \forall x \in (x_0, x_0+ \delta)\]
        und deshalb
        \begin{align*}
            &\abs{\frac{F(x)-F(x_0)}{x-x_0} - f_+(x_0)} \\
            &\leq \frac{1}{x-x_0} \cdot 
            \integralx{\abs{f(t)-f_+(x_0)}}{x_0}{x}{t} \\
            &< \frac{ \varepsilon (x - x_0) }{x - x_0} \\
            &= \varepsilon \;\forall x\in (x_0, x_0 + \delta).
        \end{align*}
        \[ \Rightarrow \frac{F(x) - F(x_0)}{x - x_0} 
        = f_+(x_0) = \limesx{x}{x_0+} f(x). \]
        Genauso:
        \[ \limesx{x}{x_0-} \frac{F(x) - F(x_0)}{x - x_0} 
        = f_-(x_0). \]    
        Ist \(f\) stetig in \(x_0\), so folgt 
        \[ F'_-(x_0) = F'_+(x_0) = f(x_0) \]
        (d.\ h.\  \(F\) ist differenzierbar in \(x_0\)).
    \end{bew}

    2.\\
    Nur für \( \abb{f}{I}{\C} \) stetig und 
    \( \abb{\phi}{I}{\C} \) differenzierbar.
    Setze 
    \[ F(x) = \integralx{f(t)}{a}{x}{t} \text{ und }
    h := F - \phi. \]

    \[ \Rightarrow h' = F' - \phi' = f - f = 0 \]
    \[ h \text{ ist konstant.} \]
    \[ \Rightarrow F(b) - \phi(b) = h(b) = h(a) = F(a)- \phi(a)
    = -\phi(a) \]

    \[ \Rightarrow \integralx{f(t)}{a}{b}{t} 
    = F(b) = \phi(b) - \phi(a) 
    =: \phi \vert_a^b. \]
\end{bew}
\end{document}