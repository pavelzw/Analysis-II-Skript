\documentclass[../ana2.tex]{subfiles}

\begin{document}
\setcounter{section}{13}
\section{Extrema}
(lokales) Extremum = (lokales) Minimum oder Maximum.
\begin{defi}
    Sei \(D \subset \R^d, \abb{f}{D}{\R}\) hat in \(x \in D\)
    ein lokales Minimum, falls \(\delta > 0\) existiert, sodass
    \[ \forall y \in B_\delta(x) \cap D: f(y) \geq f(x) \]
    (in einer Umgebung von \(x\) minimal). \\
    Das Minimum heißt strikt (oder isoliert), falls 
    \[ \forall y \in (B_\delta(x) \cap D) \setminus \set{x}: f(y) > f(x) \]
    \( \abb{f}{D}{\R} \) hat in \(x\) ein lokales (isoliertes) Maximum,
    falls \(-f\) in \(x\) ein lokales (striktes, isoliertes) Minimum hat.
\end{defi}
\begin{bsp}
    \(\abb{f}{[-1, 1]}{\R}, x \mapsto x^2\) hat in \(x = 0\) ein striktes Minimum.\\
    \( \abb{f}{[-1,1]\times \R}{\R}, (x_1, x_2) \mapsto x^2 \) hat in 
    \((x_1, x_2) = (0,0)\) ein lokales Minimum (auch in jedem Punkt 
    \((0, x_2), x_2 \in \R \)
\end{bsp}
\begin{satz}
    Sei \(U\subset \R^n\) offen, \(\abb{f}{U}{\R}\) habe ein lokales Extremum
    in \(x \in U\) und sei in \(x\) differenzierbar.
    \[ \Rightarrow Df(x) = 0 \]
    (\(0\) ist die lineare Abbildung, die alles von \(\R^n)\) auf \(0 \in \R\)
    abbildet).\\
    Hatten \(Df(x)[h] = \scalarprod{\nabla f(x)}{h} \)\\
    \( = \sum_{j=1}^n \partial_j f(x) h_j = (\partial_1 f(x), \ldots, \partial_n f(x)) \begin{pmatrix}
        h_1\\
        \vdots\\
        h_n
    \end{pmatrix} \)
    Sei \(\abb{A}{\R^n}{\R}\) linear. \(e_1, \ldots, e_n\) Basis von \(\R^n\).
    \[ h = \sum_{j = 1}^{n} h_j e_j \]
    \[ \Rightarrow A[h] = A[\sum_{j = 1}^{n} h_j e_j ] = \sum_{j=1}^n A[e_j]h_j \]
    \[ = (A[e_1], \ldots, A[e_n]) \begin{pmatrix}
        h_1\\
        \vdots\\
        h_n
    \end{pmatrix} = \scalarprod{v_A}{h} \]
    mit \(v_A := \begin{pmatrix}
        A[e_1]\\
        \vdots\\
        A[e_n]
    \end{pmatrix}\)
    \[ \Rightarrow Df(x) = 0 \Leftrightarrow \nabla f(x) = \begin{pmatrix}
        0\\
        \vdots\\
        0
    \end{pmatrix} \in \R^n \]
    \[ \Leftrightarrow \partial_jf(x) = 0, j=1,\ldots,n \]
\end{satz}
\begin{bew}
    Sei \(x \in D \subset \R^n\) offen \(\Rightarrow \forall h \in \R^n \; 
    \exists \delta_1 > 0: x +th \in D \; \forall \abs{t} < \delta_1, t \in \R^n \) \\
    \[ \abb{g}{(-\delta, \delta)}{\R}, t \mapsto g(t):= f(x+th) \]
    Achtung: \(g\) ist in \(0\) differenzierbar!
    \[ \ddx{t} g(t) \vert_{t=0} = \limesx{t}{0} \frac{g(t)-g(0)}{t} \]
    \[ = \limesx{t}{0} \frac{f(x+th)- f(x)}{t} \]
    \[ = D_hf(x) = Df(x)[h] \; \forall h \in \R^n \]
    \[ \Rightarrow Df(x) = 0 \]
    \[ \Rightarrow 0 = Df(x)[h] = \scalarprod{\nabla f(x)}{h} \]
    Angenommen: \(v \in \R^n, \scalarprod{v}{h} = 0 \; \forall h \in \R^n\)
    \[ \Rightarrow v = 0 = \begin{pmatrix}
        0\\
        \vdots\\
        0
    \end{pmatrix} \]
    \[ \Rightarrow \nabla f(x) = 0 \]
\end{bew}
\begin{bem}
    Punkt \(x \in D\) mit \(\nabla f(x) = 0\) heißt kritischer Punkt.\\
    \( \Rightarrow \) kritische Punkte sind Kandidaten für lineare Extrema.
\end{bem}
\begin{bsp}
    \(f(x_1, x_2) := x_1^2 - x_2^2 \Rightarrow \nabla f(0, 0) = 0\)\\
    aber \(f\) hat in \((0,0)\) kein Extrema (Sattelpunkt).\\
    Angenommen \(g\) ist \(2\)-mal differenzierbar auf \((-\delta, \delta)\)\\
    Ana I \(\rightarrow\) Vorzeichen von \(g''(x)\) ist entscheidend für 
    Maximum oder Minimum.\\
    Ist \(f\) differenzierbar auf \(D\)
    \[ \Rightarrow g'(t) = \ddx{t} f(x+th) = D_hf(x+th) \]
    \[ g'(t) = D_h f(xk+th) \]
    Angenommen\(g'\) ist wieder differenzierbar
    \[ \Rightarrow g''(t) = \ddx{t} (D_hf(x+th)) = D_h(D_hf(x+th)) \]
    oder \( h = \sum_{j=1}^n h_j e_j \)
    \[ \Rightarrow g'(t) = D_hf(x+th) = Df(x+th)[h] \]
    \[ = \sum_{j=1}^n Df(x+th)[e_j]h_j = \sum_{j=1}^n \partial_j f(x+th)h_j \]
    \[ g'(t) = \sum_{j=1}^n \partial_j f(x+th) h_j \]
    Angenommen \( \abb{\partial_j f}{D}{\R} \) ist wieder differenzierbar \(\forall j=1, \ldots, n\)
    \[ \Rightarrow g''(t) = \ddx{t} (g'(t)) = \ddx{t} (\sum_{j=1}^n \partial_j f(x+th)h_j) \]
    \[ = \sum_{j=1}^n  (\ddx{t}(\partial_j f(x+th)))h_j\]
    \[ = \sum_{j=1}^n 
    \underbrace{(D(\partial_j f)(x+th)[h])}_{= \sum-{k=1}^n D(\partial_jf)(x+th)[e_k]h_k} h_j \]
    \[ = \sum_{j=1}^n ( \sum_{k=1}^n \partial_k \partial_j f(x+th)h_k)h_j \]
    \[ = \sum_{j,k = 1}^n \partial_k \partial_j f(x+th)h_k h_j = \scalarprod{h}{H(x+th)h} \]
    \[ \Rightarrow g''(0) = \sum_{j,k = 1}^n \partial_k \partial_j f(x) h_k h_j 
    = \scalarprod{h}{H(x)h} \]
    mit \(H(x) := (\partial_k \partial_j f(x))_{j,k=1,\ldots,n} \) 
\end{bsp}
\begin{defi}
    Sei \(U \subset \R^n\) offen \(\abb{f}{U}{\R^n}\) existieren alle partiellen Ableitungen
    erster und zweiter Ordnung von \(f\), so heißt die obige \(n\times n\) Matrix die Hessematrix 
    von \(f\).
\end{defi}
Wichtig: Sind alle partiellen Ableitungen \(\partial_j f\) und 
\( \partial_k \partial_j f \)
stetig \(\forall j,k = 1, \ldots , n\), so sind die zweiten partiellen Ableitungen symmetrisch
nach dem Satz von Schwarz.
\[ \partial_k \partial_j f = \partial_j \partial_k f \; \forall j,k \]
\(\Rightarrow\) Unter dieser Voraussetzung ist die Hessematrix \(H(x)\) symmetrisch.
\begin{defi}
    1. Sei \(U \subset \R^n\) offen, 
    \(f \in \mathcal{C}^2(U, \R^m)\), falls alle
    partiellen \(\partial_jf, \abb{\partial_k \partial_j}{U}{\R^m}\) stetig auf \(U\) sind.\\
    (\( \mathcal{C}^2(U) := \mathcal{C}^2(U, \R) \)).\\
    2. Ähnlich \(f \in \mathcal{C}^r(U, \R^m)\), 
    falls alle partiellen Ableitungen
    \( \partial_{j_1}f, \partial_{j_2}\partial_{j_1}f, \ldots, 
    \abb{\partial_{j_r}\ldots \partial_{j_1}}{U}{\R^m} \)
    \( j_1, \ldots, j_r = 1, \ldots, n \) stetig auf \(U\) sind.\\
    Ist \(f \in \mathcal{C}^2(U,\R) = \mathcal{C}^2\)
    \[ \Rightarrow H(x) = (\partial_k \partial_j f(x))_{j,l 
    = 1,\ldots, n} \]
    ist symmetrische \(n\times n \) Matrix.
    \[ \Rightarrow b_H(u,v) .= \scalarprod{u}{H(x)v}, u,v \in \R^n \]
    symmetrische Bilinearform.\\
    \( H(x)v = H v \)
\end{defi}
\begin{lem}
    Sei \(U \subset \R^n\) offen, \(f \in \mathcal{C}^2(U)\), 
    dann gilt
    \[ \limesx{u}{0} \frac{f(x+u) - f(x) - Df(x)[h] - \frac{1}{2} 
    \scalarprod{u}{H(x)}}{\abs{u}^2} = 0 \]
\end{lem}
\begin{bew}
    Sei \(u \in \R^n, f \in \mathcal{C}^2(U), x+tu \in U 
    \; \forall 0 \leq t \leq 1\)\\
    \(g(t) = f(x+tu) \) \\
    \[ \Rightarrow g'(t) = Df(x+tu)[u] \text{ und } g''(t) 
    = \scalarprod{u}{H(x+tu)u} \tag{*} \]
    \[ g(1) - g(0) \oversett{HDI}{=} 
    \integralx{\ddx{t}g(t)}{0}{1}{t} \]
    \[ 1 = - \ddx{t}(1-t) = -\integralx{\ddx{t} (1-t)g'(t)}{0}{1}{t} \]
    \[ = - [(1-t)g'(t)]_0^1 + \integralx{(1-t)g''(t)}{0}{1}{t} \]
    \[ = g'(0) + \integralx{(1-t) \underbrace{g''(t)}
    _{= g''(t)-g''(0)+g''(0)}}{0}{1}{t} \]
    \[ g'(0) + \integralx{(1-t)g''(0)}{0}{1}{t} 
    + \integralx{(1-t)(g''(t)- g''(0))}{0}{1}{t} \]
    \[ \Rightarrow g(t) = \underbrace{g(0)}_{f(x)} 
    + \underbrace{g'(0)}_{Df(x)[u]} + \frac{1}{2} 
    \underbrace{g''(0)}_{\scalarprod{u}{H(x)u}} + \integralx{(1-t)
    \underbrace{(g''(t)-g''(0))}_{=\scalarprod{u}{H(x+tu)u} 
    - \scalarprod{u}{H(x)u}\\
    =\scalarprod{u}{(H(x+tu) - H(x))u}}}{0}{1}{t} \]
    \[ \Rightarrow f(x+u) - f(x)- Df(x)[u] - \frac{1}{2} 
    \scalarprod{u}{H(x)u} \]
    \[ = \integralx{(1-t)(\underbrace{\scalarprod{u}{(H(x+tu) 
    - H(x))u}}
    _{\scalarprod{u}{Au} \leq \abs{u}\abs{Au} \leq 
    \norm{A}\abs{u}^2 }}{0}{1}{t} \]
    \[ \leq \abs{u^2} \underbrace{\norm{H(x+tu)-H(x)}}
    _{\rightarrow 0 (u \rightarrow 0)} \]
    \begin{align*}
        \Rightarrow &\frac{\abs{f(x+v) - f(x)- Df(x)[v] - \frac{1}{2}
        \scalarprod{v}{H(x)v}}}{\abs{v}^2}\\
        &= \frac{1}{\abs{v}^2} \abs{\integralx{ (1-t) 
        \scalarprod{v}{ (H(x + tv) - H(x))v } }{0}{1}{t}} \\
        &\leq \abs{v}^{-2} \integralx{ (1-t) 
        \abs{\scalarprod{v}{(H(x + tv) - H(x))v  }} }{0}{1}{t}\\
        &\leq \abs{v}^{-2} \integralx{ (1-t) 
        \abs{v}\abs{(H(x + tv) - H(x))v  } }{0}{1}{t}\\
        &\leq \abs{v}^{-2} \integralx{ (1-t) 
        \abs{v}\abs{(H(x + tv) - H(x))}\abs{v} }{0}{1}{t}\\
        &\leq \abs{v}^{-2} \integralx{(1-t) \abs{v}^2 \norm{H(x+tv) 
        - H(x)}}{0}{1}{t}\\
        &= \integralx{(1-t) \norm{H(x+tv) - H(x)}}{0}{1}{t}
    \end{align*}
    z.B. \(A n \times n \) Matrix \( \norm{A} \leq \norm{A}_{HS} 
    = (\sum_{l,m = 1}^n (
    \underbrace{H_{lm}(x+tv)-H_{lm}}_{\rightarrow 0 (v \rightarrow 0)}
    )^2)^{1/2}\)\\
    \( H_{lm}(x) = \partial_l \partial_m f(x) \) stetig in \(x\).\\
    Sei \(\varepsilon > 0 \) und \(\delta > 0: 
    \abs{H_{lm}(x+u)-H_{lm}(x)} \leq \varepsilon 
    \; \forall u \in \R^n, \abs{u}\leq \delta 
    \; \forall l,m = 1,\ldots,n\)
    Ist \( \abs{v} \leq \delta, 0 \leq t \leq 1 \Rightarrow 
    \abs{tv} = t\abs{v} \leq \delta \).\\
    \( \Rightarrow \abs{ H_{lm}(x + tv) - H_{lm}(x) } \leq \varepsilon 
    \forall v \in \R^n, \abs{v} \leq \delta \) und alle \( 0 \leq t \leq 1, 
    l,m = 1,\ldots,n \).
    \[ \Rightarrow \norm{H(x+tv)-H(x)}_{HS} \leq (\sum_{l,m = 1}^n
    \varepsilon)^{1/2} = n \varepsilon 
    \; \forall 0 \leq t \leq 1, \abs{v} \leq \delta \]    
    \[ \Rightarrow \integralx{ (1-t) \norm{H(x + tv) - H(x)} }{0}{1}{t} 
    \leq n \varepsilon \integralx{(1-t)}{0}{1}{t} = \frac{n\varepsilon}{2} 
    \;\forall \abs{v} \leq \delta \]
    \[ \Rightarrow \limesx{v}{0} 
    \integralx{(1-t) \norm{ H(x+tv) - H(x) }}{0}{1}{t} = 0. \]
\end{bew}
\begin{defi}
    Sei \( \abb{b}{\R^n \times \R^n}{\R} \) 
    eine symmetrische Bilinearform, z.B. 
    \( b(u,v) = \scalarprod{v}{Hu}, H \)
    eine symmetrische \( n\times n \)-Matrix.\\
    Wir nennen \( b \) positiv semidefinit
    (besser positiv), falls \( b(u,u) \geq 0 
    \;\forall u \in \R^n \), \\
    \(b\) heißt positiv definit (besser strikt positiv), 
    falls \( b(u,u) > 0 \;\forall u \in \R^n \setminus \set{0} \),\\
    \(b\) negativ semidefinit (besser negativ),
    falls \( b(u,u) \leq 0 \;\forall u \in \R^n \),\\
    negativ definit (besser strikt negativ), falls 
    \( b(u,u) < 0 \; \forall u \in \R^n \setminus \set{0} \)\\
    \(b\) heißt indefinit, falls \( u_1, u_2 \in \R^n \) 
    existieren mit 
    \( b(u_1, u_1) > 0, b(u_2, u_2) < 0 \).
\end{defi}
\begin{satz}
    Sei \(U \subset \R^n\) offen, \(f \in \mathcal{C}^2(U,\R)\), 
    \( H = H(x) \) Hessematrix von \(f\) in \(x\).\\
    \(f\) hat genau dann in \(x\in U\) ein lokales Minimum, wenn
    \( \nabla f(x) = 0 \) und \( b(v,v) := \scalarprod{v}{H(x)v} \)
    positiv semidefinit.
\end{satz}
\begin{bem}
    Analog für Maximum, ersetze \(f\) durch \(-f\).
\end{bem}
\begin{bew}
    \gqq{\(\Rightarrow\)}: Hat \(f\) in \(x\) ein lokales Minimum \(\Rightarrow \nabla f(x) = 0\)
    (Satz 2)\\
    \( H(x) = \) Hessematrix von \(f\) in \(x\).
    \[ \oversett{Lem. 5}{\Rightarrow}
    \limesx{v}{0} \frac{f(x+v) - f(x) - Df(x)[v] - \frac{1}{2} 
    \scalarprod{v}{H(x)v}}{\abs{v}^2} = 0 \]
    \(t \neq 0:\)
    \[ \frac{\scalarprod{v}{H(x)v}}{\scalarprod{v}{v}} 
    = \frac{\scalarprod{tv}{H(x)tv}}{\scalarprod{tv}{tv}}
    = \limesx{t}{0} \frac{ \scalarprod{tv}{H(x)tv} }{\abs{tv}^2} \]
    mit Lemma 5: \( f(x+v) - f(x)-Df(x){v} = \frac{1}{2}
    \scalarprod{v}{H(x)v} + \eta(\abs{v})\abs{v}^2 \), mit
    \( \limesx{v}{0} \eta(v) = 0 \)
    \[ \Rightarrow f(x+v)-f(x) = \frac{1}{2} \scalarprod{v}{Hv}
    + \eta(v)\abs{v}^2 \]
    \[ \frac{\scalarprod{v}{H(x)v}}{\scalarprod{v}{v}} 
    = \frac{\scalarprod{tv}{H(x)tv}}{\scalarprod{tv}{tv}}
    = \limesx{t}{0} \frac{ \scalarprod{tv}{Htv} }{\abs{tv}^2} \]
    \[ \oversett{Lem. 5}{=} \limesx{t}{0} 
    \frac{f(x+tv)-f(x)}{\scalarprod{tv}{tv}} \geq 0 \]
    \[ \Rightarrow \scalarprod{v}{Hv} \geq 0 \; \forall v \in \R^n \]
    \gqq{\(\Leftarrow\)}: Angenommen \( \nabla f(x) = 0, 
    \scalarprod{v}{H(x)v} = 0 \;\forall v\in \R^n \).\\
    \begin{beh}
        \( \exists \delta > 0: \scalarprod{v}{H(x)v} \geq \delta \abs{v}^2 \;\forall v\in \R^n \).
    \end{beh}
    \begin{align*}
        \underset{\substack{v\in\R^n\\v \neq 0}}{\inf} 
        \frac{\scalarprod{v}{Hv}}{\scalarprod{v}{v}}
        = \underset{v \in \R^n \setminus \set{0}}{\inf} 
        \frac{\scalarprod{v}{Hv}}{\abs{v}^2}
        = \underset{\abs{v} = 1}{\inf} \scalarprod{v}{Hv}
    \end{align*}
    a) \( \R^n \ni v \mapsto \scalarprod{v}{Hv} \) ist stetig.\\
    b) \( S^{n-1} := \set{v \in \R^n; \abs{v} = 1} \) ist eine
    kompakte Teilmenge von \(\R^n\)\\    
    \( \Rightarrow \exists v \in S^{n-1} \)
    \[ \underset{\abs{v} = 1}{\inf} \scalarprod{v}{Hv} 
    = \underbrace{\scalarprod{v_1}{Hv_1}}_{=:\delta} > 0. \]    
    \( \Rightarrow \) das Infimum existiert.
    \[ \oversett{Lem 5}{\Rightarrow} f(x+u) 
    = f(x) + \frac{1}{2} \scalarprod{u}{Hu} + \eta(u) \abs{u}^2 
    \geq f(x) + (\frac{\delta}{2} + \eta(u))\abs{u}^2 \]
    \( \eta(u) \rightarrow 0, u \rightarrow 0 \).
    \( \Rightarrow \exists r > 0: \abs{\eta(u)} \geq - \frac{\delta}{4} \;\forall \abs{u} \leq r \).
    \[ \Rightarrow \forall \abs{u} \leq r: f(x+u) \geq f(x)
    + \frac{\delta}{4}\abs{u}^2\]
    \(\Rightarrow\) isoliertes Minimum in \(x\).
\end{bew}
\(H\) symmetrische Matrix \( \oversett{LA}{\Rightarrow} \exists \)
Orthonormalbasis \( u_1,\ldots,u_n \) an 
Eigenvektoren von \(H\), \( H u_j = \lambda_j u_j \)
\[ \Rightarrow v = \sum_{j=1}^n x_j u_j \in \R^n \]
\begin{align*}
    \scalarprod{v}{Hv} &= 
    \scalarprod{\sum_{j=1}^n x_j u_j}{H(\sum_{j=1}^n x_j u_j)}\\
    &= \sum_{j=1}^n \sum_{k=1}^n x_j x_k \scalarprod{u_j}{Hu_k}\\
    &= \sum_{j=1}^n \sum_{k=1}^n x_j x_k \lambda_k \scalarprod{u_j}{u_k}\\
    &= \sum_{j=1}^n \sum_{k=1}^n x_j x_k \lambda_k \delta_{jk} \\
    &= \sum_{j=1}^n \lambda_j x_j^2
\end{align*}
\(\Rightarrow\) Eigenwerte \(\lambda_j\) von \(H\) bestimmen ob
\( \scalarprod{v}{Hv} \) partiell (demi)definit oder anders ist.\\
Analytischer Beweis des Spektraltheorems für symmetrische \(n \times n\)
Matrizen.\\
Sei \(A\) relle symmetrische \(n \times n\) Matrix.\\
1. Schritt:\\ 
Rayleighquotient: \(x \in \R^n \setminus \set{0}, R(x) 
= \frac{\scalarprod{x}{Ax}}{\scalarprod{x}{x}}\) \\
\( \R^n \setminus \set{0} \ni x \mapsto R(x) \) homogen von Grad \(0\).\\
\( R(tx) = R(x) \; \forall t \neq 0, x \neq 0 \)
\[ \Rightarrow \alpha := 
\underset{x \in \R^n\setminus \set{0}}{\sup}
R(x) = \underset{x\in\R^n \setminus \set{0}}{\sup} 
R(\frac{x}{\abs{x}}) 
= \underset{y\in S^{n-1}}R(y). \]
Supremum von stetiger Funktion auf kompakten Mengen werden 
angenommen.
\( \Rightarrow \exists u \in S^{n-1}: \alpha = R(u) \), 
d. h. \(R\) hat ein Maximum.\\
2. Schritt:
\[ \partial_j R(x) = \partial_j 
\frac{\scalarprod{x}{Ax}}{\scalarprod{x}{x}} 
= \frac{ \scalarprod{x}{x} \partial_j \scalarprod{x}{Ax} 
- \scalarprod{x}{Ax} \partial_j 
\scalarprod{x}{x} }{\scalarprod{x}{x}^2} \]
\[ \partial_j \scalarprod{x}{x}  = \partial_j \sum_{k = 1}^n x_k^2
= \ddxpartial{x_j} \sum_{k=1}^n x_k^2 = 2x_j \]
\[ \partial_j \scalarprod{x}{Ax} 
= \partial_j \sum_{l=1}^n \sum_{k=1}^n A_{lm}x_k 
= \sum_{k=1}^n A_{jm} x_m + \sum_{l=1}^n x_lA_{lj} \]
\[ = 2\sum_{l=1}^n A_{jl} x_l 
= 2(Ax)_j \]
\[ \Rightarrow \partial_j R(x) 
= \frac{\abs{x}^2 (Ax)_j - \scalarprod{x}{Ax}2x_j}{\abs{x}^4} 
\text{ stetig in } x\neq 0. \]
\( \oversett{Satz 2}{\Rightarrow} x=u \) ist Maximum, d. h. \( \nabla R(x) = 0 \).
\( \partial_j R(u) = 0 \;\forall j = 1,\ldots,n \).
\[ \Rightarrow \abs{u}^2 2(Au)_j = \scalarprod{u}{Au}2u_j \]
oder \( \abs{u}^2 Au = \scalarprod{u}{Au} u \)
\[ \Rightarrow Au = \frac{\scalarprod{u}{Au}}{\abs{u}^2}u = \lambda u, 
\lambda = \frac{\scalarprod{u}{Au}}{\abs{u}^2} \in \R. \]
\[ x \in \R^n, x = x_1 u + \text{ Rest} = x_1 u + x^\bot \]
\(V := \mathrm{Span}(u) = \set{tu : t \in \R}\)\\
\( V^\bot := \set{y \in \R^n: \scalarprod{u}{y} = 0} \)\\
\( \R^n = V \oplus V^+ \)
\[ \Rightarrow Ax = x_1\underbrace{Au}_{=\lambda u} + Ax^\bot 
= \underbrace{\lambda x_1 u}_{\in V} 
+ \underbrace{A(x^\bot)}_{\in V^\bot} \]
Denn: ist \( v \in V^\bot: \scalarprod{u}{Av} = \scalarprod{Au}{v} 
= \lambda \scalarprod{u}{v} = 0 \).\\
Spalten wir \( \R^n = V \oplus V^\bot \) auf.\\
\( A \) Blockstatur \gqq{\(=\)} 
\[ \begin{pmatrix} \lambda_1 & 0\\ 0 & A\vert_{V^\bot} \end{pmatrix} 
\begin{pmatrix} x_1 \\ x^\bot \end{pmatrix} \]
letztes Mal: \( \abb{f}{U}{\R}, \mathcal{C}^2 \) Funktion.
\(\Rightarrow\)
\begin{enumerate}
    \item Hat \(f\) in \(x \in U\) ein lokales Minimum
    \[ \Rightarrow \nabla f(x) = 0 \] 
    und \(H(x)\) ist positiv semidefinit.
    \item Ist \( \nabla f(x) = 0 \) 
    und \(H(x)\) positiv definit (\(> 0\))
    \( \Rightarrow x \) ist eine isoliertes Minimum.
\end{enumerate}
Analog für Maxima.\\
Tod: 
\[ f(x + v) = f(x) + Df(x)[v] + \frac{1}{2} \scalarprod{v}{H(x)v} 
+ \integralx{ (1 - t) (\scalarprod{v}{H(x + tv)v} 
- \scalarprod{v}{H(x)v}) }{0}{1}{t}. \]
Die erste Beobachtung:
\[ \R^n \setminus \set{X} \ni x \mapsto R(x) \]
hat ein Maximum.
\[ R(x) = \frac{\scalarprod{x}{Ax}}{\abs{x}^2} 
= \scalarprod{\frac{x}{\abs{x}}}{A\frac{x}{\abs{x}}} = R(\frac{x}{\abs{x}}). \]
Zweite Beobachtung:
Ist \(x \neq 0\) ein kritischer Punkt von \(R\), dann ist 
\(x\) eine Eigenvektor von \(A\).
\begin{bew}
    Sei \(x \neq 0\) kritischer Punkt von \(R \Rightarrow D R(x) = 0\)
    \[ \Rightarrow D_h R(x) = 0 \; \forall h \in \R^n \]
    \[ 0 = D_hR(x) = D_h \frac{\scalarprod{x}{Ax}}{\scalarprod{x}{x}}
    = \frac{ \scalarprod{x}{x} D_h(\scalarprod{x}{Ax}) 
    - \scalarprod{x}{Ax} D_h(\scalarprod{x}{x}) }{\scalarprod{x}{x}^2} \]
    \[ D(\scalarprod{x}{Ax}) = \ddxpartial{t} \scalarprod{x+th}{A(x+th)} \vert_{t=0} \]
    \[ = \scalarprod{x}{Ah} + \scalarprod{h}{Ax} = 2 \scalarprod{h}{Ax} \]
    \[ D_h(\scalarprod{x}{x}) = 2 \scalarprod{h}{x} \]
    \[ \Rightarrow 0 = \abs{x}^2 2 \scalarprod{h}{Ax} - \scalarprod{x}{Ax}
    2 \scalarprod{h}{x} \]
    \[ \Rightarrow \scalarprod{h}{Ax} = \frac{\scalarprod{x}{Ax}}{\abs{x}^2}
    \scalarprod{h}{x} = \scalarprod{h}{\frac{\scalarprod{x}{Ax}}{\abs{x}^2}x} \; \forall h \in \R^n  \]
    \[ \Rightarrow Ax = \frac{\scalarprod{x}{Ax}}{\abs{x}^2}x = R(x)x = 2x \]
    \( \Rightarrow A\) ist symmetrisch \(n \times n\) Matrix hat Eigenwert 
    \(\lambda = \underset{x \neq 0}{\sup} R(x)\) und Eigenvektor \(u_1 \neq 0\),
    o.B.d.A \(\abs{u} = 1\).    
    \begin{align*}
        V_1 = \mathrm{Span}(u_1) = \set{zx, z \in \R} \text{ Linie in }\R^n
        V_1^\bot = \text{ orthogonales Komplement von } V_1 
        = \set{y \in \R^n: \scalarprod{u}{y} = 0}
    \end{align*}
    \( P = \) orthogonale Projektion auf \(V_1 = |u_1 \rangle \langle u_1 | 
    = u_1 \scalarprod{u_1}{\cdot} \)
    \(Px = u_1 \scalarprod{u_1}{x}\)\\
    \( \mathds{Q} = \mathds{1} - P = \) Projektion auf \(V_1^\bot\).\\
    \(P + \mathds{Q} = \mathds{1}\) auf \(\R^n\).    
    \begin{align*}
        x = (P + \mathds{Q})X &= Px + \mathds{Q}x \\
        &= u_1 \scalarprod{u_1}{x} + \mathds{Q}x \\
        &= x_1 u_1 + \mathds{Q}x
    \end{align*}
    \[ \Rightarrow \mathds{Q} A x 
    = \lambda_1 x_1 \mathds{Q} u_1 + \mathds{Q} A \mathds{Q} x
    = \mathds{Q}A\mathds{Q}x \]
    Insbesondere ist \(x \in V_1^\bot \Leftrightarrow \mathds{Q}x = x\)\\
    \( A \mathds{Q}x = ? \)
    \[ \scalarprod{u_1}{A\mathds{Q}x} = \scalarprod{Au_1}{\mathds{Q}}
    = \lambda_1 \scalarprod{u_1}{\mathds{Q}x} = 0\]    
    \[ \Rightarrow A\mathds{Q}x \in V_1^\bot \;\forall x \in \R^n. \]
    \[ \Rightarrow Ax = \lambda_1 x_1 u_1 + A\mathds{Q}x
    = \lambda_1 x_1 u_1 + \mathds{Q}A\mathds{Q}x \]
    \[ \Rightarrow x = x_1 u_1 + x^\bot (x^\bot = \mathds{Q}x) \]
    \[ \Rightarrow Ax = \lambda_1 x_1 u_1 + \mathds{Q}A\mathds{Q} x^\bot \]
    \[ \abb{\mathds{Q} A \mathds{Q}}{V_1^\bot}{V_1^\bot} \]
    mit \(\dim V_1^\bot = n-1\).
    \[ x = x_1 u_1 + x^\bot \text{ \gqq{\(=\)} } 
    \begin{pmatrix} x_1 \\ x^\bot \end{pmatrix} \]
    \[ A(x_1u_1+x^\bot) \text{ \gqq{\(=\)} } A \begin{pmatrix}
        x_1\\
        x^\bot
    \end{pmatrix} = \begin{pmatrix}
        \lambda_1x_1\\
        \mathds{Q}A\mathds{Q}x^\bot
    \end{pmatrix} = \begin{pmatrix}
        \lambda_1 & 0\\
        0 & \mathds{Q}A\mathds{Q}
    \end{pmatrix} \begin{pmatrix}
        x_1\\
        x^\bot
    \end{pmatrix} \]
    \( U = (u_1, u_2, \ldots, u_n) \) ist unitäre \(n \times n\) Matrix.
    \[ AU = \begin{pmatrix}
        Au_1 & Au_2 & \cdots & A u_n
    \end{pmatrix} 
    = \begin{pmatrix}
        \lambda u_1 & \cdots & \lambda u_n
    \end{pmatrix} \]
\end{bew}
\end{document}