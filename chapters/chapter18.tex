\documentclass[../ana2.tex]{subfiles}
\begin{document}
\setcounter{section}{17}
\section{Extrema mit Nebenbedingungen}
\begin{defi}
    Sei \( 1 \leq m \leq n \), eine Menge \( M \subset \R^n \) 
    heißt ein-dimensionale Untermannigfaltigkeit 
    (U-Mfk) des \( \R^n \) der Klasse \( \mathcal{C}^r \) 
    (\(r \in \N_0 \cup \set{\infty} \)) falls zu jedem 
    \( p \in M \) eine offene Menge \( U \subset \R^n \) 
    und ein \( \mathcal{C}^r \)-Diffeomorphismus
    \(\abb{\Phi}{U}{\Phi(U)}\) existiert mit 
    \[ \Phi(M \cap U) = (\R^m \times \set{0}) \cap \Phi(U)\]
    lokale Plättung.
\end{defi}
\begin{satz}[Untermannigfaltigkeitskriterium]
    Sei \( M \subset \R^n, m + k = n \). Dann sind äquivalent
    \begin{enumerate}
        \item \(M\) ist \(\mathcal{C}^r \) Untermannigfaltigkeit.        
        \item Niveaumengenkriterium: Zu \( p \in M \exists \) offene 
        Umgebung \( U \subset \R^n \) mit 
        \( f \in \mathcal{C}^r(U, \R^k) \), sodass
        \[ M \cap U = f^{-1}(0) \]
        und
        \[ \rk Df = k \text{ auf } U \]
        \item Graphenkriterium zu \(p \in M \; \exists \) offene
        Umgebung \(U \times V \in \R^m \times \R^k = \R^n\) und
        \(g \in \mathcal{C}^r (U, V)\) so, dass
        nach einer Permutuation der Koordinaten gilt \(M \cap (U \times V) 
        = \set{x,g(x) : x \in U}\)
    \end{enumerate}    
\end{satz}
\begin{bew}
    Übung (oder Skript)
\end{bew}
\begin{bsp}
    Späre \(S^m = \set{x \in \R^{m+1}: \abs{x} = 1}\) in dem
    \(\mathcal{C}^\infty\) Untermannigfaltigkeit im \(\R^{m+1}\).\\
    Z. B. \(\abb{f}{\R^{m+1} \setminus \set{0}}{\R}; x \mapsto \abs{x}^2 - 1 \) 
    in \(\mathcal{C}^\infty\)\\
    \( S^m = f^{-1}(0) \)
\end{bsp}
\begin{defi}
    \(E\) in Vektor \(v \in \R^n\) heißt Tangentialvektor von \(M \subset R^n\)
    im Punkt \(p \in M\), falls
    es eine Abbildung \( \abb{\gamma}{(-\varepsilon, \varepsilon)}{M} \) mit 
    \[ \gamma(0) = p, \quad \gamma'(0) = v \]
    Schreiben 
    \( T_p M = \) Menge aller Tangentialvektoren von \(M\) im Punkt \(p\).
\end{defi}
\begin{lem}
    Sei \( M \in \R^n \) eine \(m\)-dimensionale 
    \( \mathcal{C}^1 \) Untermannigfaltigkeit  und \(n = m + k\).
    Ist \( p \in M \cap U = f^{-1}(0) \) für eine
    Funktion \( f \in \mathcal{C}^1(U, \R^k) \) mit 
    \( \rk Df = k \) auf \(U\),
    so gilt 
    \[ T_p M = \rk(Df(p))\].
    Insbesondere ist \( T_p M \) ein \(m\)-dimensionaler 
    Vektorraum.
\end{lem}
\begin{bew}
    Für \(\abb{\gamma}{(-\varepsilon, \varepsilon)}{M}\)
    \[ \gamma(0) = p, \gamma'(0) = w \]
    \[ f(\gamma(t)) = 0 \Rightarrow 0 = \ddx{t}f(\gamma(t))\vert_{t=0} \]
    \[ Df(\gamma(0))[\gamma'(0)] = Df(p)[w] \]
    \[ \Rightarrow \forall w \in T_p M: Df(p)[w] = 0 \]
    \[ \Rightarrow T_pM \subset \ker Df(p) \]
    Nach Satz 2 gibt es, eventuell nach Permutation der Koordianten, offene Mengen
    \(U \subset \R^m, V \subset \R^k, p \in U \times V\) und
    \(g \in \mathcal{C}^1(U, V) \) mit
    \[ M \cap (U \times V) = \set{(x, g(x))^t, x \in U} \]
    Graphenabbildung \( G \in \mathcal{C}^1(U, \R^n) \)
    \[ G(x) := \begin{pmatrix}
        x \\ g(x) 
    \end{pmatrix} \in M \]
    bildet nach \(M\) ab.\\
    Schreiben \(p = \begin{pmatrix}
        x_0 \\
        g(x_0)
    \end{pmatrix} \)\\
    Sei \(v \in \R^m\)
    \[ DG(x_0)[v] = \ddx{t} G(x_0+tv) \vert_{t=0} \in T_pM \]
    \[ \Rightarrow \Bild DG(x_0) \subset T_p M\]
    \[ \Rightarrow \Bild DG(x_0) \subset T_pM \subset \ker Df(p) \]
    \begin{align*}
        \dim \ker Df(p) &= n - \dim \Bild Df(p) \\
        &= n -k = m
    \end{align*}
    Beachte: 
    \[DG(x_0) = \begin{pmatrix}
        v \\
        D g(x_0)[v]
    \end{pmatrix}  \]
    also falls \( DG(x_0)[v_1] 
    = D G(x_0)[v_2] \Leftrightarrow v_1 = v_2 \)
    \(\Rightarrow D G(x_0) \) ist injektiv!
    \(\overundersett{Dimensions-}{formel}{\Rightarrow} 
    \dim \Bild DG(x_0) = \ker Df(p) = T_pM\)
    \( \abb{DG}{\R^m}{\R^{m+n}} \)
\end{bew}
\begin{satz}[Extrema mit Nebenbedingungen]
    Sei \( U \subset \R^n \) offen, \( f \in \mathcal{C}^1(U, \R^k) \) 
    (\(k\) Nebenbedingungen) und \(\phi \in \mathds(C)^1(U, \R)\)
    Gilt dann für ein \(p \in f`^{-1}(0)\)
    \begin{enumerate}
        \item \( \varphi(q) \geq \varphi(p) \;\forall q \in U \) mit
        \( f(q) = 0 \) (lokales Minimum unter Nebenbedingungen)
        \item \( \rk Df (p) = k \) (maximaler Rang)
    \end{enumerate}
    Dann gibt es \( \lambda_1, \ldots, \lambda_k \in \R \) (Lagrange Multiplikatoren)
    mit 
    \[ \nabla \varphi(p) = \sum_{j=1}^k \lambda_j \nabla f_j(p). \]
\end{satz}
\begin{bew}
    Nach Verkleinern von \(U\) ist \(\rk Df = k \) 
    auf \(U\) und \(M = f^{-1}(0)\) eine
    \(m=n-k dim \mathcal(C)^1\) Untermannigfaltigkeit.
    \[ \abb{\gamma}{(-\varepsilon, \varepsilon)}{M}, \gamma(0) = p, \gamma'(0) = v \in T_pM \]
    \( \varphi \circ \gamma \) hat in \( t = 0 \) ein lokales Minimum.
    \[ \Rightarrow 0 = \ddx{t} \varphi(\gamma(t)) \vert_{t=0} 
    = D\varphi(p) [\gamma'(0)] = D\varphi(p)[v] 
    = \scalarprod{\nabla \varphi(p)}{v} \]
    \[ \Rightarrow \forall v \in T_pM \text{ ist } 
    \scalarprod{\nabla \varphi(p)}{v} = 0 \]
    \[ \Rightarrow \nabla \varphi(p) \in (T_pM)^\bot \]    
    \[ \Rightarrow \nabla f_j(p) \in (T_p M)^\bot, j = 1,\ldots,k. \]
    \[ \dim(T_pM)^\bot = n-m = k \]
\end{bew}
\end{document}